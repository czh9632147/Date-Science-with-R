% Options for packages loaded elsewhere
\PassOptionsToPackage{unicode}{hyperref}
\PassOptionsToPackage{hyphens}{url}
%
\documentclass[
]{article}
\usepackage{amsmath,amssymb}
\usepackage{iftex}
\ifPDFTeX
  \usepackage[T1]{fontenc}
  \usepackage[utf8]{inputenc}
  \usepackage{textcomp} % provide euro and other symbols
\else % if luatex or xetex
  \usepackage{unicode-math} % this also loads fontspec
  \defaultfontfeatures{Scale=MatchLowercase}
  \defaultfontfeatures[\rmfamily]{Ligatures=TeX,Scale=1}
\fi
\usepackage{lmodern}
\ifPDFTeX\else
  % xetex/luatex font selection
\fi
% Use upquote if available, for straight quotes in verbatim environments
\IfFileExists{upquote.sty}{\usepackage{upquote}}{}
\IfFileExists{microtype.sty}{% use microtype if available
  \usepackage[]{microtype}
  \UseMicrotypeSet[protrusion]{basicmath} % disable protrusion for tt fonts
}{}
\makeatletter
\@ifundefined{KOMAClassName}{% if non-KOMA class
  \IfFileExists{parskip.sty}{%
    \usepackage{parskip}
  }{% else
    \setlength{\parindent}{0pt}
    \setlength{\parskip}{6pt plus 2pt minus 1pt}}
}{% if KOMA class
  \KOMAoptions{parskip=half}}
\makeatother
\usepackage{xcolor}
\usepackage[margin=1in]{geometry}
\usepackage{color}
\usepackage{fancyvrb}
\newcommand{\VerbBar}{|}
\newcommand{\VERB}{\Verb[commandchars=\\\{\}]}
\DefineVerbatimEnvironment{Highlighting}{Verbatim}{commandchars=\\\{\}}
% Add ',fontsize=\small' for more characters per line
\usepackage{framed}
\definecolor{shadecolor}{RGB}{248,248,248}
\newenvironment{Shaded}{\begin{snugshade}}{\end{snugshade}}
\newcommand{\AlertTok}[1]{\textcolor[rgb]{0.94,0.16,0.16}{#1}}
\newcommand{\AnnotationTok}[1]{\textcolor[rgb]{0.56,0.35,0.01}{\textbf{\textit{#1}}}}
\newcommand{\AttributeTok}[1]{\textcolor[rgb]{0.13,0.29,0.53}{#1}}
\newcommand{\BaseNTok}[1]{\textcolor[rgb]{0.00,0.00,0.81}{#1}}
\newcommand{\BuiltInTok}[1]{#1}
\newcommand{\CharTok}[1]{\textcolor[rgb]{0.31,0.60,0.02}{#1}}
\newcommand{\CommentTok}[1]{\textcolor[rgb]{0.56,0.35,0.01}{\textit{#1}}}
\newcommand{\CommentVarTok}[1]{\textcolor[rgb]{0.56,0.35,0.01}{\textbf{\textit{#1}}}}
\newcommand{\ConstantTok}[1]{\textcolor[rgb]{0.56,0.35,0.01}{#1}}
\newcommand{\ControlFlowTok}[1]{\textcolor[rgb]{0.13,0.29,0.53}{\textbf{#1}}}
\newcommand{\DataTypeTok}[1]{\textcolor[rgb]{0.13,0.29,0.53}{#1}}
\newcommand{\DecValTok}[1]{\textcolor[rgb]{0.00,0.00,0.81}{#1}}
\newcommand{\DocumentationTok}[1]{\textcolor[rgb]{0.56,0.35,0.01}{\textbf{\textit{#1}}}}
\newcommand{\ErrorTok}[1]{\textcolor[rgb]{0.64,0.00,0.00}{\textbf{#1}}}
\newcommand{\ExtensionTok}[1]{#1}
\newcommand{\FloatTok}[1]{\textcolor[rgb]{0.00,0.00,0.81}{#1}}
\newcommand{\FunctionTok}[1]{\textcolor[rgb]{0.13,0.29,0.53}{\textbf{#1}}}
\newcommand{\ImportTok}[1]{#1}
\newcommand{\InformationTok}[1]{\textcolor[rgb]{0.56,0.35,0.01}{\textbf{\textit{#1}}}}
\newcommand{\KeywordTok}[1]{\textcolor[rgb]{0.13,0.29,0.53}{\textbf{#1}}}
\newcommand{\NormalTok}[1]{#1}
\newcommand{\OperatorTok}[1]{\textcolor[rgb]{0.81,0.36,0.00}{\textbf{#1}}}
\newcommand{\OtherTok}[1]{\textcolor[rgb]{0.56,0.35,0.01}{#1}}
\newcommand{\PreprocessorTok}[1]{\textcolor[rgb]{0.56,0.35,0.01}{\textit{#1}}}
\newcommand{\RegionMarkerTok}[1]{#1}
\newcommand{\SpecialCharTok}[1]{\textcolor[rgb]{0.81,0.36,0.00}{\textbf{#1}}}
\newcommand{\SpecialStringTok}[1]{\textcolor[rgb]{0.31,0.60,0.02}{#1}}
\newcommand{\StringTok}[1]{\textcolor[rgb]{0.31,0.60,0.02}{#1}}
\newcommand{\VariableTok}[1]{\textcolor[rgb]{0.00,0.00,0.00}{#1}}
\newcommand{\VerbatimStringTok}[1]{\textcolor[rgb]{0.31,0.60,0.02}{#1}}
\newcommand{\WarningTok}[1]{\textcolor[rgb]{0.56,0.35,0.01}{\textbf{\textit{#1}}}}
\usepackage{graphicx}
\makeatletter
\def\maxwidth{\ifdim\Gin@nat@width>\linewidth\linewidth\else\Gin@nat@width\fi}
\def\maxheight{\ifdim\Gin@nat@height>\textheight\textheight\else\Gin@nat@height\fi}
\makeatother
% Scale images if necessary, so that they will not overflow the page
% margins by default, and it is still possible to overwrite the defaults
% using explicit options in \includegraphics[width, height, ...]{}
\setkeys{Gin}{width=\maxwidth,height=\maxheight,keepaspectratio}
% Set default figure placement to htbp
\makeatletter
\def\fps@figure{htbp}
\makeatother
\setlength{\emergencystretch}{3em} % prevent overfull lines
\providecommand{\tightlist}{%
  \setlength{\itemsep}{0pt}\setlength{\parskip}{0pt}}
\setcounter{secnumdepth}{-\maxdimen} % remove section numbering
\ifLuaTeX
  \usepackage{selnolig}  % disable illegal ligatures
\fi
\usepackage{bookmark}
\IfFileExists{xurl.sty}{\usepackage{xurl}}{} % add URL line breaks if available
\urlstyle{same}
\hypersetup{
  pdftitle={Ch4},
  pdfauthor={CZH},
  hidelinks,
  pdfcreator={LaTeX via pandoc}}

\title{Ch4}
\author{CZH}
\date{2024-07-04}

\begin{document}
\maketitle

\subsection{Explore the gapminder
data.frame}\label{explore-the-gapminder-data.frame}

\begin{Shaded}
\begin{Highlighting}[]
\FunctionTok{library}\NormalTok{(gapminder)}
\FunctionTok{library}\NormalTok{(tidyverse)}
\end{Highlighting}
\end{Shaded}

\begin{verbatim}
## -- Attaching core tidyverse packages ------------------------ tidyverse 2.0.0 --
## v dplyr     1.1.4     v readr     2.1.5
## v forcats   1.0.0     v stringr   1.5.1
## v ggplot2   3.5.1     v tibble    3.2.1
## v lubridate 1.9.3     v tidyr     1.3.1
## v purrr     1.0.2     
## -- Conflicts ------------------------------------------ tidyverse_conflicts() --
## x dplyr::filter() masks stats::filter()
## x dplyr::lag()    masks stats::lag()
## i Use the conflicted package (<http://conflicted.r-lib.org/>) to force all conflicts to become errors
\end{verbatim}

\begin{Shaded}
\begin{Highlighting}[]
\FunctionTok{head}\NormalTok{(gapminder) }\CommentTok{\# shows first 6}
\end{Highlighting}
\end{Shaded}

\begin{verbatim}
## # A tibble: 6 x 6
##   country     continent  year lifeExp      pop gdpPercap
##   <fct>       <fct>     <int>   <dbl>    <int>     <dbl>
## 1 Afghanistan Asia       1952    28.8  8425333      779.
## 2 Afghanistan Asia       1957    30.3  9240934      821.
## 3 Afghanistan Asia       1962    32.0 10267083      853.
## 4 Afghanistan Asia       1967    34.0 11537966      836.
## 5 Afghanistan Asia       1972    36.1 13079460      740.
## 6 Afghanistan Asia       1977    38.4 14880372      786.
\end{verbatim}

\begin{Shaded}
\begin{Highlighting}[]
\FunctionTok{tail}\NormalTok{(gapminder) }\CommentTok{\# shows last 6}
\end{Highlighting}
\end{Shaded}

\begin{verbatim}
## # A tibble: 6 x 6
##   country  continent  year lifeExp      pop gdpPercap
##   <fct>    <fct>     <int>   <dbl>    <int>     <dbl>
## 1 Zimbabwe Africa     1982    60.4  7636524      789.
## 2 Zimbabwe Africa     1987    62.4  9216418      706.
## 3 Zimbabwe Africa     1992    60.4 10704340      693.
## 4 Zimbabwe Africa     1997    46.8 11404948      792.
## 5 Zimbabwe Africa     2002    40.0 11926563      672.
## 6 Zimbabwe Africa     2007    43.5 12311143      470.
\end{verbatim}

\begin{Shaded}
\begin{Highlighting}[]
\FunctionTok{head}\NormalTok{(gapminder, }\DecValTok{10}\NormalTok{) }\CommentTok{\# shows first X that you indicate}
\end{Highlighting}
\end{Shaded}

\begin{verbatim}
## # A tibble: 10 x 6
##    country     continent  year lifeExp      pop gdpPercap
##    <fct>       <fct>     <int>   <dbl>    <int>     <dbl>
##  1 Afghanistan Asia       1952    28.8  8425333      779.
##  2 Afghanistan Asia       1957    30.3  9240934      821.
##  3 Afghanistan Asia       1962    32.0 10267083      853.
##  4 Afghanistan Asia       1967    34.0 11537966      836.
##  5 Afghanistan Asia       1972    36.1 13079460      740.
##  6 Afghanistan Asia       1977    38.4 14880372      786.
##  7 Afghanistan Asia       1982    39.9 12881816      978.
##  8 Afghanistan Asia       1987    40.8 13867957      852.
##  9 Afghanistan Asia       1992    41.7 16317921      649.
## 10 Afghanistan Asia       1997    41.8 22227415      635.
\end{verbatim}

\begin{Shaded}
\begin{Highlighting}[]
\FunctionTok{str}\NormalTok{(gapminder) }\CommentTok{\# ?str {-} displays the structure of an object}
\end{Highlighting}
\end{Shaded}

\begin{verbatim}
## tibble [1,704 x 6] (S3: tbl_df/tbl/data.frame)
##  $ country  : Factor w/ 142 levels "Afghanistan",..: 1 1 1 1 1 1 1 1 1 1 ...
##  $ continent: Factor w/ 5 levels "Africa","Americas",..: 3 3 3 3 3 3 3 3 3 3 ...
##  $ year     : int [1:1704] 1952 1957 1962 1967 1972 1977 1982 1987 1992 1997 ...
##  $ lifeExp  : num [1:1704] 28.8 30.3 32 34 36.1 ...
##  $ pop      : int [1:1704] 8425333 9240934 10267083 11537966 13079460 14880372 12881816 13867957 16317921 22227415 ...
##  $ gdpPercap: num [1:1704] 779 821 853 836 740 ...
\end{verbatim}

\subsection{dplyr basics}\label{dplyr-basics}

There are five dplyr functions that you will use to do the vast majority
of data manipulations:\\
filter(): pick observations by their values\\
select(): pick variables by their names\\
mutate(): create new variables with functions of existing variables\\
summarise(): collapse many values down to a single summary\\
arrange(): reorder the rows\\
These can all be used in conjunction with group\_by() which changes the
scope of each function from operating on the entire dataset to operating
on it group-by-group. These six functions provide the verbs for a
language of data manipulation. All verbs work similarly:\\
1. The first argument is a data frame.\\
2. The subsequent arguments describe what to do with the data frame. You
can refer to columns in the data frame directly without using \$.\\
3. The result is a new data frame.\\
Together these properties make it easy to chain together multiple simple
steps to achieve a complex result

\paragraph{filter() subsets data row-wise
(observations)}\label{filter-subsets-data-row-wise-observations}

You will want to isolate bits of your data; maybe you want to only look
at a single country or a few years. R calls this subsetting.\\
filter() is a function in dplyr that takes logical expressions and
returns the rows for which all are TRUE.

\begin{Shaded}
\begin{Highlighting}[]
\FunctionTok{filter}\NormalTok{(gapminder, lifeExp }\SpecialCharTok{\textless{}} \DecValTok{29}\NormalTok{)}
\end{Highlighting}
\end{Shaded}

\begin{verbatim}
## # A tibble: 2 x 6
##   country     continent  year lifeExp     pop gdpPercap
##   <fct>       <fct>     <int>   <dbl>   <int>     <dbl>
## 1 Afghanistan Asia       1952    28.8 8425333      779.
## 2 Rwanda      Africa     1992    23.6 7290203      737.
\end{verbatim}

Let's try another: ``Filter the gapminder data for the country Mexico''.

\begin{Shaded}
\begin{Highlighting}[]
\FunctionTok{filter}\NormalTok{(gapminder, country }\SpecialCharTok{==} \StringTok{"Mexico"}\NormalTok{)}
\end{Highlighting}
\end{Shaded}

\begin{verbatim}
## # A tibble: 12 x 6
##    country continent  year lifeExp       pop gdpPercap
##    <fct>   <fct>     <int>   <dbl>     <int>     <dbl>
##  1 Mexico  Americas   1952    50.8  30144317     3478.
##  2 Mexico  Americas   1957    55.2  35015548     4132.
##  3 Mexico  Americas   1962    58.3  41121485     4582.
##  4 Mexico  Americas   1967    60.1  47995559     5755.
##  5 Mexico  Americas   1972    62.4  55984294     6809.
##  6 Mexico  Americas   1977    65.0  63759976     7675.
##  7 Mexico  Americas   1982    67.4  71640904     9611.
##  8 Mexico  Americas   1987    69.5  80122492     8688.
##  9 Mexico  Americas   1992    71.5  88111030     9472.
## 10 Mexico  Americas   1997    73.7  95895146     9767.
## 11 Mexico  Americas   2002    74.9 102479927    10742.
## 12 Mexico  Americas   2007    76.2 108700891    11978.
\end{verbatim}

How about if we want two country names? We can't use the == operator
here, because it can only operate on one thing at a time. We will use
the \%in\% operator:

\begin{Shaded}
\begin{Highlighting}[]
\FunctionTok{filter}\NormalTok{(gapminder, country }\SpecialCharTok{\%in\%} \FunctionTok{c}\NormalTok{(}\StringTok{"Mexico"}\NormalTok{, }\StringTok{"Peru"}\NormalTok{))}
\end{Highlighting}
\end{Shaded}

\begin{verbatim}
## # A tibble: 24 x 6
##    country continent  year lifeExp      pop gdpPercap
##    <fct>   <fct>     <int>   <dbl>    <int>     <dbl>
##  1 Mexico  Americas   1952    50.8 30144317     3478.
##  2 Mexico  Americas   1957    55.2 35015548     4132.
##  3 Mexico  Americas   1962    58.3 41121485     4582.
##  4 Mexico  Americas   1967    60.1 47995559     5755.
##  5 Mexico  Americas   1972    62.4 55984294     6809.
##  6 Mexico  Americas   1977    65.0 63759976     7675.
##  7 Mexico  Americas   1982    67.4 71640904     9611.
##  8 Mexico  Americas   1987    69.5 80122492     8688.
##  9 Mexico  Americas   1992    71.5 88111030     9472.
## 10 Mexico  Americas   1997    73.7 95895146     9767.
## # i 14 more rows
\end{verbatim}

How about if we want Mexico in 2002? You can pass filter different
criteria:

\begin{Shaded}
\begin{Highlighting}[]
\FunctionTok{filter}\NormalTok{(gapminder, country }\SpecialCharTok{==} \StringTok{"Mexico"}\NormalTok{, year }\SpecialCharTok{==} \DecValTok{2002}\NormalTok{)}
\end{Highlighting}
\end{Shaded}

\begin{verbatim}
## # A tibble: 1 x 6
##   country continent  year lifeExp       pop gdpPercap
##   <fct>   <fct>     <int>   <dbl>     <int>     <dbl>
## 1 Mexico  Americas   2002    74.9 102479927    10742.
\end{verbatim}

\paragraph{select() subsets data column-wise
(variables)}\label{select-subsets-data-column-wise-variables}

We use select() to subset the data on variables or columns.\\
We can select multiple columns with a comma, after we specify the data
frame (gapminder).

\begin{Shaded}
\begin{Highlighting}[]
\NormalTok{gap1 }\OtherTok{\textless{}{-}}\NormalTok{ dplyr}\SpecialCharTok{::}\FunctionTok{select}\NormalTok{(gapminder, year, country, lifeExp)}
\FunctionTok{head}\NormalTok{(gap1, }\DecValTok{3}\NormalTok{)}
\end{Highlighting}
\end{Shaded}

\begin{verbatim}
## # A tibble: 3 x 3
##    year country     lifeExp
##   <int> <fct>         <dbl>
## 1  1952 Afghanistan    28.8
## 2  1957 Afghanistan    30.3
## 3  1962 Afghanistan    32.0
\end{verbatim}

We can select a range of variables with a semicolon.

\begin{Shaded}
\begin{Highlighting}[]
\NormalTok{gap2 }\OtherTok{\textless{}{-}}\NormalTok{ dplyr}\SpecialCharTok{::}\FunctionTok{select}\NormalTok{(gapminder, year}\SpecialCharTok{:}\NormalTok{lifeExp)}
\FunctionTok{head}\NormalTok{(gap2, }\DecValTok{3}\NormalTok{)}
\end{Highlighting}
\end{Shaded}

\begin{verbatim}
## # A tibble: 3 x 2
##    year lifeExp
##   <int>   <dbl>
## 1  1952    28.8
## 2  1957    30.3
## 3  1962    32.0
\end{verbatim}

We can select columns with indices.

\begin{Shaded}
\begin{Highlighting}[]
\NormalTok{gap3 }\OtherTok{\textless{}{-}}\NormalTok{ dplyr}\SpecialCharTok{::}\FunctionTok{select}\NormalTok{(gapminder, }\DecValTok{1}\NormalTok{, }\DecValTok{2}\NormalTok{, }\DecValTok{4}\NormalTok{)}
\FunctionTok{head}\NormalTok{(gap3, }\DecValTok{3}\NormalTok{)}
\end{Highlighting}
\end{Shaded}

\begin{verbatim}
## # A tibble: 3 x 3
##   country     continent lifeExp
##   <fct>       <fct>       <dbl>
## 1 Afghanistan Asia         28.8
## 2 Afghanistan Asia         30.3
## 3 Afghanistan Asia         32.0
\end{verbatim}

We can also use - to deselect columns.

\begin{Shaded}
\begin{Highlighting}[]
\NormalTok{gap4 }\OtherTok{\textless{}{-}}\NormalTok{ dplyr}\SpecialCharTok{::}\FunctionTok{select}\NormalTok{(gapminder, }\SpecialCharTok{{-}}\NormalTok{continent, }\SpecialCharTok{{-}}\NormalTok{lifeExp) }\CommentTok{\# you can use}
\FunctionTok{head}\NormalTok{(gap4, }\DecValTok{3}\NormalTok{)}
\end{Highlighting}
\end{Shaded}

\begin{verbatim}
## # A tibble: 3 x 4
##   country      year      pop gdpPercap
##   <fct>       <int>    <int>     <dbl>
## 1 Afghanistan  1952  8425333      779.
## 2 Afghanistan  1957  9240934      821.
## 3 Afghanistan  1962 10267083      853.
\end{verbatim}

\paragraph{Use select() and filter()
together}\label{use-select-and-filter-together}

Let's filter for Cambodia and remove the continent and lifeExp columns.
We'll save this as a variable. Actually, as two temporary variables,
which means that for the second one we need to operate on gap\_cambodia,
not gapminder.

\begin{Shaded}
\begin{Highlighting}[]
\NormalTok{gap\_cambodia }\OtherTok{\textless{}{-}} \FunctionTok{filter}\NormalTok{(gapminder, country }\SpecialCharTok{==} \StringTok{"Cambodia"}\NormalTok{)}
\NormalTok{gap\_cambodia2 }\OtherTok{\textless{}{-}}\NormalTok{ dplyr}\SpecialCharTok{::}\FunctionTok{select}\NormalTok{(gap\_cambodia, }\SpecialCharTok{{-}}\NormalTok{continent, }\SpecialCharTok{{-}}\NormalTok{lifeExp)}
\FunctionTok{head}\NormalTok{(gap\_cambodia2)}
\end{Highlighting}
\end{Shaded}

\begin{verbatim}
## # A tibble: 6 x 4
##   country   year     pop gdpPercap
##   <fct>    <int>   <int>     <dbl>
## 1 Cambodia  1952 4693836      368.
## 2 Cambodia  1957 5322536      434.
## 3 Cambodia  1962 6083619      497.
## 4 Cambodia  1967 6960067      523.
## 5 Cambodia  1972 7450606      422.
## 6 Cambodia  1977 6978607      525.
\end{verbatim}

\paragraph{Meet the new pipe \%\textgreater\% (\textbar\textgreater)
operator}\label{meet-the-new-pipe-operator}

\begin{Shaded}
\begin{Highlighting}[]
\NormalTok{gapminder }\SpecialCharTok{|\textgreater{}} \FunctionTok{head}\NormalTok{(}\DecValTok{3}\NormalTok{)}
\end{Highlighting}
\end{Shaded}

\begin{verbatim}
## # A tibble: 3 x 6
##   country     continent  year lifeExp      pop gdpPercap
##   <fct>       <fct>     <int>   <dbl>    <int>     <dbl>
## 1 Afghanistan Asia       1952    28.8  8425333      779.
## 2 Afghanistan Asia       1957    30.3  9240934      821.
## 3 Afghanistan Asia       1962    32.0 10267083      853.
\end{verbatim}

This means that:

\begin{Shaded}
\begin{Highlighting}[]
\NormalTok{gap\_cambodia }\OtherTok{\textless{}{-}}\NormalTok{ gapminder }\SpecialCharTok{|\textgreater{}} \FunctionTok{filter}\NormalTok{(country }\SpecialCharTok{==} \StringTok{"Cambodia"}\NormalTok{)}
\NormalTok{gap\_cambodia2 }\OtherTok{\textless{}{-}}\NormalTok{ gap\_cambodia }\SpecialCharTok{|\textgreater{}}\NormalTok{ dplyr}\SpecialCharTok{::}\FunctionTok{select}\NormalTok{(}\SpecialCharTok{{-}}\NormalTok{continent, }\SpecialCharTok{{-}}\NormalTok{lifeExp)}
\end{Highlighting}
\end{Shaded}

We can use the pipe to chain those two operations together:

\begin{Shaded}
\begin{Highlighting}[]
\NormalTok{gap\_cambodia }\OtherTok{\textless{}{-}}\NormalTok{ gapminder }\SpecialCharTok{|\textgreater{}}
 \FunctionTok{filter}\NormalTok{(country }\SpecialCharTok{==} \StringTok{"Cambodia"}\NormalTok{) }\SpecialCharTok{|\textgreater{}}
\NormalTok{ dplyr}\SpecialCharTok{::}\FunctionTok{select}\NormalTok{(}\SpecialCharTok{{-}}\NormalTok{continent, }\SpecialCharTok{{-}}\NormalTok{lifeExp)}
\end{Highlighting}
\end{Shaded}

\paragraph{mutate() adds new variables}\label{mutate-adds-new-variables}

Imagine we want to know each country's annual GDP. We can multiply pop
by gdpPercap to create a new column named gdp.

\begin{Shaded}
\begin{Highlighting}[]
\NormalTok{gapminder }\SpecialCharTok{|\textgreater{}}
\FunctionTok{mutate}\NormalTok{(}\AttributeTok{gdp =}\NormalTok{ pop }\SpecialCharTok{*}\NormalTok{ gdpPercap)}
\end{Highlighting}
\end{Shaded}

\begin{verbatim}
## # A tibble: 1,704 x 7
##    country     continent  year lifeExp      pop gdpPercap          gdp
##    <fct>       <fct>     <int>   <dbl>    <int>     <dbl>        <dbl>
##  1 Afghanistan Asia       1952    28.8  8425333      779.  6567086330.
##  2 Afghanistan Asia       1957    30.3  9240934      821.  7585448670.
##  3 Afghanistan Asia       1962    32.0 10267083      853.  8758855797.
##  4 Afghanistan Asia       1967    34.0 11537966      836.  9648014150.
##  5 Afghanistan Asia       1972    36.1 13079460      740.  9678553274.
##  6 Afghanistan Asia       1977    38.4 14880372      786. 11697659231.
##  7 Afghanistan Asia       1982    39.9 12881816      978. 12598563401.
##  8 Afghanistan Asia       1987    40.8 13867957      852. 11820990309.
##  9 Afghanistan Asia       1992    41.7 16317921      649. 10595901589.
## 10 Afghanistan Asia       1997    41.8 22227415      635. 14121995875.
## # i 1,694 more rows
\end{verbatim}

\paragraph{group\_by() operates on
groups}\label{group_by-operates-on-groups}

What if we wanted to know the total population on each continent in
2002? Answering this question requires a grouping variable.\\
By using group\_by() we can set our grouping variable to continent and
create a new column called cont\_pop that will add up all country
populations by their associated continents.

\begin{Shaded}
\begin{Highlighting}[]
\NormalTok{gapminder }\SpecialCharTok{|\textgreater{}}
 \FunctionTok{filter}\NormalTok{(year }\SpecialCharTok{==} \DecValTok{2002}\NormalTok{) }\SpecialCharTok{|\textgreater{}}
 \FunctionTok{group\_by}\NormalTok{(continent) }\SpecialCharTok{|\textgreater{}}
 \FunctionTok{mutate}\NormalTok{(}\AttributeTok{cont\_pop =} \FunctionTok{sum}\NormalTok{(pop))}
\end{Highlighting}
\end{Shaded}

\begin{verbatim}
## # A tibble: 142 x 7
## # Groups:   continent [5]
##    country     continent  year lifeExp       pop gdpPercap   cont_pop
##    <fct>       <fct>     <int>   <dbl>     <int>     <dbl>      <dbl>
##  1 Afghanistan Asia       2002    42.1  25268405      727. 3601802203
##  2 Albania     Europe     2002    75.7   3508512     4604.  578223869
##  3 Algeria     Africa     2002    71.0  31287142     5288.  833723916
##  4 Angola      Africa     2002    41.0  10866106     2773.  833723916
##  5 Argentina   Americas   2002    74.3  38331121     8798.  849772762
##  6 Australia   Oceania    2002    80.4  19546792    30688.   23454829
##  7 Austria     Europe     2002    79.0   8148312    32418.  578223869
##  8 Bahrain     Asia       2002    74.8    656397    23404. 3601802203
##  9 Bangladesh  Asia       2002    62.0 135656790     1136. 3601802203
## 10 Belgium     Europe     2002    78.3  10311970    30486.  578223869
## # i 132 more rows
\end{verbatim}

\paragraph{summarize() with group\_by()}\label{summarize-with-group_by}

We want to operate on a group, but actually collapse or distill the
output from that group. The summarize() function will do that for us.

\begin{Shaded}
\begin{Highlighting}[]
\NormalTok{gapminder }\SpecialCharTok{|\textgreater{}}
 \FunctionTok{group\_by}\NormalTok{(continent) }\SpecialCharTok{|\textgreater{}}
 \FunctionTok{summarize}\NormalTok{(}\AttributeTok{cont\_pop =} \FunctionTok{sum}\NormalTok{(pop)) }\SpecialCharTok{|\textgreater{}}
 \FunctionTok{ungroup}\NormalTok{()}
\end{Highlighting}
\end{Shaded}

\begin{verbatim}
## # A tibble: 5 x 2
##   continent    cont_pop
##   <fct>           <dbl>
## 1 Africa     6187585961
## 2 Americas   7351438499
## 3 Asia      30507333901
## 4 Europe     6181115304
## 5 Oceania     212992136
\end{verbatim}

summarize() will actually only keep the columns that are grouped\_by or
summarized. So if we wanted to keep other columns, we'd have to do have
a few more steps. ungroup() removes the grouping and it's good to get in
the habit of using it after a group\_by().\\
We can use more than one grouping variable. Let's get total populations
by continent and year.

\begin{Shaded}
\begin{Highlighting}[]
\NormalTok{gapminder }\SpecialCharTok{|\textgreater{}}
 \FunctionTok{group\_by}\NormalTok{(continent, year) }\SpecialCharTok{|\textgreater{}}
 \FunctionTok{summarize}\NormalTok{(}\AttributeTok{cont\_pop =} \FunctionTok{sum}\NormalTok{(pop))}
\end{Highlighting}
\end{Shaded}

\begin{verbatim}
## `summarise()` has grouped output by 'continent'. You can override using the
## `.groups` argument.
\end{verbatim}

\begin{verbatim}
## # A tibble: 60 x 3
## # Groups:   continent [5]
##    continent  year  cont_pop
##    <fct>     <int>     <dbl>
##  1 Africa     1952 237640501
##  2 Africa     1957 264837738
##  3 Africa     1962 296516865
##  4 Africa     1967 335289489
##  5 Africa     1972 379879541
##  6 Africa     1977 433061021
##  7 Africa     1982 499348587
##  8 Africa     1987 574834110
##  9 Africa     1992 659081517
## 10 Africa     1997 743832984
## # i 50 more rows
\end{verbatim}

\paragraph{arrange() orders columns}\label{arrange-orders-columns}

This is ordered alphabetically, which is cool. But let's say we wanted
to order it in ascending order for year. The dplyr function is
arrange().

\begin{Shaded}
\begin{Highlighting}[]
\NormalTok{gapminder }\SpecialCharTok{|\textgreater{}}
 \FunctionTok{group\_by}\NormalTok{(continent, year) }\SpecialCharTok{|\textgreater{}}
 \FunctionTok{summarize}\NormalTok{(}\AttributeTok{cont\_pop =} \FunctionTok{sum}\NormalTok{(pop)) }\SpecialCharTok{|\textgreater{}}
 \FunctionTok{arrange}\NormalTok{(year)}
\end{Highlighting}
\end{Shaded}

\begin{verbatim}
## `summarise()` has grouped output by 'continent'. You can override using the
## `.groups` argument.
\end{verbatim}

\begin{verbatim}
## # A tibble: 60 x 3
## # Groups:   continent [5]
##    continent  year   cont_pop
##    <fct>     <int>      <dbl>
##  1 Africa     1952  237640501
##  2 Americas   1952  345152446
##  3 Asia       1952 1395357351
##  4 Europe     1952  418120846
##  5 Oceania    1952   10686006
##  6 Africa     1957  264837738
##  7 Americas   1957  386953916
##  8 Asia       1957 1562780599
##  9 Europe     1957  437890351
## 10 Oceania    1957   11941976
## # i 50 more rows
\end{verbatim}

\subsection{Compare to base R}\label{compare-to-base-r}

Instead of calculating the max for each country like we did with dplyr
above, here we will calculate the max for one country, Mexico.

\begin{Shaded}
\begin{Highlighting}[]
\NormalTok{gapminder }\OtherTok{\textless{}{-}} \FunctionTok{read.csv}\NormalTok{(}\StringTok{\textquotesingle{}data/gapminder.csv\textquotesingle{}}\NormalTok{, }\AttributeTok{stringsAsFactors =} \ConstantTok{FALSE}\NormalTok{)}
\NormalTok{x1 }\OtherTok{\textless{}{-}}\NormalTok{ gapminder[ , }\FunctionTok{c}\NormalTok{(}\StringTok{\textquotesingle{}country\textquotesingle{}}\NormalTok{, }\StringTok{\textquotesingle{}year\textquotesingle{}}\NormalTok{, }\StringTok{\textquotesingle{}pop\textquotesingle{}}\NormalTok{, }\StringTok{\textquotesingle{}gdpPercap\textquotesingle{}}\NormalTok{) ] }\CommentTok{\# subse}
\NormalTok{mex }\OtherTok{\textless{}{-}}\NormalTok{ x1[x1}\SpecialCharTok{$}\NormalTok{country }\SpecialCharTok{==} \StringTok{"Mexico"}\NormalTok{, ] }\CommentTok{\# subset rows}
\NormalTok{mex}\SpecialCharTok{$}\NormalTok{gdp }\OtherTok{\textless{}{-}}\NormalTok{ mex}\SpecialCharTok{$}\NormalTok{pop }\SpecialCharTok{*}\NormalTok{ mex}\SpecialCharTok{$}\NormalTok{gdpPercap }\CommentTok{\# add new columns}
\NormalTok{mex}\SpecialCharTok{$}\NormalTok{max\_gdp }\OtherTok{\textless{}{-}} \FunctionTok{max}\NormalTok{(mex}\SpecialCharTok{$}\NormalTok{gdp)}
\end{Highlighting}
\end{Shaded}

\subsection{Joining datasets}\label{joining-datasets}

Most of the time you will have data coming from different places or
indifferent files, and you want to put them together so you can analyze
them. Datasets you'll be joining can be called relational data, because
it has some kind of relationship between them that you'll be acting
upon. In the tidyverse, combining data that has a relationship is called
``joining''.\\
We will only talk about this briefly here, but you can refer to this
more as youhave your own datasets that you want to join. This describes
the figure above:\\
left\_join keeps everything from the left table and matches as much as
it can from the right table. In R, the first thing that you type will be
the left table (because it's on the left)\\
right\_join keeps everything from the right table and matches as much as
it can from the left table\\
inner\_join only keeps the observations that are similar between the two
tables\\
full\_join keeps all observations from both tables.\\
Let's play with these CO2 emissions data to illustrate:\\

\begin{Shaded}
\begin{Highlighting}[]
\DocumentationTok{\#\# read in the data. (same URL as yesterday, with co2.csv instead of}
\NormalTok{co2 }\OtherTok{\textless{}{-}} \FunctionTok{read\_csv}\NormalTok{(}\StringTok{"data/co2.csv"}\NormalTok{)}
\end{Highlighting}
\end{Shaded}

\begin{verbatim}
## Rows: 12 Columns: 2
## -- Column specification --------------------------------------------------------
## Delimiter: ","
## chr (1): country
## dbl (1): co2_2007
## 
## i Use `spec()` to retrieve the full column specification for this data.
## i Specify the column types or set `show_col_types = FALSE` to quiet this message.
\end{verbatim}

\begin{Shaded}
\begin{Highlighting}[]
\DocumentationTok{\#\# explore}
\NormalTok{co2 }\SpecialCharTok{|\textgreater{}} \FunctionTok{head}\NormalTok{()}
\end{Highlighting}
\end{Shaded}

\begin{verbatim}
## # A tibble: 6 x 2
##   country        co2_2007
##   <chr>             <dbl>
## 1 Afghanistan      2938. 
## 2 Albania          4218. 
## 3 Algeria        105838. 
## 4 American Samoa     18.4
## 5 Angola          17405. 
## 6 Anguilla           12.4
\end{verbatim}

\begin{Shaded}
\begin{Highlighting}[]
\NormalTok{co2 }\SpecialCharTok{|\textgreater{}} \FunctionTok{dim}\NormalTok{() }\CommentTok{\# 12}
\end{Highlighting}
\end{Shaded}

\begin{verbatim}
## [1] 12  2
\end{verbatim}

\begin{Shaded}
\begin{Highlighting}[]
\DocumentationTok{\#\# create new variable that is only 2007 data}
\NormalTok{gap\_2007 }\OtherTok{\textless{}{-}}\NormalTok{ gapminder }\SpecialCharTok{|\textgreater{}}
\FunctionTok{filter}\NormalTok{(year }\SpecialCharTok{==} \DecValTok{2007}\NormalTok{)}
\NormalTok{gap\_2007 }\SpecialCharTok{|\textgreater{}} \FunctionTok{dim}\NormalTok{() }\CommentTok{\# 142}
\end{Highlighting}
\end{Shaded}

\begin{verbatim}
## [1] 142   6
\end{verbatim}

\begin{Shaded}
\begin{Highlighting}[]
\DocumentationTok{\#\# left\_join gap\_2007 to co2}
\NormalTok{lj }\OtherTok{\textless{}{-}} \FunctionTok{left\_join}\NormalTok{(gap\_2007, co2, }\AttributeTok{by =} \StringTok{"country"}\NormalTok{)}
\DocumentationTok{\#\# explore}
\NormalTok{lj }\SpecialCharTok{|\textgreater{}} \FunctionTok{dim}\NormalTok{() }\CommentTok{\#142}
\end{Highlighting}
\end{Shaded}

\begin{verbatim}
## [1] 142   7
\end{verbatim}

\begin{Shaded}
\begin{Highlighting}[]
\NormalTok{lj }\SpecialCharTok{|\textgreater{}} \FunctionTok{head}\NormalTok{(}\DecValTok{3}\NormalTok{) }\CommentTok{\# lots of NAs in the co2\_2017 columm}
\end{Highlighting}
\end{Shaded}

\begin{verbatim}
##       country year      pop continent lifeExp gdpPercap   co2_2007
## 1 Afghanistan 2007 31889923      Asia  43.828  974.5803   2937.883
## 2     Albania 2007  3600523    Europe  76.423 5937.0295   4217.551
## 3     Algeria 2007 33333216    Africa  72.301 6223.3675 105838.129
\end{verbatim}

\begin{Shaded}
\begin{Highlighting}[]
\DocumentationTok{\#\# right\_join gap\_2007 and co2}
\NormalTok{rj }\OtherTok{\textless{}{-}} \FunctionTok{right\_join}\NormalTok{(gap\_2007, co2, }\AttributeTok{by =} \StringTok{"country"}\NormalTok{)}
\DocumentationTok{\#\# explore}
\NormalTok{rj }\SpecialCharTok{|\textgreater{}} \FunctionTok{dim}\NormalTok{() }\CommentTok{\# 12}
\end{Highlighting}
\end{Shaded}

\begin{verbatim}
## [1] 12  7
\end{verbatim}

\begin{Shaded}
\begin{Highlighting}[]
\NormalTok{rj }\SpecialCharTok{|\textgreater{}} \FunctionTok{head}\NormalTok{(}\DecValTok{3}\NormalTok{)}
\end{Highlighting}
\end{Shaded}

\begin{verbatim}
##       country year      pop continent lifeExp gdpPercap   co2_2007
## 1 Afghanistan 2007 31889923      Asia  43.828  974.5803   2937.883
## 2     Albania 2007  3600523    Europe  76.423 5937.0295   4217.551
## 3     Algeria 2007 33333216    Africa  72.301 6223.3675 105838.129
\end{verbatim}

\subsection{Data Wrangling: tidyr}\label{data-wrangling-tidyr}

An example of data in a wide format is the AirPassengers dataset which
provides information on monthly airline passenger numbers from
1949-1960. You'll notice that each row is a single year and the columns
are each month Jan - Dec.

\begin{Shaded}
\begin{Highlighting}[]
\NormalTok{AirPassengers}
\end{Highlighting}
\end{Shaded}

\begin{verbatim}
##      Jan Feb Mar Apr May Jun Jul Aug Sep Oct Nov Dec
## 1949 112 118 132 129 121 135 148 148 136 119 104 118
## 1950 115 126 141 135 125 149 170 170 158 133 114 140
## 1951 145 150 178 163 172 178 199 199 184 162 146 166
## 1952 171 180 193 181 183 218 230 242 209 191 172 194
## 1953 196 196 236 235 229 243 264 272 237 211 180 201
## 1954 204 188 235 227 234 264 302 293 259 229 203 229
## 1955 242 233 267 269 270 315 364 347 312 274 237 278
## 1956 284 277 317 313 318 374 413 405 355 306 271 306
## 1957 315 301 356 348 355 422 465 467 404 347 305 336
## 1958 340 318 362 348 363 435 491 505 404 359 310 337
## 1959 360 342 406 396 420 472 548 559 463 407 362 405
## 1960 417 391 419 461 472 535 622 606 508 461 390 432
\end{verbatim}

Often, data must be reshaped for it to become tidy data. What does that
mean? There are four main verbs we'll use, which are essentially pairs
of opposites:\\
turn columns into rows (gather()),\\
turn rows into columns (spread()),\\
turn a character column into multiple columns (separate()),\\
turn multiple character columns into a single column (unite())\\

\subsection{Explore gapminder dataset}\label{explore-gapminder-dataset}

\begin{Shaded}
\begin{Highlighting}[]
\FunctionTok{library}\NormalTok{(tidyverse)}
\DocumentationTok{\#\# wide format}
\NormalTok{gap\_wide }\OtherTok{\textless{}{-}}\NormalTok{ readr}\SpecialCharTok{::}\FunctionTok{read\_csv}\NormalTok{(}\StringTok{\textquotesingle{}data/gapminder\_wide.csv\textquotesingle{}}\NormalTok{)}
\end{Highlighting}
\end{Shaded}

\begin{verbatim}
## Rows: 142 Columns: 38
## -- Column specification --------------------------------------------------------
## Delimiter: ","
## chr  (2): continent, country
## dbl (36): gdpPercap_1952, gdpPercap_1957, gdpPercap_1962, gdpPercap_1967, gd...
## 
## i Use `spec()` to retrieve the full column specification for this data.
## i Specify the column types or set `show_col_types = FALSE` to quiet this message.
\end{verbatim}

\begin{Shaded}
\begin{Highlighting}[]
\NormalTok{gapminder }\OtherTok{\textless{}{-}}\NormalTok{ readr}\SpecialCharTok{::}\FunctionTok{read\_csv}\NormalTok{(}\StringTok{\textquotesingle{}data/gapminder.csv\textquotesingle{}}\NormalTok{)}
\end{Highlighting}
\end{Shaded}

\begin{verbatim}
## Rows: 1704 Columns: 6
## -- Column specification --------------------------------------------------------
## Delimiter: ","
## chr (2): country, continent
## dbl (4): year, pop, lifeExp, gdpPercap
## 
## i Use `spec()` to retrieve the full column specification for this data.
## i Specify the column types or set `show_col_types = FALSE` to quiet this message.
\end{verbatim}

\begin{Shaded}
\begin{Highlighting}[]
\CommentTok{\#head(gap\_wide)}
\FunctionTok{str}\NormalTok{(gap\_wide)}
\end{Highlighting}
\end{Shaded}

\begin{verbatim}
## spc_tbl_ [142 x 38] (S3: spec_tbl_df/tbl_df/tbl/data.frame)
##  $ continent     : chr [1:142] "Africa" "Africa" "Africa" "Africa" ...
##  $ country       : chr [1:142] "Algeria" "Angola" "Benin" "Botswana" ...
##  $ gdpPercap_1952: num [1:142] 2449 3521 1063 851 543 ...
##  $ gdpPercap_1957: num [1:142] 3014 3828 960 918 617 ...
##  $ gdpPercap_1962: num [1:142] 2551 4269 949 984 723 ...
##  $ gdpPercap_1967: num [1:142] 3247 5523 1036 1215 795 ...
##  $ gdpPercap_1972: num [1:142] 4183 5473 1086 2264 855 ...
##  $ gdpPercap_1977: num [1:142] 4910 3009 1029 3215 743 ...
##  $ gdpPercap_1982: num [1:142] 5745 2757 1278 4551 807 ...
##  $ gdpPercap_1987: num [1:142] 5681 2430 1226 6206 912 ...
##  $ gdpPercap_1992: num [1:142] 5023 2628 1191 7954 932 ...
##  $ gdpPercap_1997: num [1:142] 4797 2277 1233 8647 946 ...
##  $ gdpPercap_2002: num [1:142] 5288 2773 1373 11004 1038 ...
##  $ gdpPercap_2007: num [1:142] 6223 4797 1441 12570 1217 ...
##  $ lifeExp_1952  : num [1:142] 43.1 30 38.2 47.6 32 ...
##  $ lifeExp_1957  : num [1:142] 45.7 32 40.4 49.6 34.9 ...
##  $ lifeExp_1962  : num [1:142] 48.3 34 42.6 51.5 37.8 ...
##  $ lifeExp_1967  : num [1:142] 51.4 36 44.9 53.3 40.7 ...
##  $ lifeExp_1972  : num [1:142] 54.5 37.9 47 56 43.6 ...
##  $ lifeExp_1977  : num [1:142] 58 39.5 49.2 59.3 46.1 ...
##  $ lifeExp_1982  : num [1:142] 61.4 39.9 50.9 61.5 48.1 ...
##  $ lifeExp_1987  : num [1:142] 65.8 39.9 52.3 63.6 49.6 ...
##  $ lifeExp_1992  : num [1:142] 67.7 40.6 53.9 62.7 50.3 ...
##  $ lifeExp_1997  : num [1:142] 69.2 41 54.8 52.6 50.3 ...
##  $ lifeExp_2002  : num [1:142] 71 41 54.4 46.6 50.6 ...
##  $ lifeExp_2007  : num [1:142] 72.3 42.7 56.7 50.7 52.3 ...
##  $ pop_1952      : num [1:142] 9279525 4232095 1738315 442308 4469979 ...
##  $ pop_1957      : num [1:142] 10270856 4561361 1925173 474639 4713416 ...
##  $ pop_1962      : num [1:142] 11000948 4826015 2151895 512764 4919632 ...
##  $ pop_1967      : num [1:142] 12760499 5247469 2427334 553541 5127935 ...
##  $ pop_1972      : num [1:142] 14760787 5894858 2761407 619351 5433886 ...
##  $ pop_1977      : num [1:142] 17152804 6162675 3168267 781472 5889574 ...
##  $ pop_1982      : num [1:142] 20033753 7016384 3641603 970347 6634596 ...
##  $ pop_1987      : num [1:142] 23254956 7874230 4243788 1151184 7586551 ...
##  $ pop_1992      : num [1:142] 26298373 8735988 4981671 1342614 8878303 ...
##  $ pop_1997      : num [1:142] 29072015 9875024 6066080 1536536 10352843 ...
##  $ pop_2002      : num [1:142] 31287142 10866106 7026113 1630347 12251209 ...
##  $ pop_2007      : num [1:142] 33333216 12420476 8078314 1639131 14326203 ...
##  - attr(*, "spec")=
##   .. cols(
##   ..   continent = col_character(),
##   ..   country = col_character(),
##   ..   gdpPercap_1952 = col_double(),
##   ..   gdpPercap_1957 = col_double(),
##   ..   gdpPercap_1962 = col_double(),
##   ..   gdpPercap_1967 = col_double(),
##   ..   gdpPercap_1972 = col_double(),
##   ..   gdpPercap_1977 = col_double(),
##   ..   gdpPercap_1982 = col_double(),
##   ..   gdpPercap_1987 = col_double(),
##   ..   gdpPercap_1992 = col_double(),
##   ..   gdpPercap_1997 = col_double(),
##   ..   gdpPercap_2002 = col_double(),
##   ..   gdpPercap_2007 = col_double(),
##   ..   lifeExp_1952 = col_double(),
##   ..   lifeExp_1957 = col_double(),
##   ..   lifeExp_1962 = col_double(),
##   ..   lifeExp_1967 = col_double(),
##   ..   lifeExp_1972 = col_double(),
##   ..   lifeExp_1977 = col_double(),
##   ..   lifeExp_1982 = col_double(),
##   ..   lifeExp_1987 = col_double(),
##   ..   lifeExp_1992 = col_double(),
##   ..   lifeExp_1997 = col_double(),
##   ..   lifeExp_2002 = col_double(),
##   ..   lifeExp_2007 = col_double(),
##   ..   pop_1952 = col_double(),
##   ..   pop_1957 = col_double(),
##   ..   pop_1962 = col_double(),
##   ..   pop_1967 = col_double(),
##   ..   pop_1972 = col_double(),
##   ..   pop_1977 = col_double(),
##   ..   pop_1982 = col_double(),
##   ..   pop_1987 = col_double(),
##   ..   pop_1992 = col_double(),
##   ..   pop_1997 = col_double(),
##   ..   pop_2002 = col_double(),
##   ..   pop_2007 = col_double()
##   .. )
##  - attr(*, "problems")=<externalptr>
\end{verbatim}

We need to name two new variables in the key-value pair, one for the
key, one for the value. It can be hard to wrap your mind around this, so
let's give it a try. Let's name them obstype\_year and obs\_values.

\begin{Shaded}
\begin{Highlighting}[]
\NormalTok{gap\_long }\OtherTok{\textless{}{-}}\NormalTok{ gap\_wide }\SpecialCharTok{|\textgreater{}}
 \FunctionTok{gather}\NormalTok{(}\AttributeTok{key =}\NormalTok{ obstype\_year,}
  \AttributeTok{value =}\NormalTok{ obs\_values)}
\FunctionTok{str}\NormalTok{(gap\_long)}
\end{Highlighting}
\end{Shaded}

\begin{verbatim}
## tibble [5,396 x 2] (S3: tbl_df/tbl/data.frame)
##  $ obstype_year: chr [1:5396] "continent" "continent" "continent" "continent" ...
##  $ obs_values  : chr [1:5396] "Africa" "Africa" "Africa" "Africa" ...
\end{verbatim}

\begin{Shaded}
\begin{Highlighting}[]
\FunctionTok{head}\NormalTok{(gap\_long)}
\end{Highlighting}
\end{Shaded}

\begin{verbatim}
## # A tibble: 6 x 2
##   obstype_year obs_values
##   <chr>        <chr>     
## 1 continent    Africa    
## 2 continent    Africa    
## 3 continent    Africa    
## 4 continent    Africa    
## 5 continent    Africa    
## 6 continent    Africa
\end{verbatim}

\begin{Shaded}
\begin{Highlighting}[]
\FunctionTok{tail}\NormalTok{(gap\_long)}
\end{Highlighting}
\end{Shaded}

\begin{verbatim}
## # A tibble: 6 x 2
##   obstype_year obs_values
##   <chr>        <chr>     
## 1 pop_2007     9031088   
## 2 pop_2007     7554661   
## 3 pop_2007     71158647  
## 4 pop_2007     60776238  
## 5 pop_2007     20434176  
## 6 pop_2007     4115771
\end{verbatim}

\begin{Shaded}
\begin{Highlighting}[]
\NormalTok{gap\_long }\OtherTok{\textless{}{-}}\NormalTok{ gap\_wide }\SpecialCharTok{|\textgreater{}}
  \FunctionTok{gather}\NormalTok{(}\AttributeTok{key =}\NormalTok{ obstype\_year,}
        \AttributeTok{value =}\NormalTok{ obs\_values,}
\NormalTok{        dplyr}\SpecialCharTok{::}\FunctionTok{starts\_with}\NormalTok{(}\StringTok{\textquotesingle{}pop\textquotesingle{}}\NormalTok{),}
\NormalTok{        dplyr}\SpecialCharTok{::}\FunctionTok{starts\_with}\NormalTok{(}\StringTok{\textquotesingle{}lifeExp\textquotesingle{}}\NormalTok{),}
\NormalTok{        dplyr}\SpecialCharTok{::}\FunctionTok{starts\_with}\NormalTok{(}\StringTok{\textquotesingle{}gdpPercap\textquotesingle{}}\NormalTok{)) }\CommentTok{\# here i\textquotesingle{}m listing all the}
\FunctionTok{str}\NormalTok{(gap\_long)}
\end{Highlighting}
\end{Shaded}

\begin{verbatim}
## tibble [5,112 x 4] (S3: tbl_df/tbl/data.frame)
##  $ continent   : chr [1:5112] "Africa" "Africa" "Africa" "Africa" ...
##  $ country     : chr [1:5112] "Algeria" "Angola" "Benin" "Botswana" ...
##  $ obstype_year: chr [1:5112] "pop_1952" "pop_1952" "pop_1952" "pop_1952" ...
##  $ obs_values  : num [1:5112] 9279525 4232095 1738315 442308 4469979 ...
\end{verbatim}

\begin{Shaded}
\begin{Highlighting}[]
\FunctionTok{head}\NormalTok{(gap\_long)}
\end{Highlighting}
\end{Shaded}

\begin{verbatim}
## # A tibble: 6 x 4
##   continent country      obstype_year obs_values
##   <chr>     <chr>        <chr>             <dbl>
## 1 Africa    Algeria      pop_1952        9279525
## 2 Africa    Angola       pop_1952        4232095
## 3 Africa    Benin        pop_1952        1738315
## 4 Africa    Botswana     pop_1952         442308
## 5 Africa    Burkina Faso pop_1952        4469979
## 6 Africa    Burundi      pop_1952        2445618
\end{verbatim}

\begin{Shaded}
\begin{Highlighting}[]
\FunctionTok{tail}\NormalTok{(gap\_long)}
\end{Highlighting}
\end{Shaded}

\begin{verbatim}
## # A tibble: 6 x 4
##   continent country        obstype_year   obs_values
##   <chr>     <chr>          <chr>               <dbl>
## 1 Europe    Sweden         gdpPercap_2007     33860.
## 2 Europe    Switzerland    gdpPercap_2007     37506.
## 3 Europe    Turkey         gdpPercap_2007      8458.
## 4 Europe    United Kingdom gdpPercap_2007     33203.
## 5 Oceania   Australia      gdpPercap_2007     34435.
## 6 Oceania   New Zealand    gdpPercap_2007     25185.
\end{verbatim}

Success! And there is another way that is nice to use if your columns
don't follow such a structured pattern: you can exclude the columns you
don't want.

\begin{Shaded}
\begin{Highlighting}[]
\NormalTok{gap\_long }\OtherTok{\textless{}{-}}\NormalTok{ gap\_wide }\SpecialCharTok{|\textgreater{}}
  \FunctionTok{gather}\NormalTok{(}\AttributeTok{key =}\NormalTok{ obstype\_year,}
         \AttributeTok{value =}\NormalTok{ obs\_values,}
         \SpecialCharTok{{-}}\NormalTok{continent, }\SpecialCharTok{{-}}\NormalTok{country)}
\FunctionTok{str}\NormalTok{(gap\_long)}
\end{Highlighting}
\end{Shaded}

\begin{verbatim}
## tibble [5,112 x 4] (S3: tbl_df/tbl/data.frame)
##  $ continent   : chr [1:5112] "Africa" "Africa" "Africa" "Africa" ...
##  $ country     : chr [1:5112] "Algeria" "Angola" "Benin" "Botswana" ...
##  $ obstype_year: chr [1:5112] "gdpPercap_1952" "gdpPercap_1952" "gdpPercap_1952" "gdpPercap_1952" ...
##  $ obs_values  : num [1:5112] 2449 3521 1063 851 543 ...
\end{verbatim}

\begin{Shaded}
\begin{Highlighting}[]
\FunctionTok{head}\NormalTok{(gap\_long, }\DecValTok{3}\NormalTok{)}
\end{Highlighting}
\end{Shaded}

\begin{verbatim}
## # A tibble: 3 x 4
##   continent country obstype_year   obs_values
##   <chr>     <chr>   <chr>               <dbl>
## 1 Africa    Algeria gdpPercap_1952      2449.
## 2 Africa    Angola  gdpPercap_1952      3521.
## 3 Africa    Benin   gdpPercap_1952      1063.
\end{verbatim}

?separate --\textgreater{} the main arguments are separate(data, col,
into, sep \ldots). So we need to specify which column we want separated,
name the new columns that we want to create, and specify what we want it
to separate by. Since the obstype\_year variable has observation types
and years separated by a \_, we'll use that.

\begin{Shaded}
\begin{Highlighting}[]
\NormalTok{gap\_long }\OtherTok{\textless{}{-}}\NormalTok{ gap\_wide }\SpecialCharTok{|\textgreater{}}
  \FunctionTok{gather}\NormalTok{(}\AttributeTok{key =}\NormalTok{ obstype\_year,}
         \AttributeTok{value =}\NormalTok{ obs\_values,}
         \SpecialCharTok{{-}}\NormalTok{continent, }\SpecialCharTok{{-}}\NormalTok{country) }\SpecialCharTok{|\textgreater{}}
  \FunctionTok{separate}\NormalTok{(obstype\_year,}
           \AttributeTok{into =} \FunctionTok{c}\NormalTok{(}\StringTok{\textquotesingle{}obs\_type\textquotesingle{}}\NormalTok{,}\StringTok{\textquotesingle{}year\textquotesingle{}}\NormalTok{),}
           \AttributeTok{sep =} \StringTok{"\_"}\NormalTok{,}
          \AttributeTok{convert =} \ConstantTok{TRUE}\NormalTok{) }\CommentTok{\# this ensures that the year column is an i}
\FunctionTok{str}\NormalTok{(gap\_long)}
\end{Highlighting}
\end{Shaded}

\begin{verbatim}
## tibble [5,112 x 5] (S3: tbl_df/tbl/data.frame)
##  $ continent : chr [1:5112] "Africa" "Africa" "Africa" "Africa" ...
##  $ country   : chr [1:5112] "Algeria" "Angola" "Benin" "Botswana" ...
##  $ obs_type  : chr [1:5112] "gdpPercap" "gdpPercap" "gdpPercap" "gdpPercap" ...
##  $ year      : int [1:5112] 1952 1952 1952 1952 1952 1952 1952 1952 1952 1952 ...
##  $ obs_values: num [1:5112] 2449 3521 1063 851 543 ...
\end{verbatim}

\begin{Shaded}
\begin{Highlighting}[]
\FunctionTok{head}\NormalTok{(gap\_long, }\DecValTok{3}\NormalTok{)}
\end{Highlighting}
\end{Shaded}

\begin{verbatim}
## # A tibble: 3 x 5
##   continent country obs_type   year obs_values
##   <chr>     <chr>   <chr>     <int>      <dbl>
## 1 Africa    Algeria gdpPercap  1952      2449.
## 2 Africa    Angola  gdpPercap  1952      3521.
## 3 Africa    Benin   gdpPercap  1952      1063.
\end{verbatim}

\subsection{Plot long format data}\label{plot-long-format-data}

The long format is the preferred format for plotting with ggplot2. Let's
look at an example by plotting just Canada's life expectancy.

\begin{Shaded}
\begin{Highlighting}[]
\NormalTok{canada\_df }\OtherTok{\textless{}{-}}\NormalTok{ gap\_long }\SpecialCharTok{|\textgreater{}}
  \FunctionTok{filter}\NormalTok{(obs\_type }\SpecialCharTok{==} \StringTok{"lifeExp"}\NormalTok{,}
\NormalTok{         country }\SpecialCharTok{==} \StringTok{"Canada"}\NormalTok{)}
\FunctionTok{ggplot}\NormalTok{(canada\_df, }\FunctionTok{aes}\NormalTok{(}\AttributeTok{x =}\NormalTok{ year, }\AttributeTok{y =}\NormalTok{ obs\_values)) }\SpecialCharTok{+}
  \FunctionTok{geom\_line}\NormalTok{()}
\end{Highlighting}
\end{Shaded}

\includegraphics{Ch4_files/figure-latex/unnamed-chunk-39-1.pdf} We can
also look at all countries in the Americas:

\begin{Shaded}
\begin{Highlighting}[]
\NormalTok{life\_df }\OtherTok{\textless{}{-}}\NormalTok{ gap\_long }\SpecialCharTok{|\textgreater{}}
  \FunctionTok{filter}\NormalTok{(obs\_type }\SpecialCharTok{==} \StringTok{"lifeExp"}\NormalTok{,}
\NormalTok{         continent }\SpecialCharTok{==} \StringTok{"Americas"}\NormalTok{)}
\FunctionTok{ggplot}\NormalTok{(life\_df, }\FunctionTok{aes}\NormalTok{(}\AttributeTok{x =}\NormalTok{ year, }\AttributeTok{y =}\NormalTok{ obs\_values, }\AttributeTok{color =}\NormalTok{ country)) }\SpecialCharTok{+}
  \FunctionTok{geom\_line}\NormalTok{()}
\end{Highlighting}
\end{Shaded}

\includegraphics{Ch4_files/figure-latex/unnamed-chunk-40-1.pdf}

\subsection{spread()}\label{spread}

The function spread() is used to transform data from long to wide
format.

\begin{Shaded}
\begin{Highlighting}[]
\NormalTok{gap\_normal }\OtherTok{\textless{}{-}}\NormalTok{ gap\_long }\SpecialCharTok{|\textgreater{}}
  \FunctionTok{spread}\NormalTok{(obs\_type, obs\_values)}
\FunctionTok{dim}\NormalTok{(gap\_normal)}
\end{Highlighting}
\end{Shaded}

\begin{verbatim}
## [1] 1704    6
\end{verbatim}

\begin{Shaded}
\begin{Highlighting}[]
\FunctionTok{dim}\NormalTok{(gapminder)}
\end{Highlighting}
\end{Shaded}

\begin{verbatim}
## [1] 1704    6
\end{verbatim}

\begin{Shaded}
\begin{Highlighting}[]
\FunctionTok{names}\NormalTok{(gap\_normal)}
\end{Highlighting}
\end{Shaded}

\begin{verbatim}
## [1] "continent" "country"   "year"      "gdpPercap" "lifeExp"   "pop"
\end{verbatim}

\begin{Shaded}
\begin{Highlighting}[]
\FunctionTok{names}\NormalTok{(gapminder)}
\end{Highlighting}
\end{Shaded}

\begin{verbatim}
## [1] "country"   "year"      "pop"       "continent" "lifeExp"   "gdpPercap"
\end{verbatim}

Now we've got a dataframe gap\_normal with the same dimensions as the
original gapminder.

\end{document}
