% Options for packages loaded elsewhere
\PassOptionsToPackage{unicode}{hyperref}
\PassOptionsToPackage{hyphens}{url}
%
\documentclass[
]{article}
\usepackage{amsmath,amssymb}
\usepackage{iftex}
\ifPDFTeX
  \usepackage[T1]{fontenc}
  \usepackage[utf8]{inputenc}
  \usepackage{textcomp} % provide euro and other symbols
\else % if luatex or xetex
  \usepackage{unicode-math} % this also loads fontspec
  \defaultfontfeatures{Scale=MatchLowercase}
  \defaultfontfeatures[\rmfamily]{Ligatures=TeX,Scale=1}
\fi
\usepackage{lmodern}
\ifPDFTeX\else
  % xetex/luatex font selection
\fi
% Use upquote if available, for straight quotes in verbatim environments
\IfFileExists{upquote.sty}{\usepackage{upquote}}{}
\IfFileExists{microtype.sty}{% use microtype if available
  \usepackage[]{microtype}
  \UseMicrotypeSet[protrusion]{basicmath} % disable protrusion for tt fonts
}{}
\makeatletter
\@ifundefined{KOMAClassName}{% if non-KOMA class
  \IfFileExists{parskip.sty}{%
    \usepackage{parskip}
  }{% else
    \setlength{\parindent}{0pt}
    \setlength{\parskip}{6pt plus 2pt minus 1pt}}
}{% if KOMA class
  \KOMAoptions{parskip=half}}
\makeatother
\usepackage{xcolor}
\usepackage[margin=1in]{geometry}
\usepackage{color}
\usepackage{fancyvrb}
\newcommand{\VerbBar}{|}
\newcommand{\VERB}{\Verb[commandchars=\\\{\}]}
\DefineVerbatimEnvironment{Highlighting}{Verbatim}{commandchars=\\\{\}}
% Add ',fontsize=\small' for more characters per line
\usepackage{framed}
\definecolor{shadecolor}{RGB}{248,248,248}
\newenvironment{Shaded}{\begin{snugshade}}{\end{snugshade}}
\newcommand{\AlertTok}[1]{\textcolor[rgb]{0.94,0.16,0.16}{#1}}
\newcommand{\AnnotationTok}[1]{\textcolor[rgb]{0.56,0.35,0.01}{\textbf{\textit{#1}}}}
\newcommand{\AttributeTok}[1]{\textcolor[rgb]{0.13,0.29,0.53}{#1}}
\newcommand{\BaseNTok}[1]{\textcolor[rgb]{0.00,0.00,0.81}{#1}}
\newcommand{\BuiltInTok}[1]{#1}
\newcommand{\CharTok}[1]{\textcolor[rgb]{0.31,0.60,0.02}{#1}}
\newcommand{\CommentTok}[1]{\textcolor[rgb]{0.56,0.35,0.01}{\textit{#1}}}
\newcommand{\CommentVarTok}[1]{\textcolor[rgb]{0.56,0.35,0.01}{\textbf{\textit{#1}}}}
\newcommand{\ConstantTok}[1]{\textcolor[rgb]{0.56,0.35,0.01}{#1}}
\newcommand{\ControlFlowTok}[1]{\textcolor[rgb]{0.13,0.29,0.53}{\textbf{#1}}}
\newcommand{\DataTypeTok}[1]{\textcolor[rgb]{0.13,0.29,0.53}{#1}}
\newcommand{\DecValTok}[1]{\textcolor[rgb]{0.00,0.00,0.81}{#1}}
\newcommand{\DocumentationTok}[1]{\textcolor[rgb]{0.56,0.35,0.01}{\textbf{\textit{#1}}}}
\newcommand{\ErrorTok}[1]{\textcolor[rgb]{0.64,0.00,0.00}{\textbf{#1}}}
\newcommand{\ExtensionTok}[1]{#1}
\newcommand{\FloatTok}[1]{\textcolor[rgb]{0.00,0.00,0.81}{#1}}
\newcommand{\FunctionTok}[1]{\textcolor[rgb]{0.13,0.29,0.53}{\textbf{#1}}}
\newcommand{\ImportTok}[1]{#1}
\newcommand{\InformationTok}[1]{\textcolor[rgb]{0.56,0.35,0.01}{\textbf{\textit{#1}}}}
\newcommand{\KeywordTok}[1]{\textcolor[rgb]{0.13,0.29,0.53}{\textbf{#1}}}
\newcommand{\NormalTok}[1]{#1}
\newcommand{\OperatorTok}[1]{\textcolor[rgb]{0.81,0.36,0.00}{\textbf{#1}}}
\newcommand{\OtherTok}[1]{\textcolor[rgb]{0.56,0.35,0.01}{#1}}
\newcommand{\PreprocessorTok}[1]{\textcolor[rgb]{0.56,0.35,0.01}{\textit{#1}}}
\newcommand{\RegionMarkerTok}[1]{#1}
\newcommand{\SpecialCharTok}[1]{\textcolor[rgb]{0.81,0.36,0.00}{\textbf{#1}}}
\newcommand{\SpecialStringTok}[1]{\textcolor[rgb]{0.31,0.60,0.02}{#1}}
\newcommand{\StringTok}[1]{\textcolor[rgb]{0.31,0.60,0.02}{#1}}
\newcommand{\VariableTok}[1]{\textcolor[rgb]{0.00,0.00,0.00}{#1}}
\newcommand{\VerbatimStringTok}[1]{\textcolor[rgb]{0.31,0.60,0.02}{#1}}
\newcommand{\WarningTok}[1]{\textcolor[rgb]{0.56,0.35,0.01}{\textbf{\textit{#1}}}}
\usepackage{graphicx}
\makeatletter
\def\maxwidth{\ifdim\Gin@nat@width>\linewidth\linewidth\else\Gin@nat@width\fi}
\def\maxheight{\ifdim\Gin@nat@height>\textheight\textheight\else\Gin@nat@height\fi}
\makeatother
% Scale images if necessary, so that they will not overflow the page
% margins by default, and it is still possible to overwrite the defaults
% using explicit options in \includegraphics[width, height, ...]{}
\setkeys{Gin}{width=\maxwidth,height=\maxheight,keepaspectratio}
% Set default figure placement to htbp
\makeatletter
\def\fps@figure{htbp}
\makeatother
\setlength{\emergencystretch}{3em} % prevent overfull lines
\providecommand{\tightlist}{%
  \setlength{\itemsep}{0pt}\setlength{\parskip}{0pt}}
\setcounter{secnumdepth}{-\maxdimen} % remove section numbering
\ifLuaTeX
  \usepackage{selnolig}  % disable illegal ligatures
\fi
\usepackage{bookmark}
\IfFileExists{xurl.sty}{\usepackage{xurl}}{} % add URL line breaks if available
\urlstyle{same}
\hypersetup{
  pdftitle={Mathematical practice final exam 2024},
  hidelinks,
  pdfcreator={LaTeX via pandoc}}

\title{Mathematical practice final exam 2024}
\author{}
\date{\vspace{-2.5em}2024-07-08}

\begin{document}
\maketitle

\subsection{1.}\label{section}

Solve the following system of equations using the \texttt{solve()}
function: \[
\left(\begin{array}{cccc}
9 & 4 & 12 & 2 \\
5 & 0 & 7 & 9 \\
2 & 6 & 8 & 0 \\
9 & 2 & 9 & 11
\end{array}\right) \times\left(\begin{array}{l}
x \\
y \\
z \\
t
\end{array}\right)=\left(\begin{array}{c}
7 \\
18 \\
1 \\
0
\end{array}\right)
\]

\begin{Shaded}
\begin{Highlighting}[]
\CommentTok{\# Define the coefficient matrix A}
\NormalTok{A }\OtherTok{\textless{}{-}} \FunctionTok{matrix}\NormalTok{(}\FunctionTok{c}\NormalTok{(}\DecValTok{9}\NormalTok{, }\DecValTok{4}\NormalTok{, }\DecValTok{12}\NormalTok{, }\DecValTok{2}\NormalTok{,}
              \DecValTok{5}\NormalTok{, }\DecValTok{0}\NormalTok{, }\DecValTok{7}\NormalTok{, }\DecValTok{9}\NormalTok{,}
              \DecValTok{2}\NormalTok{, }\DecValTok{6}\NormalTok{, }\DecValTok{8}\NormalTok{, }\DecValTok{0}\NormalTok{,}
              \DecValTok{9}\NormalTok{, }\DecValTok{2}\NormalTok{, }\DecValTok{9}\NormalTok{, }\DecValTok{11}\NormalTok{), }
            \AttributeTok{nrow =} \DecValTok{4}\NormalTok{, }\AttributeTok{byrow =} \ConstantTok{TRUE}\NormalTok{)}

\CommentTok{\# Define the right{-}hand side vector b}
\NormalTok{b }\OtherTok{\textless{}{-}} \FunctionTok{c}\NormalTok{(}\DecValTok{7}\NormalTok{, }\DecValTok{18}\NormalTok{, }\DecValTok{1}\NormalTok{, }\DecValTok{0}\NormalTok{)}

\CommentTok{\# Solve the system of equations}
\NormalTok{solution }\OtherTok{\textless{}{-}} \FunctionTok{solve}\NormalTok{(A, b)}

\CommentTok{\# Print the solution}
\FunctionTok{print}\NormalTok{(solution)}
\end{Highlighting}
\end{Shaded}

\begin{verbatim}
## [1] -4.2028571 -6.2285714  5.8471429 -0.2128571
\end{verbatim}

\subsection{2.}\label{section-1}

Execute the following lines which create two vectors of random integers
which are chosen with replacement from the integers
\(0, 1, \dots , 999\). Both vectors have length 250.

\begin{Shaded}
\begin{Highlighting}[]
\NormalTok{xVec }\OtherTok{\textless{}{-}} \FunctionTok{sample}\NormalTok{(}\DecValTok{0}\SpecialCharTok{:}\DecValTok{999}\NormalTok{, }\DecValTok{250}\NormalTok{, }\AttributeTok{replace=}\NormalTok{T)}
\NormalTok{yVec }\OtherTok{\textless{}{-}} \FunctionTok{sample}\NormalTok{(}\DecValTok{0}\SpecialCharTok{:}\DecValTok{999}\NormalTok{, }\DecValTok{250}\NormalTok{, }\AttributeTok{replace=}\NormalTok{T)}
\end{Highlighting}
\end{Shaded}

\begin{enumerate}
\def\labelenumi{(\alph{enumi})}
\tightlist
\item
  Create the vector \((y_2 - x_1, \cdots, y_n - x_{n-1}).\)
\end{enumerate}

\begin{Shaded}
\begin{Highlighting}[]
\CommentTok{\# Create the vector (y\_2 {-} x\_1, ..., y\_n {-} x\_\{n{-}1\})}
\NormalTok{newVec }\OtherTok{\textless{}{-}}\NormalTok{ yVec[}\SpecialCharTok{{-}}\DecValTok{1}\NormalTok{] }\SpecialCharTok{{-}}\NormalTok{ xVec[}\SpecialCharTok{{-}}\FunctionTok{length}\NormalTok{(xVec)]}
\FunctionTok{print}\NormalTok{(newVec)}
\end{Highlighting}
\end{Shaded}

\begin{verbatim}
##   [1] -428  -56  507  -42  -97  -21 -132 -592    2 -249 -254 -173  738 -171  758
##  [16]  -82 -439  512  332 -898  305 -117   52  269 -730  -99  114  315  339  106
##  [31] -415    3  325 -622  107  543   11  706  149  792  715  664 -561  887 -433
##  [46]  443 -685 -570  272 -364 -133  -64  706  222 -401   93 -328  280  260 -452
##  [61]  267  310  108 -358  -11  790 -258  527 -231 -253  -83  209    1  117  615
##  [76] -587  128 -326 -176 -210 -232 -446   -1  611 -710    9  319   97  -73  105
##  [91]   49  322 -367  534 -199  334 -406  455  576 -714 -165 -110  262  599 -398
## [106] -586 -526  461   45 -205   21 -209 -447 -167   16  -50  332  915 -239  898
## [121]  -70   17 -287 -113 -259  189  142 -116  790 -408  165  -57 -207  261 -160
## [136]  243  -55  891 -543  257  496  161  167 -572 -482 -398   29 -490  -14 -538
## [151] -348 -962   49 -462 -473 -234  -62  735 -249 -413 -405 -399 -110 -310  525
## [166] -226 -178  195 -525  559  374 -565 -206 -869 -430 -166  -48   48  263  218
## [181] -638 -215  257   -7 -154 -131  -18  131  187 -318  -22 -133 -333  162 -130
## [196] -664 -810 -254  785   48  404  253  -39 -112 -465 -167  -19 -241   48   36
## [211]  433  827 -666 -213  377 -353 -786 -435  744   76 -469 -455 -450 -320  692
## [226] -209  -12 -446  -10 -250 -281  416 -206 -762  416  266 -331  -18 -185 -460
## [241]  -71  593 -165 -555 -201  211  489 -852 -363
\end{verbatim}

\begin{enumerate}
\def\labelenumi{(\alph{enumi})}
\setcounter{enumi}{1}
\tightlist
\item
  Pick out the values in yVec which are \(> 600\).
\end{enumerate}

\begin{Shaded}
\begin{Highlighting}[]
\CommentTok{\# Values in yVec greater than 600}
\NormalTok{values\_gt\_600 }\OtherTok{\textless{}{-}}\NormalTok{ yVec[yVec }\SpecialCharTok{\textgreater{}} \DecValTok{600}\NormalTok{]}
\FunctionTok{print}\NormalTok{(values\_gt\_600)}
\end{Highlighting}
\end{Shaded}

\begin{verbatim}
##  [1] 675 712 735 661 847 799 671 965 993 629 917 936 998 975 768 717 889 727 999
## [20] 860 888 665 909 646 684 920 914 760 663 722 795 785 642 784 824 714 815 787
## [39] 976 868 986 713 654 677 772 812 788 946 900 710 806 800 943 954 627 813 908
## [58] 755 968 638 705 937 952 969 721 666 830 734 610 888 847 787 852 722 658 971
## [77] 895 898 904 759 759 699 976 737 909 715 663 988 735 850 726 961
\end{verbatim}

\begin{enumerate}
\def\labelenumi{(\alph{enumi})}
\setcounter{enumi}{2}
\tightlist
\item
  What are the index positions in yVec of the values which are
  \(> 600\)?
\end{enumerate}

\begin{Shaded}
\begin{Highlighting}[]
\CommentTok{\# Index positions of values in yVec greater than 600}
\NormalTok{indices\_gt\_600 }\OtherTok{\textless{}{-}} \FunctionTok{which}\NormalTok{(yVec }\SpecialCharTok{\textgreater{}} \DecValTok{600}\NormalTok{)}
\FunctionTok{print}\NormalTok{(indices\_gt\_600)}
\end{Highlighting}
\end{Shaded}

\begin{verbatim}
##  [1]   3   4   5   8  14  16  19  29  34  36  37  39  40  41  42  43  45  47  50
## [20]  52  54  57  59  63  66  67  69  75  76  80  85  87  88  89  90  91  93  95
## [39]  97  99 100 103 104 105 109 117 118 119 121 123 130 133 135 139 141 143 144
## [58] 157 159 166 167 169 171 172 179 185 189 190 195 200 203 204 205 207 208 210
## [77] 211 212 213 216 220 221 226 227 228 230 233 237 239 243 246 248
\end{verbatim}

\begin{enumerate}
\def\labelenumi{(\alph{enumi})}
\setcounter{enumi}{3}
\tightlist
\item
  Sort the numbers in the vector xVec in the order of increasing values
  in yVec.
\end{enumerate}

\begin{Shaded}
\begin{Highlighting}[]
\CommentTok{\# Sort xVec according to the order of yVec}
\NormalTok{sorted\_xVec }\OtherTok{\textless{}{-}}\NormalTok{ xVec[}\FunctionTok{order}\NormalTok{(yVec)]}
\FunctionTok{print}\NormalTok{(sorted\_xVec)}
\end{Highlighting}
\end{Shaded}

\begin{verbatim}
##   [1] 314 653 727 683 629 288 188 230 802 577 393 695 349 413 989 402 284  59
##  [19] 821 862 114 657  16 916 582 927 586 659 938 358 508 326  89  34  15 182
##  [37] 499 889 284 868 382 543 356 736 103 448 557 173 878 994 793 988 535 763
##  [55] 783 995 159 386 109  41  71 776 842 727 493 399   2 370 184 725 708 472
##  [73] 253 614 865   2 699  26 113 362 533 588 175 620 622 682 834 643 522 923
##  [91] 233 574 816 639 662 993 586 650 740 627 266 898 225 963 266 753 387 598
## [109] 668  63 284 633 857 732 693 899 336 397 404 661 825 356  16 673 315 311
## [127] 149 731 594 212 255 257 799 151 742 751 572 749 247  67 125 333 534 255
## [145] 151 980 823 811 321 457 467 673 642 722 933 732 652 482 584 101 374 931
## [163] 687 109  78 573 713 679 303 343 668 224 205 929 130 842 700 425 777 392
## [181] 164 332 828 286 495 677  32 785 893 125 697 921 396 462 623  48  53 409
## [199] 897 323 780 964  31 902 380 505 798 456 741 611 609 547 360 826 258 593
## [217] 188 410 368  72 564 465  77 509 713 826 109 680 341  24 653 849 565 710
## [235] 457 595 763 957 216 814 748 859  53 458 946 916 724 954 183 728
\end{verbatim}

\begin{enumerate}
\def\labelenumi{(\alph{enumi})}
\setcounter{enumi}{4}
\tightlist
\item
  Pick out the elements in yVec at index positions
  \(1, 4, 7, 10, 13, \cdots\)
\end{enumerate}

\begin{Shaded}
\begin{Highlighting}[]
\CommentTok{\# Elements in yVec at positions 1, 4, 7, 10, ...}
\NormalTok{indices }\OtherTok{\textless{}{-}} \FunctionTok{seq}\NormalTok{(}\DecValTok{1}\NormalTok{, }\FunctionTok{length}\NormalTok{(yVec), }\AttributeTok{by =} \DecValTok{3}\NormalTok{)}
\NormalTok{elements\_at\_indices }\OtherTok{\textless{}{-}}\NormalTok{ yVec[indices]}
\FunctionTok{print}\NormalTok{(elements\_at\_indices)}
\end{Highlighting}
\end{Shaded}

\begin{verbatim}
##  [1] 441 712 167  36 189 799 671 364 595 369 257 993 917 998 717 131  12 860 590
## [20]  15 288 217 920 110 464 663 382 301 795 642 714 244 976 986 713 531 772  37
## [39] 189 788 900 138 455 806 800 550 954 597 254 178 145 363 755 565 443 638 937
## [58] 969 126 460 343 578  20 734 599 454 148 493 852 658 895  47 109 759 172 976
## [77] 234 540 154 393 339  93 243 296
\end{verbatim}

\subsection{3.}\label{section-2}

For this problem we'll use the (built-in) dataset state.x77.

\begin{Shaded}
\begin{Highlighting}[]
\FunctionTok{data}\NormalTok{(state)}
\NormalTok{state.x77 }\OtherTok{\textless{}{-}} \FunctionTok{as\_tibble}\NormalTok{(state.x77, }\AttributeTok{rownames  =} \StringTok{\textquotesingle{}State\textquotesingle{}}\NormalTok{)}
\end{Highlighting}
\end{Shaded}

\begin{enumerate}
\def\labelenumi{\alph{enumi}.}
\tightlist
\item
  Select all the states having an income less than 4300, and calculate
  the average income of these states.
\end{enumerate}

\begin{Shaded}
\begin{Highlighting}[]
\CommentTok{\# Select states with income less than 4300}
\NormalTok{low\_income\_states }\OtherTok{\textless{}{-}}\NormalTok{ state.x77 }\SpecialCharTok{\%\textgreater{}\%} \FunctionTok{filter}\NormalTok{(Income }\SpecialCharTok{\textless{}} \DecValTok{4300}\NormalTok{)}

\CommentTok{\# Calculate the average income of these states}
\NormalTok{average\_low\_income }\OtherTok{\textless{}{-}} \FunctionTok{mean}\NormalTok{(low\_income\_states}\SpecialCharTok{$}\NormalTok{Income)}
\NormalTok{average\_low\_income}
\end{Highlighting}
\end{Shaded}

\begin{verbatim}
## [1] 3830.6
\end{verbatim}

\begin{enumerate}
\def\labelenumi{\alph{enumi}.}
\setcounter{enumi}{1}
\tightlist
\item
  Sort the data by income and select the state with the highest income.
\end{enumerate}

\begin{Shaded}
\begin{Highlighting}[]
\CommentTok{\# Sort the data by income}
\NormalTok{sorted\_states }\OtherTok{\textless{}{-}}\NormalTok{ state.x77 }\SpecialCharTok{\%\textgreater{}\%} \FunctionTok{arrange}\NormalTok{(}\FunctionTok{desc}\NormalTok{(Income))}

\CommentTok{\# Select the state with the highest income}
\NormalTok{highest\_income\_state }\OtherTok{\textless{}{-}}\NormalTok{ sorted\_states }\SpecialCharTok{\%\textgreater{}\%} \FunctionTok{slice}\NormalTok{(}\DecValTok{1}\NormalTok{)}
\NormalTok{highest\_income\_state}
\end{Highlighting}
\end{Shaded}

\begin{verbatim}
## # A tibble: 1 x 9
##   State  Population Income Illiteracy `Life Exp` Murder `HS Grad` Frost   Area
##   <chr>       <dbl>  <dbl>      <dbl>      <dbl>  <dbl>     <dbl> <dbl>  <dbl>
## 1 Alaska        365   6315        1.5       69.3   11.3      66.7   152 566432
\end{verbatim}

\begin{enumerate}
\def\labelenumi{\alph{enumi}.}
\setcounter{enumi}{2}
\tightlist
\item
  Add a variable to the data frame which categorizes the size of
  population: \(<= 4500\) is \texttt{S}, \$ \textgreater{} 4500 \$ is
  \texttt{L}.
\end{enumerate}

\begin{Shaded}
\begin{Highlighting}[]
\CommentTok{\# Add a variable categorizing the size of population}
\NormalTok{state.x77 }\OtherTok{\textless{}{-}}\NormalTok{ state.x77 }\SpecialCharTok{\%\textgreater{}\%} 
  \FunctionTok{mutate}\NormalTok{(}\AttributeTok{PopulationSize =} \FunctionTok{ifelse}\NormalTok{(Population }\SpecialCharTok{\textless{}=} \DecValTok{4500}\NormalTok{, }\StringTok{\textquotesingle{}S\textquotesingle{}}\NormalTok{, }\StringTok{\textquotesingle{}L\textquotesingle{}}\NormalTok{))}
\FunctionTok{head}\NormalTok{(state.x77)}
\end{Highlighting}
\end{Shaded}

\begin{verbatim}
## # A tibble: 6 x 10
##   State    Population Income Illiteracy `Life Exp` Murder `HS Grad` Frost   Area
##   <chr>         <dbl>  <dbl>      <dbl>      <dbl>  <dbl>     <dbl> <dbl>  <dbl>
## 1 Alabama        3615   3624        2.1       69.0   15.1      41.3    20  50708
## 2 Alaska          365   6315        1.5       69.3   11.3      66.7   152 566432
## 3 Arizona        2212   4530        1.8       70.6    7.8      58.1    15 113417
## 4 Arkansas       2110   3378        1.9       70.7   10.1      39.9    65  51945
## 5 Califor~      21198   5114        1.1       71.7   10.3      62.6    20 156361
## 6 Colorado       2541   4884        0.7       72.1    6.8      63.9   166 103766
## # i 1 more variable: PopulationSize <chr>
\end{verbatim}

\begin{enumerate}
\def\labelenumi{\alph{enumi}.}
\setcounter{enumi}{3}
\tightlist
\item
  Find out the average income and illiteracy of the two groups of
  states, distinguishing by whether the states are small or large.
\end{enumerate}

\begin{Shaded}
\begin{Highlighting}[]
\CommentTok{\# Find the average income and illiteracy of the two groups of states}
\NormalTok{grouped\_stats }\OtherTok{\textless{}{-}}\NormalTok{ state.x77 }\SpecialCharTok{\%\textgreater{}\%} 
  \FunctionTok{group\_by}\NormalTok{(PopulationSize) }\SpecialCharTok{\%\textgreater{}\%} 
  \FunctionTok{summarise}\NormalTok{(}
    \AttributeTok{Average\_Income =} \FunctionTok{mean}\NormalTok{(Income),}
    \AttributeTok{Average\_Illiteracy =} \FunctionTok{mean}\NormalTok{(Illiteracy)}
\NormalTok{  )}

\NormalTok{grouped\_stats}
\end{Highlighting}
\end{Shaded}

\begin{verbatim}
## # A tibble: 2 x 3
##   PopulationSize Average_Income Average_Illiteracy
##   <chr>                   <dbl>              <dbl>
## 1 L                       4608.               1.2 
## 2 S                       4355.               1.16
\end{verbatim}

\subsection{4.}\label{section-3}

\begin{enumerate}
\def\labelenumi{\alph{enumi}.}
\tightlist
\item
  Write a function to simulate \texttt{n} observations of \((X_1, X_2)\)
  which follow the uniform distribution over the square
  \([0, 1] \times [0, 1]\).
\end{enumerate}

\begin{Shaded}
\begin{Highlighting}[]
\NormalTok{simulate\_observations }\OtherTok{\textless{}{-}} \ControlFlowTok{function}\NormalTok{(n) \{}
\NormalTok{  X1 }\OtherTok{\textless{}{-}} \FunctionTok{runif}\NormalTok{(n, }\DecValTok{0}\NormalTok{, }\DecValTok{1}\NormalTok{)}
\NormalTok{  X2 }\OtherTok{\textless{}{-}} \FunctionTok{runif}\NormalTok{(n, }\DecValTok{0}\NormalTok{, }\DecValTok{1}\NormalTok{)}
  \FunctionTok{return}\NormalTok{(}\FunctionTok{data.frame}\NormalTok{(X1, X2))}
\NormalTok{\}}
\end{Highlighting}
\end{Shaded}

\begin{enumerate}
\def\labelenumi{\alph{enumi}.}
\setcounter{enumi}{1}
\tightlist
\item
  Write a function to calculate the proportion of the observations that
  the distance between \((X_1, X_2)\) and the nearest edge is less than
  0.25, and the proportion of them with the distance to the nearest
  vertex less than 0.25.
\end{enumerate}

\begin{Shaded}
\begin{Highlighting}[]
\NormalTok{calculate\_proportions }\OtherTok{\textless{}{-}} \ControlFlowTok{function}\NormalTok{(observations) \{}
\NormalTok{  n }\OtherTok{\textless{}{-}} \FunctionTok{nrow}\NormalTok{(observations)}
\NormalTok{  X1 }\OtherTok{\textless{}{-}}\NormalTok{ observations}\SpecialCharTok{$}\NormalTok{X1}
\NormalTok{  X2 }\OtherTok{\textless{}{-}}\NormalTok{ observations}\SpecialCharTok{$}\NormalTok{X2}
  
  \CommentTok{\# Distance to the nearest edge}
\NormalTok{  edge\_distances }\OtherTok{\textless{}{-}} \FunctionTok{pmin}\NormalTok{(X1, }\DecValTok{1} \SpecialCharTok{{-}}\NormalTok{ X1, X2, }\DecValTok{1} \SpecialCharTok{{-}}\NormalTok{ X2)}
\NormalTok{  edge\_proportion }\OtherTok{\textless{}{-}} \FunctionTok{mean}\NormalTok{(edge\_distances }\SpecialCharTok{\textless{}} \FloatTok{0.25}\NormalTok{)}
  
  \CommentTok{\# Distance to the nearest vertex}
\NormalTok{  vertex\_distances }\OtherTok{\textless{}{-}} \FunctionTok{pmin}\NormalTok{(}
    \FunctionTok{sqrt}\NormalTok{((X1 }\SpecialCharTok{{-}} \DecValTok{0}\NormalTok{)}\SpecialCharTok{\^{}}\DecValTok{2} \SpecialCharTok{+}\NormalTok{ (X2 }\SpecialCharTok{{-}} \DecValTok{0}\NormalTok{)}\SpecialCharTok{\^{}}\DecValTok{2}\NormalTok{),  }\CommentTok{\# distance to (0,0)}
    \FunctionTok{sqrt}\NormalTok{((X1 }\SpecialCharTok{{-}} \DecValTok{0}\NormalTok{)}\SpecialCharTok{\^{}}\DecValTok{2} \SpecialCharTok{+}\NormalTok{ (X2 }\SpecialCharTok{{-}} \DecValTok{1}\NormalTok{)}\SpecialCharTok{\^{}}\DecValTok{2}\NormalTok{),  }\CommentTok{\# distance to (0,1)}
    \FunctionTok{sqrt}\NormalTok{((X1 }\SpecialCharTok{{-}} \DecValTok{1}\NormalTok{)}\SpecialCharTok{\^{}}\DecValTok{2} \SpecialCharTok{+}\NormalTok{ (X2 }\SpecialCharTok{{-}} \DecValTok{0}\NormalTok{)}\SpecialCharTok{\^{}}\DecValTok{2}\NormalTok{),  }\CommentTok{\# distance to (1,0)}
    \FunctionTok{sqrt}\NormalTok{((X1 }\SpecialCharTok{{-}} \DecValTok{1}\NormalTok{)}\SpecialCharTok{\^{}}\DecValTok{2} \SpecialCharTok{+}\NormalTok{ (X2 }\SpecialCharTok{{-}} \DecValTok{1}\NormalTok{)}\SpecialCharTok{\^{}}\DecValTok{2}\NormalTok{)   }\CommentTok{\# distance to (1,1)}
\NormalTok{  )}
\NormalTok{  vertex\_proportion }\OtherTok{\textless{}{-}} \FunctionTok{mean}\NormalTok{(vertex\_distances }\SpecialCharTok{\textless{}} \FloatTok{0.25}\NormalTok{)}
  
  \FunctionTok{return}\NormalTok{(}\FunctionTok{list}\NormalTok{(}\AttributeTok{edge\_proportion =}\NormalTok{ edge\_proportion, }\AttributeTok{vertex\_proportion =}\NormalTok{ vertex\_proportion))}
\NormalTok{\}}
\end{Highlighting}
\end{Shaded}

\paragraph{Example Usage:}\label{example-usage}

Here's how you can use the functions to simulate observations and
calculate the proportions:

\begin{Shaded}
\begin{Highlighting}[]
\CommentTok{\# Simulate 1000 observations}
\FunctionTok{set.seed}\NormalTok{(}\DecValTok{123}\NormalTok{)  }\CommentTok{\# Setting seed for reproducibility}
\NormalTok{observations }\OtherTok{\textless{}{-}} \FunctionTok{simulate\_observations}\NormalTok{(}\DecValTok{1000}\NormalTok{)}

\CommentTok{\# Calculate proportions}
\NormalTok{proportions }\OtherTok{\textless{}{-}} \FunctionTok{calculate\_proportions}\NormalTok{(observations)}
\FunctionTok{print}\NormalTok{(proportions)}
\end{Highlighting}
\end{Shaded}

\begin{verbatim}
## $edge_proportion
## [1] 0.735
## 
## $vertex_proportion
## [1] 0.198
\end{verbatim}

\subsection{5.}\label{section-4}

To estimate \(\pi\) with a Monte Carlo simulation, we draw the unit
circle inside the unit square, the ratio of the area of the circle to
the area of the square will be \(\pi / 4\). Then shot \(K\) arrows at
the square, roughly \(K * \pi / 4\) should have fallen inside the
circle. So if now you shoot \(N\) arrows at the square, and \(M\) fall
inside the circle, you have the following relationship
\(M = N * \pi / 4\). You can thus compute \(\pi\) like so:
\(\pi = 4 * M / N\). The more arrows \(N\) you throw at the square, the
better approximation of \(\pi\) you'll have.

\begin{Shaded}
\begin{Highlighting}[]
\FunctionTok{library}\NormalTok{(tibble)}
\FunctionTok{library}\NormalTok{(dplyr)}
\FunctionTok{library}\NormalTok{(purrr)}
\FunctionTok{library}\NormalTok{(ggplot2)}

\NormalTok{n }\OtherTok{\textless{}{-}} \DecValTok{10000}

\FunctionTok{set.seed}\NormalTok{(}\DecValTok{1}\NormalTok{)}
\NormalTok{points }\OtherTok{\textless{}{-}} \FunctionTok{tibble}\NormalTok{(}\StringTok{"x"} \OtherTok{=} \FunctionTok{runif}\NormalTok{(n), }\StringTok{"y"} \OtherTok{=} \FunctionTok{runif}\NormalTok{(n))}
\end{Highlighting}
\end{Shaded}

Now, to know if a point is inside the unit circle, we need to check
whether \(x^2 + y^2 < 1\). Let's add a new column to the points tibble,
called \texttt{inside} equal to \texttt{1} if the point is inside the
unit circle and \texttt{0} if not:

\begin{Shaded}
\begin{Highlighting}[]
\NormalTok{points }\OtherTok{\textless{}{-}}\NormalTok{ points }\SpecialCharTok{|\textgreater{}} 
    \FunctionTok{mutate}\NormalTok{(}\AttributeTok{inside =} \FunctionTok{map2\_dbl}\NormalTok{(}\AttributeTok{.x =}\NormalTok{ x, }\AttributeTok{.y =}\NormalTok{ y, }\SpecialCharTok{\textasciitilde{}}\FunctionTok{ifelse}\NormalTok{(.x}\SpecialCharTok{**}\DecValTok{2} \SpecialCharTok{+}\NormalTok{ .y}\SpecialCharTok{**}\DecValTok{2} \SpecialCharTok{\textless{}} \DecValTok{1}\NormalTok{, }\DecValTok{1}\NormalTok{, }\DecValTok{0}\NormalTok{))) }\SpecialCharTok{|\textgreater{}} 
    \FunctionTok{rowid\_to\_column}\NormalTok{(}\StringTok{"N"}\NormalTok{)}
\end{Highlighting}
\end{Shaded}

\begin{enumerate}
\def\labelenumi{\alph{enumi}.}
\tightlist
\item
  Compute the estimation of \(\pi\) at each row, by computing the
  cumulative sum of the 1's in the \texttt{inside} column and dividing
  that by the current value of \texttt{N} column:
\end{enumerate}

\begin{Shaded}
\begin{Highlighting}[]
\CommentTok{\# Compute the cumulative sum of the \textasciigrave{}inside\textasciigrave{} column}
\NormalTok{points }\OtherTok{\textless{}{-}}\NormalTok{ points }\SpecialCharTok{\%\textgreater{}\%}
  \FunctionTok{mutate}\NormalTok{(}\AttributeTok{cumulative\_inside =} \FunctionTok{cumsum}\NormalTok{(inside),}
         \AttributeTok{pi\_estimate =} \DecValTok{4} \SpecialCharTok{*}\NormalTok{ cumulative\_inside }\SpecialCharTok{/}\NormalTok{ N)}

\FunctionTok{head}\NormalTok{(points)}
\end{Highlighting}
\end{Shaded}

\begin{verbatim}
## # A tibble: 6 x 6
##       N     x      y inside cumulative_inside pi_estimate
##   <int> <dbl>  <dbl>  <dbl>             <dbl>       <dbl>
## 1     1 0.266 0.0647      1                 1           4
## 2     2 0.372 0.677       1                 2           4
## 3     3 0.573 0.735       1                 3           4
## 4     4 0.908 0.111       1                 4           4
## 5     5 0.202 0.0467      1                 5           4
## 6     6 0.898 0.131       1                 6           4
\end{verbatim}

\begin{enumerate}
\def\labelenumi{\alph{enumi}.}
\setcounter{enumi}{1}
\tightlist
\item
  Plot the estimates of \(\pi\) against \texttt{N}.
\end{enumerate}

\begin{Shaded}
\begin{Highlighting}[]
\FunctionTok{ggplot}\NormalTok{(points, }\FunctionTok{aes}\NormalTok{(}\AttributeTok{x =}\NormalTok{ N, }\AttributeTok{y =}\NormalTok{ pi\_estimate)) }\SpecialCharTok{+}
  \FunctionTok{geom\_line}\NormalTok{() }\SpecialCharTok{+}
  \FunctionTok{geom\_hline}\NormalTok{(}\AttributeTok{yintercept =}\NormalTok{ pi, }\AttributeTok{col =} \StringTok{"red"}\NormalTok{, }\AttributeTok{linetype =} \StringTok{"dashed"}\NormalTok{) }\SpecialCharTok{+}
  \FunctionTok{labs}\NormalTok{(}\AttributeTok{title =} \StringTok{"Estimation of Pi using Monte Carlo Simulation"}\NormalTok{,}
       \AttributeTok{x =} \StringTok{"Number of Points (N)"}\NormalTok{,}
       \AttributeTok{y =} \StringTok{"Estimation of Pi"}\NormalTok{) }\SpecialCharTok{+}
  \FunctionTok{theme\_minimal}\NormalTok{()}
\end{Highlighting}
\end{Shaded}

\includegraphics{Final24_files/figure-latex/unnamed-chunk-20-1.pdf}

\subsection{6.}\label{section-5}

Mortality rates per 100,000 from male suicides for a number of age
groups and a number of countries are given in the following data frame.

\begin{Shaded}
\begin{Highlighting}[]
\FunctionTok{library}\NormalTok{(tidyverse)}
\NormalTok{ suicrates }\OtherTok{\textless{}{-}} \FunctionTok{tibble}\NormalTok{(}\AttributeTok{Country =} \FunctionTok{c}\NormalTok{(}\StringTok{\textquotesingle{}Canada\textquotesingle{}}\NormalTok{, }\StringTok{\textquotesingle{}Israel\textquotesingle{}}\NormalTok{, }\StringTok{\textquotesingle{}Japan\textquotesingle{}}\NormalTok{, }\StringTok{\textquotesingle{}Austria\textquotesingle{}}\NormalTok{, }\StringTok{\textquotesingle{}France\textquotesingle{}}\NormalTok{, }\StringTok{\textquotesingle{}Germany\textquotesingle{}}\NormalTok{,}
 \StringTok{\textquotesingle{}Hungary\textquotesingle{}}\NormalTok{, }\StringTok{\textquotesingle{}Italy\textquotesingle{}}\NormalTok{, }\StringTok{\textquotesingle{}Netherlands\textquotesingle{}}\NormalTok{, }\StringTok{\textquotesingle{}Poland\textquotesingle{}}\NormalTok{, }\StringTok{\textquotesingle{}Spain\textquotesingle{}}\NormalTok{, }\StringTok{\textquotesingle{}Sweden\textquotesingle{}}\NormalTok{, }\StringTok{\textquotesingle{}Switzerland\textquotesingle{}}\NormalTok{, }\StringTok{\textquotesingle{}UK\textquotesingle{}}\NormalTok{, }\StringTok{\textquotesingle{}USA\textquotesingle{}}\NormalTok{),}
 \AttributeTok{Age25.34 =} \FunctionTok{c}\NormalTok{(}\DecValTok{22}\NormalTok{, }\DecValTok{9}\NormalTok{, }\DecValTok{22}\NormalTok{, }\DecValTok{29}\NormalTok{, }\DecValTok{16}\NormalTok{, }\DecValTok{28}\NormalTok{, }\DecValTok{48}\NormalTok{, }\DecValTok{7}\NormalTok{, }\DecValTok{8}\NormalTok{, }\DecValTok{26}\NormalTok{, }\DecValTok{4}\NormalTok{, }\DecValTok{28}\NormalTok{, }\DecValTok{22}\NormalTok{, }\DecValTok{10}\NormalTok{, }\DecValTok{20}\NormalTok{),}
 \AttributeTok{Age35.44 =} \FunctionTok{c}\NormalTok{(}\DecValTok{27}\NormalTok{, }\DecValTok{19}\NormalTok{, }\DecValTok{19}\NormalTok{, }\DecValTok{40}\NormalTok{, }\DecValTok{25}\NormalTok{, }\DecValTok{35}\NormalTok{, }\DecValTok{65}\NormalTok{, }\DecValTok{8}\NormalTok{, }\DecValTok{11}\NormalTok{, }\DecValTok{29}\NormalTok{, }\DecValTok{7}\NormalTok{, }\DecValTok{41}\NormalTok{, }\DecValTok{34}\NormalTok{, }\DecValTok{13}\NormalTok{, }\DecValTok{22}\NormalTok{),}
 \AttributeTok{Age45.54 =} \FunctionTok{c}\NormalTok{(}\DecValTok{31}\NormalTok{, }\DecValTok{10}\NormalTok{, }\DecValTok{21}\NormalTok{, }\DecValTok{52}\NormalTok{, }\DecValTok{36}\NormalTok{, }\DecValTok{41}\NormalTok{, }\DecValTok{84}\NormalTok{, }\DecValTok{11}\NormalTok{, }\DecValTok{18}\NormalTok{, }\DecValTok{36}\NormalTok{, }\DecValTok{10}\NormalTok{, }\DecValTok{46}\NormalTok{, }\DecValTok{41}\NormalTok{, }\DecValTok{15}\NormalTok{, }\DecValTok{28}\NormalTok{),}
 \AttributeTok{Age55.64 =} \FunctionTok{c}\NormalTok{(}\DecValTok{34}\NormalTok{, }\DecValTok{14}\NormalTok{, }\DecValTok{31}\NormalTok{, }\DecValTok{53}\NormalTok{, }\DecValTok{47}\NormalTok{, }\DecValTok{49}\NormalTok{, }\DecValTok{81}\NormalTok{, }\DecValTok{18}\NormalTok{, }\DecValTok{20}\NormalTok{, }\DecValTok{32}\NormalTok{, }\DecValTok{16}\NormalTok{, }\DecValTok{51}\NormalTok{, }\DecValTok{50}\NormalTok{, }\DecValTok{17}\NormalTok{, }\DecValTok{33}\NormalTok{),}
 \AttributeTok{Age65.74 =} \FunctionTok{c}\NormalTok{(}\DecValTok{24}\NormalTok{, }\DecValTok{27}\NormalTok{, }\DecValTok{49}\NormalTok{, }\DecValTok{69}\NormalTok{, }\DecValTok{56}\NormalTok{, }\DecValTok{52}\NormalTok{, }\DecValTok{107}\NormalTok{, }\DecValTok{27}\NormalTok{, }\DecValTok{28}\NormalTok{, }\DecValTok{28}\NormalTok{, }\DecValTok{22}\NormalTok{, }\DecValTok{35}\NormalTok{, }\DecValTok{51}\NormalTok{, }\DecValTok{22}\NormalTok{, }\DecValTok{37}\NormalTok{))}
\end{Highlighting}
\end{Shaded}

\begin{enumerate}
\def\labelenumi{\alph{enumi}.}
\tightlist
\item
  Transform \texttt{suicrates} into \emph{long} form.
\end{enumerate}

\begin{Shaded}
\begin{Highlighting}[]
\FunctionTok{library}\NormalTok{(tidyr)}

\CommentTok{\# Transforming the data to long form}
\NormalTok{suicrates\_long }\OtherTok{\textless{}{-}}\NormalTok{ suicrates }\SpecialCharTok{\%\textgreater{}\%}
  \FunctionTok{pivot\_longer}\NormalTok{(}\AttributeTok{cols =} \FunctionTok{starts\_with}\NormalTok{(}\StringTok{"Age"}\NormalTok{), }\AttributeTok{names\_to =} \StringTok{"AgeGroup"}\NormalTok{, }\AttributeTok{values\_to =} \StringTok{"Rate"}\NormalTok{)}

\FunctionTok{print}\NormalTok{(suicrates\_long)}
\end{Highlighting}
\end{Shaded}

\begin{verbatim}
## # A tibble: 75 x 3
##    Country AgeGroup  Rate
##    <chr>   <chr>    <dbl>
##  1 Canada  Age25.34    22
##  2 Canada  Age35.44    27
##  3 Canada  Age45.54    31
##  4 Canada  Age55.64    34
##  5 Canada  Age65.74    24
##  6 Israel  Age25.34     9
##  7 Israel  Age35.44    19
##  8 Israel  Age45.54    10
##  9 Israel  Age55.64    14
## 10 Israel  Age65.74    27
## # i 65 more rows
\end{verbatim}

\begin{enumerate}
\def\labelenumi{\alph{enumi}.}
\setcounter{enumi}{1}
\tightlist
\item
  Construct side-by-side box plots for the data from different age
  groups, and comment on what the graphic tells us about the data.
\end{enumerate}

\begin{Shaded}
\begin{Highlighting}[]
 \FunctionTok{library}\NormalTok{(ggplot2)}

\CommentTok{\# Constructing side{-}by{-}side box plots}
\FunctionTok{ggplot}\NormalTok{(suicrates\_long, }\FunctionTok{aes}\NormalTok{(}\AttributeTok{x =}\NormalTok{ AgeGroup, }\AttributeTok{y =}\NormalTok{ Rate)) }\SpecialCharTok{+}
  \FunctionTok{geom\_boxplot}\NormalTok{() }\SpecialCharTok{+}
  \FunctionTok{labs}\NormalTok{(}\AttributeTok{title =} \StringTok{"Suicide Rates by Age Group"}\NormalTok{,}
       \AttributeTok{x =} \StringTok{"Age Group"}\NormalTok{,}
       \AttributeTok{y =} \StringTok{"Suicide Rate per 100,000"}\NormalTok{) }\SpecialCharTok{+}
  \FunctionTok{theme\_minimal}\NormalTok{()}
\end{Highlighting}
\end{Shaded}

\includegraphics{Final24_files/figure-latex/unnamed-chunk-23-1.pdf}

\subsection{7.}\label{section-6}

Load the \texttt{LaborSupply} dataset from the \texttt{\{Ecdat\}}
package and answer the following questions:

\begin{Shaded}
\begin{Highlighting}[]
\CommentTok{\#data(LaborSupply)}
\NormalTok{LaborSupply }\OtherTok{\textless{}{-}} \FunctionTok{read\_csv}\NormalTok{(}\StringTok{"LaborSupply.csv"}\NormalTok{)}
\end{Highlighting}
\end{Shaded}

\begin{verbatim}
## Rows: 5320 Columns: 7
## -- Column specification --------------------------------------------------------
## Delimiter: ","
## dbl (7): lnhr, lnwg, kids, age, disab, id, year
## 
## i Use `spec()` to retrieve the full column specification for this data.
## i Specify the column types or set `show_col_types = FALSE` to quiet this message.
\end{verbatim}

\begin{Shaded}
\begin{Highlighting}[]
\CommentTok{\# create hour and wage variables}
\NormalTok{labor }\OtherTok{\textless{}{-}}\NormalTok{ LaborSupply }\SpecialCharTok{|\textgreater{}} 
  \FunctionTok{mutate}\NormalTok{(}\AttributeTok{hour =} \FunctionTok{exp}\NormalTok{(lnhr), }\AttributeTok{wage =} \FunctionTok{exp}\NormalTok{(lnwg), }\AttributeTok{.before =}\NormalTok{ kids) }\SpecialCharTok{|\textgreater{}} 
\NormalTok{  dplyr}\SpecialCharTok{::}\FunctionTok{select}\NormalTok{(}\SpecialCharTok{{-}}\NormalTok{lnhr, }\SpecialCharTok{{-}}\NormalTok{lnwg)}
\end{Highlighting}
\end{Shaded}

\begin{enumerate}
\def\labelenumi{\alph{enumi}.}
\tightlist
\item
  Compute the average annual hours worked and their standard deviations
  by year.
\end{enumerate}

\begin{Shaded}
\begin{Highlighting}[]
\CommentTok{\# Compute the average annual hours worked and their standard deviations by year}
\NormalTok{average\_hours\_by\_year }\OtherTok{\textless{}{-}}\NormalTok{ labor }\SpecialCharTok{\%\textgreater{}\%}
  \FunctionTok{group\_by}\NormalTok{(year) }\SpecialCharTok{\%\textgreater{}\%}
  \FunctionTok{summarize}\NormalTok{(}
    \AttributeTok{avg\_hours =} \FunctionTok{mean}\NormalTok{(hour, }\AttributeTok{na.rm =} \ConstantTok{TRUE}\NormalTok{),}
    \AttributeTok{sd\_hours =} \FunctionTok{sd}\NormalTok{(hour, }\AttributeTok{na.rm =} \ConstantTok{TRUE}\NormalTok{)}
\NormalTok{  )}

\FunctionTok{print}\NormalTok{(average\_hours\_by\_year)}
\end{Highlighting}
\end{Shaded}

\begin{verbatim}
## # A tibble: 10 x 3
##     year avg_hours sd_hours
##    <dbl>     <dbl>    <dbl>
##  1  1979     2202.     502.
##  2  1980     2182.     454.
##  3  1981     2185.     460.
##  4  1982     2145.     442.
##  5  1983     2124.     550.
##  6  1984     2149.     492.
##  7  1985     2203.     515.
##  8  1986     2195.     482.
##  9  1987     2219.     529.
## 10  1988     2222.     478.
\end{verbatim}

\begin{enumerate}
\def\labelenumi{\alph{enumi}.}
\setcounter{enumi}{1}
\tightlist
\item
  What age group worked the most hours in the year 1982?
\end{enumerate}

\begin{Shaded}
\begin{Highlighting}[]
\CommentTok{\# Determine the age group that worked the most hours in the year 1982}
\NormalTok{most\_hours\_1982 }\OtherTok{\textless{}{-}}\NormalTok{ labor }\SpecialCharTok{\%\textgreater{}\%}
  \FunctionTok{filter}\NormalTok{(year }\SpecialCharTok{==} \DecValTok{1982}\NormalTok{) }\SpecialCharTok{\%\textgreater{}\%}
  \FunctionTok{group\_by}\NormalTok{(age) }\SpecialCharTok{\%\textgreater{}\%}
  \FunctionTok{summarize}\NormalTok{(}\AttributeTok{avg\_hours =} \FunctionTok{mean}\NormalTok{(hour, }\AttributeTok{na.rm =} \ConstantTok{TRUE}\NormalTok{)) }\SpecialCharTok{\%\textgreater{}\%}
  \FunctionTok{arrange}\NormalTok{(}\FunctionTok{desc}\NormalTok{(avg\_hours)) }\SpecialCharTok{\%\textgreater{}\%}
  \FunctionTok{slice}\NormalTok{(}\DecValTok{1}\NormalTok{)}

\FunctionTok{print}\NormalTok{(most\_hours\_1982)}
\end{Highlighting}
\end{Shaded}

\begin{verbatim}
## # A tibble: 1 x 2
##     age avg_hours
##   <dbl>     <dbl>
## 1    46     2373.
\end{verbatim}

\begin{enumerate}
\def\labelenumi{\alph{enumi}.}
\setcounter{enumi}{2}
\tightlist
\item
  Create a variable, \texttt{n\_years} that equals the number of years
  an individual stays in the panel. Is the panel balanced?
\end{enumerate}

\begin{Shaded}
\begin{Highlighting}[]
\CommentTok{\# Create the n\_years variable}
\NormalTok{labor }\OtherTok{\textless{}{-}}\NormalTok{ labor }\SpecialCharTok{\%\textgreater{}\%}
  \FunctionTok{group\_by}\NormalTok{(id) }\SpecialCharTok{\%\textgreater{}\%}
  \FunctionTok{mutate}\NormalTok{(}\AttributeTok{n\_years =} \FunctionTok{n\_distinct}\NormalTok{(year)) }\SpecialCharTok{\%\textgreater{}\%}
  \FunctionTok{ungroup}\NormalTok{()}

\CommentTok{\# Check if the panel is balanced}
\NormalTok{panel\_balance }\OtherTok{\textless{}{-}}\NormalTok{ labor }\SpecialCharTok{\%\textgreater{}\%}
  \FunctionTok{summarize}\NormalTok{(}
    \AttributeTok{min\_years =} \FunctionTok{min}\NormalTok{(n\_years),}
    \AttributeTok{max\_years =} \FunctionTok{max}\NormalTok{(n\_years)}
\NormalTok{  )}

\FunctionTok{print}\NormalTok{(panel\_balance)}
\end{Highlighting}
\end{Shaded}

\begin{verbatim}
## # A tibble: 1 x 2
##   min_years max_years
##       <int>     <int>
## 1        10        10
\end{verbatim}

\begin{enumerate}
\def\labelenumi{\alph{enumi}.}
\setcounter{enumi}{3}
\tightlist
\item
  Which are the individuals that do not have any kids during the whole
  period? Create a variable, \texttt{no\_kids}, that flags these
  individuals (1 = no kids, 0 = kids)
\end{enumerate}

\begin{Shaded}
\begin{Highlighting}[]
\CommentTok{\# Identify individuals that do not have any kids during the whole period}
\NormalTok{labor }\OtherTok{\textless{}{-}}\NormalTok{ labor }\SpecialCharTok{\%\textgreater{}\%}
  \FunctionTok{group\_by}\NormalTok{(id) }\SpecialCharTok{\%\textgreater{}\%}
  \FunctionTok{mutate}\NormalTok{(}\AttributeTok{no\_kids =} \FunctionTok{ifelse}\NormalTok{(}\FunctionTok{all}\NormalTok{(kids }\SpecialCharTok{==} \DecValTok{0}\NormalTok{), }\DecValTok{1}\NormalTok{, }\DecValTok{0}\NormalTok{)) }\SpecialCharTok{\%\textgreater{}\%}
  \FunctionTok{ungroup}\NormalTok{()}

\CommentTok{\# Check the \textasciigrave{}no\_kids\textasciigrave{} variable}
\FunctionTok{print}\NormalTok{(labor }\SpecialCharTok{\%\textgreater{}\%} \FunctionTok{select}\NormalTok{(id, no\_kids) }\SpecialCharTok{\%\textgreater{}\%} \FunctionTok{distinct}\NormalTok{())}
\end{Highlighting}
\end{Shaded}

\begin{verbatim}
## # A tibble: 532 x 2
##       id no_kids
##    <dbl>   <dbl>
##  1     1       0
##  2     2       0
##  3     3       1
##  4     4       0
##  5     5       0
##  6     6       0
##  7     7       0
##  8     8       0
##  9     9       0
## 10    10       1
## # i 522 more rows
\end{verbatim}

\begin{enumerate}
\def\labelenumi{\alph{enumi}.}
\setcounter{enumi}{4}
\tightlist
\item
  Using the \texttt{no\_kids} variable from before compute the average
  wage, standard deviation and number of observations in each group for
  the year 1980 (no kids group vs kids group).
\end{enumerate}

\begin{Shaded}
\begin{Highlighting}[]
\CommentTok{\# Compute the average wage, standard deviation, and number of observations for each group in 1980}
\NormalTok{wage\_stats\_1980 }\OtherTok{\textless{}{-}}\NormalTok{ labor }\SpecialCharTok{\%\textgreater{}\%}
  \FunctionTok{filter}\NormalTok{(year }\SpecialCharTok{==} \DecValTok{1980}\NormalTok{) }\SpecialCharTok{\%\textgreater{}\%}
  \FunctionTok{group\_by}\NormalTok{(no\_kids) }\SpecialCharTok{\%\textgreater{}\%}
  \FunctionTok{summarize}\NormalTok{(}
    \AttributeTok{avg\_wage =} \FunctionTok{mean}\NormalTok{(wage, }\AttributeTok{na.rm =} \ConstantTok{TRUE}\NormalTok{),}
    \AttributeTok{sd\_wage =} \FunctionTok{sd}\NormalTok{(wage, }\AttributeTok{na.rm =} \ConstantTok{TRUE}\NormalTok{),}
    \AttributeTok{n\_obs =} \FunctionTok{n}\NormalTok{()}
\NormalTok{  )}

\FunctionTok{print}\NormalTok{(wage\_stats\_1980)}
\end{Highlighting}
\end{Shaded}

\begin{verbatim}
## # A tibble: 2 x 4
##   no_kids avg_wage sd_wage n_obs
##     <dbl>    <dbl>   <dbl> <int>
## 1       0     14.5    6.69   489
## 2       1     15.9    6.71    43
\end{verbatim}

\end{document}
