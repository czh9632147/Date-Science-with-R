% Options for packages loaded elsewhere
\PassOptionsToPackage{unicode}{hyperref}
\PassOptionsToPackage{hyphens}{url}
%
\documentclass[
]{article}
\usepackage{amsmath,amssymb}
\usepackage{iftex}
\ifPDFTeX
  \usepackage[T1]{fontenc}
  \usepackage[utf8]{inputenc}
  \usepackage{textcomp} % provide euro and other symbols
\else % if luatex or xetex
  \usepackage{unicode-math} % this also loads fontspec
  \defaultfontfeatures{Scale=MatchLowercase}
  \defaultfontfeatures[\rmfamily]{Ligatures=TeX,Scale=1}
\fi
\usepackage{lmodern}
\ifPDFTeX\else
  % xetex/luatex font selection
\fi
% Use upquote if available, for straight quotes in verbatim environments
\IfFileExists{upquote.sty}{\usepackage{upquote}}{}
\IfFileExists{microtype.sty}{% use microtype if available
  \usepackage[]{microtype}
  \UseMicrotypeSet[protrusion]{basicmath} % disable protrusion for tt fonts
}{}
\makeatletter
\@ifundefined{KOMAClassName}{% if non-KOMA class
  \IfFileExists{parskip.sty}{%
    \usepackage{parskip}
  }{% else
    \setlength{\parindent}{0pt}
    \setlength{\parskip}{6pt plus 2pt minus 1pt}}
}{% if KOMA class
  \KOMAoptions{parskip=half}}
\makeatother
\usepackage{xcolor}
\usepackage[margin=1in]{geometry}
\usepackage{color}
\usepackage{fancyvrb}
\newcommand{\VerbBar}{|}
\newcommand{\VERB}{\Verb[commandchars=\\\{\}]}
\DefineVerbatimEnvironment{Highlighting}{Verbatim}{commandchars=\\\{\}}
% Add ',fontsize=\small' for more characters per line
\usepackage{framed}
\definecolor{shadecolor}{RGB}{248,248,248}
\newenvironment{Shaded}{\begin{snugshade}}{\end{snugshade}}
\newcommand{\AlertTok}[1]{\textcolor[rgb]{0.94,0.16,0.16}{#1}}
\newcommand{\AnnotationTok}[1]{\textcolor[rgb]{0.56,0.35,0.01}{\textbf{\textit{#1}}}}
\newcommand{\AttributeTok}[1]{\textcolor[rgb]{0.13,0.29,0.53}{#1}}
\newcommand{\BaseNTok}[1]{\textcolor[rgb]{0.00,0.00,0.81}{#1}}
\newcommand{\BuiltInTok}[1]{#1}
\newcommand{\CharTok}[1]{\textcolor[rgb]{0.31,0.60,0.02}{#1}}
\newcommand{\CommentTok}[1]{\textcolor[rgb]{0.56,0.35,0.01}{\textit{#1}}}
\newcommand{\CommentVarTok}[1]{\textcolor[rgb]{0.56,0.35,0.01}{\textbf{\textit{#1}}}}
\newcommand{\ConstantTok}[1]{\textcolor[rgb]{0.56,0.35,0.01}{#1}}
\newcommand{\ControlFlowTok}[1]{\textcolor[rgb]{0.13,0.29,0.53}{\textbf{#1}}}
\newcommand{\DataTypeTok}[1]{\textcolor[rgb]{0.13,0.29,0.53}{#1}}
\newcommand{\DecValTok}[1]{\textcolor[rgb]{0.00,0.00,0.81}{#1}}
\newcommand{\DocumentationTok}[1]{\textcolor[rgb]{0.56,0.35,0.01}{\textbf{\textit{#1}}}}
\newcommand{\ErrorTok}[1]{\textcolor[rgb]{0.64,0.00,0.00}{\textbf{#1}}}
\newcommand{\ExtensionTok}[1]{#1}
\newcommand{\FloatTok}[1]{\textcolor[rgb]{0.00,0.00,0.81}{#1}}
\newcommand{\FunctionTok}[1]{\textcolor[rgb]{0.13,0.29,0.53}{\textbf{#1}}}
\newcommand{\ImportTok}[1]{#1}
\newcommand{\InformationTok}[1]{\textcolor[rgb]{0.56,0.35,0.01}{\textbf{\textit{#1}}}}
\newcommand{\KeywordTok}[1]{\textcolor[rgb]{0.13,0.29,0.53}{\textbf{#1}}}
\newcommand{\NormalTok}[1]{#1}
\newcommand{\OperatorTok}[1]{\textcolor[rgb]{0.81,0.36,0.00}{\textbf{#1}}}
\newcommand{\OtherTok}[1]{\textcolor[rgb]{0.56,0.35,0.01}{#1}}
\newcommand{\PreprocessorTok}[1]{\textcolor[rgb]{0.56,0.35,0.01}{\textit{#1}}}
\newcommand{\RegionMarkerTok}[1]{#1}
\newcommand{\SpecialCharTok}[1]{\textcolor[rgb]{0.81,0.36,0.00}{\textbf{#1}}}
\newcommand{\SpecialStringTok}[1]{\textcolor[rgb]{0.31,0.60,0.02}{#1}}
\newcommand{\StringTok}[1]{\textcolor[rgb]{0.31,0.60,0.02}{#1}}
\newcommand{\VariableTok}[1]{\textcolor[rgb]{0.00,0.00,0.00}{#1}}
\newcommand{\VerbatimStringTok}[1]{\textcolor[rgb]{0.31,0.60,0.02}{#1}}
\newcommand{\WarningTok}[1]{\textcolor[rgb]{0.56,0.35,0.01}{\textbf{\textit{#1}}}}
\usepackage{graphicx}
\makeatletter
\def\maxwidth{\ifdim\Gin@nat@width>\linewidth\linewidth\else\Gin@nat@width\fi}
\def\maxheight{\ifdim\Gin@nat@height>\textheight\textheight\else\Gin@nat@height\fi}
\makeatother
% Scale images if necessary, so that they will not overflow the page
% margins by default, and it is still possible to overwrite the defaults
% using explicit options in \includegraphics[width, height, ...]{}
\setkeys{Gin}{width=\maxwidth,height=\maxheight,keepaspectratio}
% Set default figure placement to htbp
\makeatletter
\def\fps@figure{htbp}
\makeatother
\setlength{\emergencystretch}{3em} % prevent overfull lines
\providecommand{\tightlist}{%
  \setlength{\itemsep}{0pt}\setlength{\parskip}{0pt}}
\setcounter{secnumdepth}{-\maxdimen} % remove section numbering
\ifLuaTeX
  \usepackage{selnolig}  % disable illegal ligatures
\fi
\usepackage{bookmark}
\IfFileExists{xurl.sty}{\usepackage{xurl}}{} % add URL line breaks if available
\urlstyle{same}
\hypersetup{
  pdftitle={Ch3},
  pdfauthor={CZH},
  hidelinks,
  pdfcreator={LaTeX via pandoc}}

\title{Ch3}
\author{CZH}
\date{2024-07-03}

\begin{document}
\maketitle

\subsection{Load data}\label{load-data}

Copy and paste the code chunk below and read it in to your RStudio to
load the five datasets we will use in this section.

\begin{Shaded}
\begin{Highlighting}[]
\FunctionTok{library}\NormalTok{(tidyverse)}
\end{Highlighting}
\end{Shaded}

\begin{verbatim}
## -- Attaching core tidyverse packages ------------------------ tidyverse 2.0.0 --
## v dplyr     1.1.4     v readr     2.1.5
## v forcats   1.0.0     v stringr   1.5.1
## v ggplot2   3.5.1     v tibble    3.2.1
## v lubridate 1.9.3     v tidyr     1.3.1
## v purrr     1.0.2     
## -- Conflicts ------------------------------------------ tidyverse_conflicts() --
## x dplyr::filter() masks stats::filter()
## x dplyr::lag()    masks stats::lag()
## i Use the conflicted package (<http://conflicted.r-lib.org/>) to force all conflicts to become errors
\end{verbatim}

\begin{Shaded}
\begin{Highlighting}[]
 \CommentTok{\#National Parks in California}
\NormalTok{ ca }\OtherTok{\textless{}{-}} \FunctionTok{read\_csv}\NormalTok{(}\StringTok{"data/ca.csv"}\NormalTok{) }
\end{Highlighting}
\end{Shaded}

\begin{verbatim}
## Rows: 789 Columns: 7
## -- Column specification --------------------------------------------------------
## Delimiter: ","
## chr (5): region, state, code, park_name, type
## dbl (2): visitors, year
## 
## i Use `spec()` to retrieve the full column specification for this data.
## i Specify the column types or set `show_col_types = FALSE` to quiet this message.
\end{verbatim}

\begin{Shaded}
\begin{Highlighting}[]
 \CommentTok{\#Acadia National Park}
\NormalTok{ acadia }\OtherTok{\textless{}{-}} \FunctionTok{read\_csv}\NormalTok{(}\StringTok{"data/acadia.csv"}\NormalTok{)}
\end{Highlighting}
\end{Shaded}

\begin{verbatim}
## Rows: 98 Columns: 7
## -- Column specification --------------------------------------------------------
## Delimiter: ","
## chr (5): region, state, code, park_name, type
## dbl (2): visitors, year
## 
## i Use `spec()` to retrieve the full column specification for this data.
## i Specify the column types or set `show_col_types = FALSE` to quiet this message.
\end{verbatim}

\begin{Shaded}
\begin{Highlighting}[]
 \CommentTok{\#Southeast US National Parks}
\NormalTok{ se }\OtherTok{\textless{}{-}} \FunctionTok{read\_csv}\NormalTok{(}\StringTok{"data/se.csv"}\NormalTok{)}
\end{Highlighting}
\end{Shaded}

\begin{verbatim}
## Rows: 453 Columns: 7
## -- Column specification --------------------------------------------------------
## Delimiter: ","
## chr (5): region, state, code, park_name, type
## dbl (2): visitors, year
## 
## i Use `spec()` to retrieve the full column specification for this data.
## i Specify the column types or set `show_col_types = FALSE` to quiet this message.
\end{verbatim}

\begin{Shaded}
\begin{Highlighting}[]
 \CommentTok{\#2016 Visitation for all Pacific West National Parks}
\NormalTok{ visit\_16 }\OtherTok{\textless{}{-}} \FunctionTok{read\_csv}\NormalTok{(}\StringTok{"data/visit\_16.csv"}\NormalTok{)}
\end{Highlighting}
\end{Shaded}

\begin{verbatim}
## Rows: 17 Columns: 7
## -- Column specification --------------------------------------------------------
## Delimiter: ","
## chr (5): region, state, code, park_name, type
## dbl (2): visitors, year
## 
## i Use `spec()` to retrieve the full column specification for this data.
## i Specify the column types or set `show_col_types = FALSE` to quiet this message.
\end{verbatim}

\begin{Shaded}
\begin{Highlighting}[]
 \CommentTok{\#All Nationally designated sites in Massachusetts}
\NormalTok{ mass }\OtherTok{\textless{}{-}} \FunctionTok{read\_csv}\NormalTok{(}\StringTok{"data/mass.csv"}\NormalTok{)}
\end{Highlighting}
\end{Shaded}

\begin{verbatim}
## Rows: 13 Columns: 7
## -- Column specification --------------------------------------------------------
## Delimiter: ","
## chr (5): region, state, code, park_name, type
## dbl (2): visitors, year
## 
## i Use `spec()` to retrieve the full column specification for this data.
## i Specify the column types or set `show_col_types = FALSE` to quiet this message.
\end{verbatim}

\subsection{Plot}\label{plot}

To add a geom to the plot use + operator. Because we have two continuous
variables, let's use geom\_point() first and then assign x and y
aesthetics ( aes ):

\begin{Shaded}
\begin{Highlighting}[]
\FunctionTok{ggplot}\NormalTok{(}\AttributeTok{data =}\NormalTok{ ca) }\SpecialCharTok{+}
\FunctionTok{geom\_point}\NormalTok{(}\FunctionTok{aes}\NormalTok{(}\AttributeTok{x =}\NormalTok{ year, }\AttributeTok{y =}\NormalTok{ visitors))}
\end{Highlighting}
\end{Shaded}

\includegraphics{Ch3_files/figure-latex/unnamed-chunk-2-1.pdf}

\subsection{Building your plots
iteratively}\label{building-your-plots-iteratively}

Building plots with ggplot is typically an iterative process. We start
by defining the dataset we'll use, lay the axes. We can distinguish each
park by added the color argument to the aes:

\begin{Shaded}
\begin{Highlighting}[]
\FunctionTok{ggplot}\NormalTok{(}\AttributeTok{data =}\NormalTok{ ca) }\SpecialCharTok{+}
\FunctionTok{geom\_point}\NormalTok{(}\FunctionTok{aes}\NormalTok{(}\AttributeTok{x =}\NormalTok{ year, }\AttributeTok{y =}\NormalTok{ visitors, }\AttributeTok{color =}\NormalTok{ park\_name))}
\end{Highlighting}
\end{Shaded}

\includegraphics{Ch3_files/figure-latex/unnamed-chunk-3-1.pdf}

\subsection{Customizing plots}\label{customizing-plots}

Take a look at the ggplot2 cheat sheet, and think of ways you could
improve the plot.\\
Now, let's capitalize the x and y axis labels and add a main title to
the figure. I also like to remove that standard gray background using a
different theme. Many themes come built into the ggplot2 package. My
preference is theme\_bw() but once you start typing theme\_ a list of
options will pop up. The last thing I'm going to do is remove the legend
title.

\begin{Shaded}
\begin{Highlighting}[]
\FunctionTok{ggplot}\NormalTok{(}\AttributeTok{data =}\NormalTok{ ca) }\SpecialCharTok{+}
 \FunctionTok{geom\_point}\NormalTok{(}\FunctionTok{aes}\NormalTok{(}\AttributeTok{x =}\NormalTok{ year, }\AttributeTok{y =}\NormalTok{ visitors, }\AttributeTok{color =}\NormalTok{ park\_name)) }\SpecialCharTok{+}
 \FunctionTok{labs}\NormalTok{(}\AttributeTok{x =} \StringTok{"Year"}\NormalTok{,}
 \AttributeTok{y =} \StringTok{"Visitation"}\NormalTok{,}
 \AttributeTok{title =} \StringTok{"California National Park Visitation"}\NormalTok{) }\SpecialCharTok{+}
 \FunctionTok{theme\_bw}\NormalTok{() }\SpecialCharTok{+}
 \FunctionTok{theme}\NormalTok{(}\AttributeTok{legend.title=}\FunctionTok{element\_blank}\NormalTok{())}
\end{Highlighting}
\end{Shaded}

\includegraphics{Ch3_files/figure-latex/unnamed-chunk-4-1.pdf}

\subsection{Faceting}\label{faceting}

ggplot has a special technique called faceting that allows the user to
split one plot into multiple plots based on data in the dataset. We will
use it to make a plot of park visitation by state:\\
You can also embed plots, for example:

\begin{Shaded}
\begin{Highlighting}[]
\FunctionTok{ggplot}\NormalTok{(}\AttributeTok{data =}\NormalTok{ se) }\SpecialCharTok{+}
\FunctionTok{geom\_point}\NormalTok{(}\FunctionTok{aes}\NormalTok{(}\AttributeTok{x =}\NormalTok{ year, }\AttributeTok{y =}\NormalTok{ visitors)) }\SpecialCharTok{+}
\FunctionTok{facet\_wrap}\NormalTok{(}\SpecialCharTok{\textasciitilde{}}\NormalTok{ state)}
\end{Highlighting}
\end{Shaded}

\includegraphics{Ch3_files/figure-latex/unnamed-chunk-5-1.pdf} We can
now make the faceted plot by splitting further by park using park\_name
(within a single plot):

\begin{Shaded}
\begin{Highlighting}[]
\FunctionTok{ggplot}\NormalTok{(}\AttributeTok{data =}\NormalTok{ se) }\SpecialCharTok{+}
\FunctionTok{geom\_point}\NormalTok{(}\FunctionTok{aes}\NormalTok{(}\AttributeTok{x =}\NormalTok{ year, }\AttributeTok{y =}\NormalTok{ visitors, }\AttributeTok{color =}\NormalTok{ park\_name)) }\SpecialCharTok{+}
\FunctionTok{facet\_wrap}\NormalTok{(}\SpecialCharTok{\textasciitilde{}}\NormalTok{ state, }\AttributeTok{scales =} \StringTok{"free"}\NormalTok{)}
\end{Highlighting}
\end{Shaded}

\includegraphics{Ch3_files/figure-latex/unnamed-chunk-6-1.pdf} \#\#
Geometric objects (geoms) A geom is the geometrical object that a plot
uses to represent data. People often describe plots by the type of geom
that the plot uses. For example, bar charts use bar geoms, line charts
use line geoms, boxplots use boxplot geoms, and so on. Scatterplots
break the trend; they use the point geom. You can use different geoms to
plot the same data. To change the geom in your plot, change the geom
function that you add to ggplot() . Let's look at a few ways of viewing
the distribution of annual visitation ( visitors ) for each park (
park\_name ).

\begin{Shaded}
\begin{Highlighting}[]
\FunctionTok{ggplot}\NormalTok{(}\AttributeTok{data =}\NormalTok{ se) }\SpecialCharTok{+} 
\FunctionTok{geom\_jitter}\NormalTok{(}\FunctionTok{aes}\NormalTok{(}\AttributeTok{x =}\NormalTok{ park\_name, }\AttributeTok{y =}\NormalTok{ visitors, }\AttributeTok{color =}\NormalTok{ park\_name), }
\AttributeTok{width =} \FloatTok{0.1}\NormalTok{, }
\AttributeTok{alpha =} \FloatTok{0.4}\NormalTok{) }\SpecialCharTok{+}
 \FunctionTok{coord\_flip}\NormalTok{() }\SpecialCharTok{+}
 \FunctionTok{theme}\NormalTok{(}\AttributeTok{legend.position =} \StringTok{"none"}\NormalTok{)}
\end{Highlighting}
\end{Shaded}

\includegraphics{Ch3_files/figure-latex/unnamed-chunk-7-1.pdf}

\begin{Shaded}
\begin{Highlighting}[]
\FunctionTok{ggplot}\NormalTok{(se, }\FunctionTok{aes}\NormalTok{(}\AttributeTok{x =}\NormalTok{ park\_name, }\AttributeTok{y =}\NormalTok{ visitors)) }\SpecialCharTok{+}
\FunctionTok{geom\_boxplot}\NormalTok{() }\SpecialCharTok{+}
\FunctionTok{theme}\NormalTok{(}\AttributeTok{axis.text.x =} \FunctionTok{element\_text}\NormalTok{(}\AttributeTok{angle =} \DecValTok{45}\NormalTok{, }\AttributeTok{hjust =} \DecValTok{1}\NormalTok{))}
\end{Highlighting}
\end{Shaded}

\includegraphics{Ch3_files/figure-latex/unnamed-chunk-8-1.pdf} None of
these are great for visualizing data over time. We can use geom\_line()
in the same way we used geom\_point .

\begin{Shaded}
\begin{Highlighting}[]
 \FunctionTok{ggplot}\NormalTok{(se, }\FunctionTok{aes}\NormalTok{(}\AttributeTok{x =}\NormalTok{ year, }\AttributeTok{y =}\NormalTok{ visitors, }\AttributeTok{color =}\NormalTok{ park\_name)) }\SpecialCharTok{+}
 \FunctionTok{geom\_line}\NormalTok{()}
\end{Highlighting}
\end{Shaded}

\includegraphics{Ch3_files/figure-latex/unnamed-chunk-9-1.pdf}

To display multiple geoms in the same plot, add multiple geom functions
to ggplot() :\\
geom\_smooth allows you to view a smoothed mean of data. Here we look at
the smooth mean of visitation over time to Acadia National Park:

\begin{Shaded}
\begin{Highlighting}[]
\FunctionTok{ggplot}\NormalTok{(}\AttributeTok{data =}\NormalTok{ acadia) }\SpecialCharTok{+}
\FunctionTok{geom\_point}\NormalTok{(}\FunctionTok{aes}\NormalTok{(}\AttributeTok{x =}\NormalTok{ year, }\AttributeTok{y =}\NormalTok{ visitors)) }\SpecialCharTok{+}
\FunctionTok{geom\_line}\NormalTok{(}\FunctionTok{aes}\NormalTok{(}\AttributeTok{x =}\NormalTok{ year, }\AttributeTok{y =}\NormalTok{ visitors)) }\SpecialCharTok{+}
\FunctionTok{geom\_smooth}\NormalTok{(}\FunctionTok{aes}\NormalTok{(}\AttributeTok{x =}\NormalTok{ year, }\AttributeTok{y =}\NormalTok{ visitors)) }\SpecialCharTok{+}
\FunctionTok{labs}\NormalTok{(}\AttributeTok{title =} \StringTok{"Acadia National Park Visitation"}\NormalTok{,}
\AttributeTok{y =} \StringTok{"Visitation"}\NormalTok{,}
\AttributeTok{x =} \StringTok{"Year"}\NormalTok{) }\SpecialCharTok{+}
\FunctionTok{theme\_bw}\NormalTok{()}
\end{Highlighting}
\end{Shaded}

\begin{verbatim}
## `geom_smooth()` using method = 'loess' and formula = 'y ~ x'
\end{verbatim}

\includegraphics{Ch3_files/figure-latex/unnamed-chunk-10-1.pdf} \#\# Bar
charts Next, let's take a look at a bar chart. Bar charts seem simple,
but they are interesting because they reveal something subtle about
plots. Consider a basic bar chart, as drawn with geom\_bar(). The
following chart displays the total number of parks in each state within
the Pacific West region.

\begin{Shaded}
\begin{Highlighting}[]
\FunctionTok{ggplot}\NormalTok{(}\AttributeTok{data =}\NormalTok{ visit\_16, }\FunctionTok{aes}\NormalTok{(}\AttributeTok{x =}\NormalTok{ state)) }\SpecialCharTok{+}
\FunctionTok{geom\_bar}\NormalTok{()}
\end{Highlighting}
\end{Shaded}

\includegraphics{Ch3_files/figure-latex/unnamed-chunk-11-1.pdf}

\subsection{Position adjustments}\label{position-adjustments}

There's one more piece of magic associated with bar charts. You can
colour a bar chart using either the color aesthetic, or, more usefully,
fill :

\begin{Shaded}
\begin{Highlighting}[]
\FunctionTok{ggplot}\NormalTok{(}\AttributeTok{data =}\NormalTok{ visit\_16, }\FunctionTok{aes}\NormalTok{(}\AttributeTok{x =}\NormalTok{ state, }\AttributeTok{y =}\NormalTok{ visitors, }\AttributeTok{fill =}\NormalTok{ park\_name)) }\SpecialCharTok{+}
\FunctionTok{geom\_bar}\NormalTok{(}\AttributeTok{stat =} \StringTok{"identity"}\NormalTok{)}
\end{Highlighting}
\end{Shaded}

\includegraphics{Ch3_files/figure-latex/unnamed-chunk-12-1.pdf}

The stacking is performed automatically by the position adjustment
specified by the position argument. If you don't want a stacked bar
chart, you can use dodge .\\
-position = ``dodge'' places overlapping objects directly beside one
another. This makes it easier to compare individual values.

\begin{Shaded}
\begin{Highlighting}[]
\FunctionTok{ggplot}\NormalTok{(}\AttributeTok{data =}\NormalTok{ visit\_16, }\FunctionTok{aes}\NormalTok{(}\AttributeTok{x =}\NormalTok{ state, }\AttributeTok{y =}\NormalTok{ visitors, }\AttributeTok{fill =}\NormalTok{ park\_name)) }\SpecialCharTok{+} 
  \FunctionTok{geom\_bar}\NormalTok{(}\AttributeTok{stat =} \StringTok{"identity"}\NormalTok{, }\AttributeTok{position =} \StringTok{"dodge"}\NormalTok{)}
\end{Highlighting}
\end{Shaded}

\includegraphics{Ch3_files/figure-latex/unnamed-chunk-13-1.pdf}

Note that the \texttt{echo\ =\ FALSE} parameter was added to the code
chunk to prevent printing of the R code that generated the plot.

\end{document}
