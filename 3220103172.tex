% Options for packages loaded elsewhere
\PassOptionsToPackage{unicode}{hyperref}
\PassOptionsToPackage{hyphens}{url}
%
\documentclass[
]{article}
\usepackage{amsmath,amssymb}
\usepackage{iftex}
\ifPDFTeX
  \usepackage[T1]{fontenc}
  \usepackage[utf8]{inputenc}
  \usepackage{textcomp} % provide euro and other symbols
\else % if luatex or xetex
  \usepackage{unicode-math} % this also loads fontspec
  \defaultfontfeatures{Scale=MatchLowercase}
  \defaultfontfeatures[\rmfamily]{Ligatures=TeX,Scale=1}
\fi
\usepackage{lmodern}
\ifPDFTeX\else
  % xetex/luatex font selection
\fi
% Use upquote if available, for straight quotes in verbatim environments
\IfFileExists{upquote.sty}{\usepackage{upquote}}{}
\IfFileExists{microtype.sty}{% use microtype if available
  \usepackage[]{microtype}
  \UseMicrotypeSet[protrusion]{basicmath} % disable protrusion for tt fonts
}{}
\makeatletter
\@ifundefined{KOMAClassName}{% if non-KOMA class
  \IfFileExists{parskip.sty}{%
    \usepackage{parskip}
  }{% else
    \setlength{\parindent}{0pt}
    \setlength{\parskip}{6pt plus 2pt minus 1pt}}
}{% if KOMA class
  \KOMAoptions{parskip=half}}
\makeatother
\usepackage{xcolor}
\usepackage[margin=1in]{geometry}
\usepackage{color}
\usepackage{fancyvrb}
\newcommand{\VerbBar}{|}
\newcommand{\VERB}{\Verb[commandchars=\\\{\}]}
\DefineVerbatimEnvironment{Highlighting}{Verbatim}{commandchars=\\\{\}}
% Add ',fontsize=\small' for more characters per line
\usepackage{framed}
\definecolor{shadecolor}{RGB}{248,248,248}
\newenvironment{Shaded}{\begin{snugshade}}{\end{snugshade}}
\newcommand{\AlertTok}[1]{\textcolor[rgb]{0.94,0.16,0.16}{#1}}
\newcommand{\AnnotationTok}[1]{\textcolor[rgb]{0.56,0.35,0.01}{\textbf{\textit{#1}}}}
\newcommand{\AttributeTok}[1]{\textcolor[rgb]{0.13,0.29,0.53}{#1}}
\newcommand{\BaseNTok}[1]{\textcolor[rgb]{0.00,0.00,0.81}{#1}}
\newcommand{\BuiltInTok}[1]{#1}
\newcommand{\CharTok}[1]{\textcolor[rgb]{0.31,0.60,0.02}{#1}}
\newcommand{\CommentTok}[1]{\textcolor[rgb]{0.56,0.35,0.01}{\textit{#1}}}
\newcommand{\CommentVarTok}[1]{\textcolor[rgb]{0.56,0.35,0.01}{\textbf{\textit{#1}}}}
\newcommand{\ConstantTok}[1]{\textcolor[rgb]{0.56,0.35,0.01}{#1}}
\newcommand{\ControlFlowTok}[1]{\textcolor[rgb]{0.13,0.29,0.53}{\textbf{#1}}}
\newcommand{\DataTypeTok}[1]{\textcolor[rgb]{0.13,0.29,0.53}{#1}}
\newcommand{\DecValTok}[1]{\textcolor[rgb]{0.00,0.00,0.81}{#1}}
\newcommand{\DocumentationTok}[1]{\textcolor[rgb]{0.56,0.35,0.01}{\textbf{\textit{#1}}}}
\newcommand{\ErrorTok}[1]{\textcolor[rgb]{0.64,0.00,0.00}{\textbf{#1}}}
\newcommand{\ExtensionTok}[1]{#1}
\newcommand{\FloatTok}[1]{\textcolor[rgb]{0.00,0.00,0.81}{#1}}
\newcommand{\FunctionTok}[1]{\textcolor[rgb]{0.13,0.29,0.53}{\textbf{#1}}}
\newcommand{\ImportTok}[1]{#1}
\newcommand{\InformationTok}[1]{\textcolor[rgb]{0.56,0.35,0.01}{\textbf{\textit{#1}}}}
\newcommand{\KeywordTok}[1]{\textcolor[rgb]{0.13,0.29,0.53}{\textbf{#1}}}
\newcommand{\NormalTok}[1]{#1}
\newcommand{\OperatorTok}[1]{\textcolor[rgb]{0.81,0.36,0.00}{\textbf{#1}}}
\newcommand{\OtherTok}[1]{\textcolor[rgb]{0.56,0.35,0.01}{#1}}
\newcommand{\PreprocessorTok}[1]{\textcolor[rgb]{0.56,0.35,0.01}{\textit{#1}}}
\newcommand{\RegionMarkerTok}[1]{#1}
\newcommand{\SpecialCharTok}[1]{\textcolor[rgb]{0.81,0.36,0.00}{\textbf{#1}}}
\newcommand{\SpecialStringTok}[1]{\textcolor[rgb]{0.31,0.60,0.02}{#1}}
\newcommand{\StringTok}[1]{\textcolor[rgb]{0.31,0.60,0.02}{#1}}
\newcommand{\VariableTok}[1]{\textcolor[rgb]{0.00,0.00,0.00}{#1}}
\newcommand{\VerbatimStringTok}[1]{\textcolor[rgb]{0.31,0.60,0.02}{#1}}
\newcommand{\WarningTok}[1]{\textcolor[rgb]{0.56,0.35,0.01}{\textbf{\textit{#1}}}}
\usepackage{graphicx}
\makeatletter
\def\maxwidth{\ifdim\Gin@nat@width>\linewidth\linewidth\else\Gin@nat@width\fi}
\def\maxheight{\ifdim\Gin@nat@height>\textheight\textheight\else\Gin@nat@height\fi}
\makeatother
% Scale images if necessary, so that they will not overflow the page
% margins by default, and it is still possible to overwrite the defaults
% using explicit options in \includegraphics[width, height, ...]{}
\setkeys{Gin}{width=\maxwidth,height=\maxheight,keepaspectratio}
% Set default figure placement to htbp
\makeatletter
\def\fps@figure{htbp}
\makeatother
\setlength{\emergencystretch}{3em} % prevent overfull lines
\providecommand{\tightlist}{%
  \setlength{\itemsep}{0pt}\setlength{\parskip}{0pt}}
\setcounter{secnumdepth}{-\maxdimen} % remove section numbering
\ifLuaTeX
  \usepackage{selnolig}  % disable illegal ligatures
\fi
\usepackage{bookmark}
\IfFileExists{xurl.sty}{\usepackage{xurl}}{} % add URL line breaks if available
\urlstyle{same}
\hypersetup{
  pdftitle={3220103172},
  pdfauthor={Zhehao Chen},
  hidelinks,
  pdfcreator={LaTeX via pandoc}}

\title{3220103172}
\author{Zhehao Chen}
\date{2024-07-08}

\begin{document}
\maketitle

\subsubsection{1.}\label{section}

\subparagraph{(a) (5,8,1,5,8,1,\ldots,5,8,1) where there are 10
occurrences of
5.}\label{a-581581581-where-there-are-10-occurrences-of-5.}

To create a vector that repeats the sequence (5, 8, 1) so that there are
10 occurrences of 5, you can use the rep function:

\begin{Shaded}
\begin{Highlighting}[]
\NormalTok{tmp }\OtherTok{\textless{}{-}} \FunctionTok{c}\NormalTok{(}\DecValTok{5}\NormalTok{, }\DecValTok{8}\NormalTok{, }\DecValTok{1}\NormalTok{)}
\NormalTok{vec\_a }\OtherTok{\textless{}{-}} \FunctionTok{rep}\NormalTok{(tmp, }\AttributeTok{times =} \DecValTok{10}\NormalTok{)}
\FunctionTok{print}\NormalTok{(vec\_a)}
\end{Highlighting}
\end{Shaded}

\begin{verbatim}
##  [1] 5 8 1 5 8 1 5 8 1 5 8 1 5 8 1 5 8 1 5 8 1 5 8 1 5 8 1 5 8 1
\end{verbatim}

\subparagraph{(b) (5,5,\ldots,5,8,8,\ldots,8,1,1,\ldots,1) where there
are 30 occurrences of 5, 20 occurrences of 8 and 10 occurrences of
1.}\label{b-555888111-where-there-are-30-occurrences-of-5-20-occurrences-of-8-and-10-occurrences-of-1.}

To create a vector with 30 occurrences of 5, 20 occurrences of 8, and 10
occurrences of 1, you can use the rep function with the times argument
as a vector:

\begin{Shaded}
\begin{Highlighting}[]
\NormalTok{tmp }\OtherTok{\textless{}{-}} \FunctionTok{c}\NormalTok{(}\DecValTok{5}\NormalTok{, }\DecValTok{8}\NormalTok{, }\DecValTok{1}\NormalTok{)}
\NormalTok{times }\OtherTok{\textless{}{-}} \FunctionTok{c}\NormalTok{(}\DecValTok{30}\NormalTok{, }\DecValTok{20}\NormalTok{, }\DecValTok{10}\NormalTok{)}
\NormalTok{vec\_b }\OtherTok{\textless{}{-}} \FunctionTok{rep}\NormalTok{(tmp, }\AttributeTok{times =}\NormalTok{ times)}
\FunctionTok{print}\NormalTok{(vec\_b)}
\end{Highlighting}
\end{Shaded}

\begin{verbatim}
##  [1] 5 5 5 5 5 5 5 5 5 5 5 5 5 5 5 5 5 5 5 5 5 5 5 5 5 5 5 5 5 5 8 8 8 8 8 8 8 8
## [39] 8 8 8 8 8 8 8 8 8 8 8 8 1 1 1 1 1 1 1 1 1 1
\end{verbatim}

\subsubsection{2.}\label{section-1}

Execute the following lines which create two vectors of random integers
which are chosen with replacement from the integers 0,1,\ldots,999. Both
vectors have length 250.

\begin{Shaded}
\begin{Highlighting}[]
\NormalTok{ xVec }\OtherTok{\textless{}{-}} \FunctionTok{sample}\NormalTok{(}\DecValTok{0}\SpecialCharTok{:}\DecValTok{999}\NormalTok{, }\DecValTok{250}\NormalTok{, }\AttributeTok{replace=}\NormalTok{T)}
\NormalTok{ yVec }\OtherTok{\textless{}{-}} \FunctionTok{sample}\NormalTok{(}\DecValTok{0}\SpecialCharTok{:}\DecValTok{999}\NormalTok{, }\DecValTok{250}\NormalTok{, }\AttributeTok{replace=}\NormalTok{T)}
\NormalTok{ xVec}
\end{Highlighting}
\end{Shaded}

\begin{verbatim}
##   [1] 417 284 993 181 633   6 401 305 889  11 768 426 387 777  30 839 519 221
##  [19] 121 463  56 973 461  97 494 377 921 601 945 215 619 569  50 316  12 741
##  [37] 275 147 964 236 338 493 101 390 212 278 259 580 674 612 129 146 545 408
##  [55] 813 365 418 857 162 740 280   7 664 139  66 791 903 872  15 581 865 891
##  [73] 526 718 854 132 334 812 549 309 410  97 595 739 986 109 371 548 844 628
##  [91] 162 662 739 425 310 191 230 589 334 264 566 826 689 687 219 654 280 710
## [109]  44 818 667 744 671 996 402   9 952 135 159 678 460 982  29 582 354  17
## [127] 655 234 892  66 700 808 238 489 537   8 597 395 334 511 795  18 860 655
## [145] 865 103  97 921  79  90 344 571 467 470 227 560  65 118 618 834 828 652
## [163] 395 242 154 957 754 914 398 909 675  55 489 624 360 122 130 193 795 668
## [181]  29 311 674 922 823 987 688 204  42 756  18  53 767 509 997 450 961 743
## [199] 993 659 489 361 938 226 869 703  18 825  24 869 836 489 664 769 980 799
## [217] 577 699 113 296 613 207 189 730 125 465 660 372 127 766 566 974 738 971
## [235] 966 401 111 381  41 861 314 435 789 170  35 666 313 204 912 241
\end{verbatim}

\begin{Shaded}
\begin{Highlighting}[]
\NormalTok{ yVec}
\end{Highlighting}
\end{Shaded}

\begin{verbatim}
##   [1] 753  24 917 328 118 248 301 437 463 646 996  40 674 461 903 582 632 648
##  [19] 732 750 894 672 488 138 105 212 197  73  19 886  39 971 172 738 984  15
##  [37] 697 201  24 887 433 554 215 370 990 904 222 551 733 342 625 278 999 853
##  [55] 413 849 220 342 928 376 191 567   1 745   5 450 734 440 367 122 467 627
##  [73] 660 942 675 571 179 952 404  10  18 916 776 570 848 471 141 561 421 785
##  [91] 528 405 666 787 242 597 462 666 376 622 827 593 559 272 183 900 237 971
## [109] 697 386 249 956 956 523 505 678 709 149 886 390 122 754 195 208 769 814
## [127] 442 119 539 959 266 482 260 612 394  11 323 678 758 470 176 347  66 578
## [145] 592 565 941 991 873 586 940 243 635 486 959 813 979 888 982 257 867 572
## [163] 562  74 783 950  30 876 571 311  22 918  18 878 253 356 671 406 560 980
## [181] 509 511 760 863 833 595 704  84 407 655 926 503 506 357 616 250 324 514
## [199] 656  42 720 589 162 172 670 775 926 184 749 806 103 591 434 340 202 188
## [217] 290 195 774 443 898 938 915 807 217 935 868 103 637 773 117 292 832 203
## [235]   4 858 626 666  53 991 416 712 812 869 247 282 776 490  55 792
\end{verbatim}

\subparagraph{(a) Create the vector (y2 − x1,··· ,yn −
xn−1).}\label{a-create-the-vector-y2-x1-yn-xn1.}

yVec{[}2:250{]} - xVec{[}1:249{]} creates a vector by subtracting each
element in xVec (from the 1st to the 249th) from the corresponding next
element in yVec (from the 2nd to the 250th).

\begin{Shaded}
\begin{Highlighting}[]
\NormalTok{aVec }\OtherTok{\textless{}{-}}\NormalTok{ yVec[}\DecValTok{2}\SpecialCharTok{:}\DecValTok{250}\NormalTok{] }\SpecialCharTok{{-}}\NormalTok{ xVec[}\DecValTok{1}\SpecialCharTok{:}\DecValTok{249}\NormalTok{]}
\FunctionTok{print}\NormalTok{(aVec)}
\end{Highlighting}
\end{Shaded}

\begin{verbatim}
##   [1] -393  633 -665  -63 -385  295   36  158 -243  985 -728  248   74  126  552
##  [16] -207  129  511  629  431  616 -485 -323    8 -282 -180 -848 -582  -59 -176
##  [31]  352 -397  688  668    3  -44  -74 -123  -77  197  216 -278  269  600  692
##  [46]  -56  292  153 -332   13  149  853  308    5   36 -145  -76   71  214 -549
##  [61]  287   -6   81 -134  384  -57 -463 -505  107 -114 -238 -231  416  -43 -283
##  [76]   47  618 -408 -539 -291  506  679  -25  109 -515   32  190 -127  -59 -100
##  [91]  243    4   48 -183  287  271  436 -213  288  563   27 -267 -417 -504  681
## [106] -417  691  -13  342 -569  289  212 -148 -491  276  700 -803  751  231 -556
## [121]  294 -787  179  187  460  425 -536  305   67  200 -218 -548  374  -95 -526
## [136]  315   81  363  136 -335 -448   48 -282  -63 -300  838  894  -48  507  850
## [151] -101   64   19  489  586  419  823  864 -361   33 -256  -90 -321  541  796
## [166] -927  122 -343  -87 -887  243  -37  389 -371   -4  549  276  367  185 -159
## [181]  482  449  189  -89 -228 -283 -604  203  613  170  485  453 -410  107 -747
## [196] -126 -447  -87 -951   61  100 -199 -766  444  -94  223  166  -76  782 -766
## [211] -245  -55 -324 -567 -792 -509 -382   75  330  602  325  708  618 -513  810
## [226]  403 -557  265  646 -649 -274 -142 -535 -967 -108  225  555 -328  950 -445
## [241]  398  377   80   77  247  110  177 -149 -120
\end{verbatim}

\subparagraph{(b) Pick out the values in yVec which are \textgreater{}
600.}\label{b-pick-out-the-values-in-yvec-which-are-600.}

yVec{[}yVec \textgreater{} 600{]} selects the values from yVec that are
greater than 600.

\begin{Shaded}
\begin{Highlighting}[]
\NormalTok{bVec }\OtherTok{\textless{}{-}}\NormalTok{ yVec[yVec }\SpecialCharTok{\textgreater{}} \DecValTok{600}\NormalTok{]}
\FunctionTok{print}\NormalTok{(bVec)}
\end{Highlighting}
\end{Shaded}

\begin{verbatim}
##   [1] 753 917 646 996 674 903 632 648 732 750 894 672 886 971 738 984 697 887
##  [19] 990 904 733 625 999 853 849 928 745 734 627 660 942 675 952 916 776 848
##  [37] 785 666 787 666 622 827 900 971 697 956 956 678 709 886 754 769 814 959
##  [55] 612 678 758 941 991 873 940 635 959 813 979 888 982 867 783 950 876 918
##  [73] 878 671 980 760 863 833 704 655 926 616 656 720 670 775 926 749 806 774
##  [91] 898 938 915 807 935 868 637 773 832 858 626 666 991 712 812 869 776 792
\end{verbatim}

\subparagraph{(c) What are the index positions in yVec of the values
which are \textgreater{}
600?}\label{c-what-are-the-index-positions-in-yvec-of-the-values-which-are-600}

which(yVec \textgreater{} 600) returns the index positions of elements
in yVec that are greater than 600.

\begin{Shaded}
\begin{Highlighting}[]
\NormalTok{cVec }\OtherTok{\textless{}{-}} \FunctionTok{which}\NormalTok{(yVec }\SpecialCharTok{\textgreater{}} \DecValTok{600}\NormalTok{)}
\FunctionTok{print}\NormalTok{(cVec)}
\end{Highlighting}
\end{Shaded}

\begin{verbatim}
##   [1]   1   3  10  11  13  15  17  18  19  20  21  22  30  32  34  35  37  40
##  [19]  45  46  49  51  53  54  56  59  64  67  72  73  74  75  78  82  83  85
##  [37]  90  93  94  98 100 101 106 108 109 112 113 116 117 119 122 125 126 130
##  [55] 134 138 139 147 148 149 151 153 155 156 157 158 159 161 165 166 168 172
##  [73] 174 177 180 183 184 185 187 190 191 195 199 201 205 206 207 209 210 219
##  [91] 221 222 223 224 226 227 229 230 233 236 237 238 240 242 243 244 247 250
\end{verbatim}

\subparagraph{(d) Sort the numbers in the vector xVec in the order of
increasing values in
yVec.}\label{d-sort-the-numbers-in-the-vector-xvec-in-the-order-of-increasing-values-in-yvec.}

xVec{[}order(yVec){]} sorts xVec based on the order of the corresponding
elements in yVec.

\begin{Shaded}
\begin{Highlighting}[]
\NormalTok{dVec }\OtherTok{\textless{}{-}}\NormalTok{ xVec[}\FunctionTok{order}\NormalTok{(yVec)]}
\FunctionTok{print}\NormalTok{(dVec)}
\end{Highlighting}
\end{Shaded}

\begin{verbatim}
##   [1] 664 966  66 309   8 741 410 489 945 675 284 964 754 619 426 659  41 912
##  [19] 860 601 242 204 836 372 494 566 633 234 581 460  97 371 135 938  50 226
##  [37] 795 334 219 825 799 280  29 699 921 147 980 971 582 377 101 125 418 259
##  [55] 280 310 571  35   6 667 450 360 834 238 700 687 146 666 577 974 401 909
##  [73] 597 961 181 769 612 857  18 122 509  15 390 740 334 818 678 537 549 662
##  [91] 193  42 813 314 844 338 664 305 872 655 296 791 777 230 889 865 511 109
## [109] 808 470 461 204  53 402 767  29 311 743 996 162 892 580 493 689 795 548
## [127] 395 103   7 739 132 398 652 655 839  90 361 489 865 826 987 191 489 997
## [145] 264 129 111 891 519 467 127  11 221 756 993 526 739 589 381 869 130 973
## [163] 387 854   9 395 275  44 688 952 435 489 121 674 903 316 139  24 463 417
## [181] 982 334 674 354 766 113 703 595 313 154 628 425 241 869 730 789 560  17
## [199] 566 738 823 986 365 408 401 922 828 660 170  79 914 624 215 159 236 118
## [217]  56 613 654  30 278 189  97 993  55  18  18 162 465 207 344  97 718 957
## [235] 812 744 671  66 227 569 710  65 668 618  12 212 921 861 768 545
\end{verbatim}

\subparagraph{(e) Pick out the elements in yVec at index positions
1,4,7,10,13,···}\label{e-pick-out-the-elements-in-yvec-at-index-positions-1471013}

yVec{[}seq(1, length(yVec), by = 3){]} picks out every third element in
yVec starting from the first position.

\begin{Shaded}
\begin{Highlighting}[]
\NormalTok{eVec }\OtherTok{\textless{}{-}}\NormalTok{ yVec[}\FunctionTok{seq}\NormalTok{(}\DecValTok{1}\NormalTok{, }\FunctionTok{length}\NormalTok{(yVec), }\AttributeTok{by =} \DecValTok{3}\NormalTok{)]}
\FunctionTok{print}\NormalTok{(eVec)}
\end{Highlighting}
\end{Shaded}

\begin{verbatim}
##  [1] 753 328 301 646 674 582 732 672 105  73  39 738 697 887 215 904 733 278 413
## [20] 342 191 745 734 122 660 571 404 916 848 561 528 787 462 622 559 900 697 956
## [39] 505 149 122 208 442 959 260  11 758 347 592 991 940 486 979 257 562 950 571
## [58] 918 253 406 509 863 704 655 506 250 656 589 670 184 103 340 290 443 915 935
## [77] 637 292   4 666 416 869 776 792
\end{verbatim}

\subsubsection{3.}\label{section-2}

\begin{Shaded}
\begin{Highlighting}[]
\NormalTok{X }\OtherTok{\textless{}{-}} \FunctionTok{c}\NormalTok{(}\DecValTok{34}\NormalTok{, }\DecValTok{33}\NormalTok{, }\DecValTok{65}\NormalTok{, }\DecValTok{37}\NormalTok{, }\DecValTok{89}\NormalTok{, }\ConstantTok{NA}\NormalTok{, }\DecValTok{43}\NormalTok{, }\ConstantTok{NA}\NormalTok{, }\DecValTok{11}\NormalTok{, }\ConstantTok{NA}\NormalTok{, }\DecValTok{23}\NormalTok{, }\ConstantTok{NA}\NormalTok{)}
\NormalTok{X}
\end{Highlighting}
\end{Shaded}

\begin{verbatim}
##  [1] 34 33 65 37 89 NA 43 NA 11 NA 23 NA
\end{verbatim}

\subparagraph{Write a piece of R code to count the number of occurrences
of NA in
X?}\label{write-a-piece-of-r-code-to-count-the-number-of-occurrences-of-na-in-x}

is.na(X) returns a logical vector of the same length as X with TRUE for
each NA and FALSE otherwise.\\
sum(is.na(X)) sums the TRUE values, effectively counting the number of
NA values in X.

\begin{Shaded}
\begin{Highlighting}[]
\NormalTok{na\_count }\OtherTok{\textless{}{-}} \FunctionTok{sum}\NormalTok{(}\FunctionTok{is.na}\NormalTok{(X))}
\FunctionTok{print}\NormalTok{(na\_count)}
\end{Highlighting}
\end{Shaded}

\begin{verbatim}
## [1] 4
\end{verbatim}

\subsubsection{4.}\label{section-3}

For this problem we'll use the (built-in) dataset state.x77.

\begin{Shaded}
\begin{Highlighting}[]
\FunctionTok{library}\NormalTok{(tidyverse)}
\end{Highlighting}
\end{Shaded}

\begin{verbatim}
## -- Attaching core tidyverse packages ------------------------ tidyverse 2.0.0 --
## v dplyr     1.1.4     v readr     2.1.5
## v forcats   1.0.0     v stringr   1.5.1
## v ggplot2   3.5.1     v tibble    3.2.1
## v lubridate 1.9.3     v tidyr     1.3.1
## v purrr     1.0.2     
## -- Conflicts ------------------------------------------ tidyverse_conflicts() --
## x dplyr::filter() masks stats::filter()
## x dplyr::lag()    masks stats::lag()
## i Use the conflicted package (<http://conflicted.r-lib.org/>) to force all conflicts to become errors
\end{verbatim}

\begin{Shaded}
\begin{Highlighting}[]
\FunctionTok{data}\NormalTok{(state)}
\NormalTok{state.x77 }\OtherTok{\textless{}{-}} \FunctionTok{as\_tibble}\NormalTok{(state.x77, }\AttributeTok{rownames =} \StringTok{\textquotesingle{}State\textquotesingle{}}\NormalTok{)}
\end{Highlighting}
\end{Shaded}

\subparagraph{(a) Select all the states having an income less than 4300,
and calculate the average income of these
states.}\label{a-select-all-the-states-having-an-income-less-than-4300-and-calculate-the-average-income-of-these-states.}

\begin{Shaded}
\begin{Highlighting}[]
\CommentTok{\# Select states with income less than 4300}
\NormalTok{low\_income\_states }\OtherTok{\textless{}{-}}\NormalTok{ state.x77 }\SpecialCharTok{\%\textgreater{}\%} \FunctionTok{filter}\NormalTok{(Income }\SpecialCharTok{\textless{}} \DecValTok{4300}\NormalTok{)}

\CommentTok{\# Calculate the average income of these states}
\NormalTok{average\_low\_income }\OtherTok{\textless{}{-}} \FunctionTok{mean}\NormalTok{(low\_income\_states}\SpecialCharTok{$}\NormalTok{Income)}
\NormalTok{average\_low\_income}
\end{Highlighting}
\end{Shaded}

\begin{verbatim}
## [1] 3830.6
\end{verbatim}

\subparagraph{(b) Sort the data by income and select the state with the
highest
income.}\label{b-sort-the-data-by-income-and-select-the-state-with-the-highest-income.}

\begin{Shaded}
\begin{Highlighting}[]
\CommentTok{\# Sort the data by income}
\NormalTok{sorted\_states }\OtherTok{\textless{}{-}}\NormalTok{ state.x77 }\SpecialCharTok{\%\textgreater{}\%} \FunctionTok{arrange}\NormalTok{(}\FunctionTok{desc}\NormalTok{(Income))}

\CommentTok{\# Select the state with the highest income}
\NormalTok{highest\_income\_state }\OtherTok{\textless{}{-}}\NormalTok{ sorted\_states }\SpecialCharTok{\%\textgreater{}\%} \FunctionTok{slice}\NormalTok{(}\DecValTok{1}\NormalTok{)}
\NormalTok{highest\_income\_state}
\end{Highlighting}
\end{Shaded}

\begin{verbatim}
## # A tibble: 1 x 9
##   State  Population Income Illiteracy `Life Exp` Murder `HS Grad` Frost   Area
##   <chr>       <dbl>  <dbl>      <dbl>      <dbl>  <dbl>     <dbl> <dbl>  <dbl>
## 1 Alaska        365   6315        1.5       69.3   11.3      66.7   152 566432
\end{verbatim}

\subparagraph{(c) Add a variable to the data frame which categorizes the
size of population: \textless= 4500 is S, \$\textgreater{} 4500 \$ is
L.}\label{c-add-a-variable-to-the-data-frame-which-categorizes-the-size-of-population-4500-is-s-4500-is-l.}

\begin{Shaded}
\begin{Highlighting}[]
\CommentTok{\# Add a variable categorizing the size of population}
\NormalTok{state.x77 }\OtherTok{\textless{}{-}}\NormalTok{ state.x77 }\SpecialCharTok{\%\textgreater{}\%} 
  \FunctionTok{mutate}\NormalTok{(}\AttributeTok{PopulationSize =} \FunctionTok{ifelse}\NormalTok{(Population }\SpecialCharTok{\textless{}=} \DecValTok{4500}\NormalTok{, }\StringTok{\textquotesingle{}S\textquotesingle{}}\NormalTok{, }\StringTok{\textquotesingle{}L\textquotesingle{}}\NormalTok{))}
\FunctionTok{head}\NormalTok{(state.x77)}
\end{Highlighting}
\end{Shaded}

\begin{verbatim}
## # A tibble: 6 x 10
##   State    Population Income Illiteracy `Life Exp` Murder `HS Grad` Frost   Area
##   <chr>         <dbl>  <dbl>      <dbl>      <dbl>  <dbl>     <dbl> <dbl>  <dbl>
## 1 Alabama        3615   3624        2.1       69.0   15.1      41.3    20  50708
## 2 Alaska          365   6315        1.5       69.3   11.3      66.7   152 566432
## 3 Arizona        2212   4530        1.8       70.6    7.8      58.1    15 113417
## 4 Arkansas       2110   3378        1.9       70.7   10.1      39.9    65  51945
## 5 Califor~      21198   5114        1.1       71.7   10.3      62.6    20 156361
## 6 Colorado       2541   4884        0.7       72.1    6.8      63.9   166 103766
## # i 1 more variable: PopulationSize <chr>
\end{verbatim}

\subparagraph{(d) Find out the average income and illiteracy of the two
groups of states, distinguishing by whether the states are small or
large.}\label{d-find-out-the-average-income-and-illiteracy-of-the-two-groups-of-states-distinguishing-by-whether-the-states-are-small-or-large.}

\begin{Shaded}
\begin{Highlighting}[]
\CommentTok{\# Find the average income and illiteracy of the two groups of states}
\NormalTok{grouped\_stats }\OtherTok{\textless{}{-}}\NormalTok{ state.x77 }\SpecialCharTok{\%\textgreater{}\%} 
  \FunctionTok{group\_by}\NormalTok{(PopulationSize) }\SpecialCharTok{\%\textgreater{}\%} 
  \FunctionTok{summarise}\NormalTok{(}
    \AttributeTok{Average\_Income =} \FunctionTok{mean}\NormalTok{(Income),}
    \AttributeTok{Average\_Illiteracy =} \FunctionTok{mean}\NormalTok{(Illiteracy)}
\NormalTok{  )}

\NormalTok{grouped\_stats}
\end{Highlighting}
\end{Shaded}

\begin{verbatim}
## # A tibble: 2 x 3
##   PopulationSize Average_Income Average_Illiteracy
##   <chr>                   <dbl>              <dbl>
## 1 L                       4608.               1.2 
## 2 S                       4355.               1.16
\end{verbatim}

\subsubsection{5.}\label{section-4}

\subparagraph{(a) Write a function to simulate n observations of (X1,X2)
which follow the uniform distribution over the square {[}0,1{]} ×
{[}0,1{]}.}\label{a-write-a-function-to-simulate-n-observations-of-x1x2-which-follow-the-uniform-distribution-over-the-square-01-01.}

simulate\_observations(n) generates n pairs of (X1,X2) uniformly
distributed over {[}0,1{]}×{[}0,1{]}.

\begin{Shaded}
\begin{Highlighting}[]
\NormalTok{simulate\_observations }\OtherTok{\textless{}{-}} \ControlFlowTok{function}\NormalTok{(n) \{}
\NormalTok{  X1 }\OtherTok{\textless{}{-}} \FunctionTok{runif}\NormalTok{(n, }\DecValTok{0}\NormalTok{, }\DecValTok{1}\NormalTok{)}
\NormalTok{  X2 }\OtherTok{\textless{}{-}} \FunctionTok{runif}\NormalTok{(n, }\DecValTok{0}\NormalTok{, }\DecValTok{1}\NormalTok{)}
  \FunctionTok{return}\NormalTok{(}\FunctionTok{data.frame}\NormalTok{(X1, X2))}
\NormalTok{\}}
\end{Highlighting}
\end{Shaded}

\subparagraph{(b) Write a function to calculate the proportion of the
observations that the distance between (X1,X2) and the nearest edge is
less than 0.25, and the proportion of them with the distance to the
nearest vertex less than
0.25.}\label{b-write-a-function-to-calculate-the-proportion-of-the-observations-that-the-distance-between-x1x2-and-the-nearest-edge-is-less-than-0.25-and-the-proportion-of-them-with-the-distance-to-the-nearest-vertex-less-than-0.25.}

edge\_proportion: The proportion of observations where the distance to
the nearest edge is less than 0.25.\\
vertex\_proportion: The proportion of observations where the distance to
the nearest vertex is less than 0.25.

\begin{Shaded}
\begin{Highlighting}[]
\NormalTok{calculate\_proportions }\OtherTok{\textless{}{-}} \ControlFlowTok{function}\NormalTok{(observations) \{}
\NormalTok{  n }\OtherTok{\textless{}{-}} \FunctionTok{nrow}\NormalTok{(observations)}
\NormalTok{  X1 }\OtherTok{\textless{}{-}}\NormalTok{ observations}\SpecialCharTok{$}\NormalTok{X1}
\NormalTok{  X2 }\OtherTok{\textless{}{-}}\NormalTok{ observations}\SpecialCharTok{$}\NormalTok{X2}
  
  \CommentTok{\# Distance to the nearest edge}
\NormalTok{  edge\_distances }\OtherTok{\textless{}{-}} \FunctionTok{pmin}\NormalTok{(X1, }\DecValTok{1} \SpecialCharTok{{-}}\NormalTok{ X1, X2, }\DecValTok{1} \SpecialCharTok{{-}}\NormalTok{ X2)}
\NormalTok{  edge\_proportion }\OtherTok{\textless{}{-}} \FunctionTok{mean}\NormalTok{(edge\_distances }\SpecialCharTok{\textless{}} \FloatTok{0.25}\NormalTok{)}
  
  \CommentTok{\# Distance to the nearest vertex}
\NormalTok{  vertex\_distances }\OtherTok{\textless{}{-}} \FunctionTok{pmin}\NormalTok{(}
    \FunctionTok{sqrt}\NormalTok{((X1 }\SpecialCharTok{{-}} \DecValTok{0}\NormalTok{)}\SpecialCharTok{\^{}}\DecValTok{2} \SpecialCharTok{+}\NormalTok{ (X2 }\SpecialCharTok{{-}} \DecValTok{0}\NormalTok{)}\SpecialCharTok{\^{}}\DecValTok{2}\NormalTok{),  }\CommentTok{\# distance to (0,0)}
    \FunctionTok{sqrt}\NormalTok{((X1 }\SpecialCharTok{{-}} \DecValTok{0}\NormalTok{)}\SpecialCharTok{\^{}}\DecValTok{2} \SpecialCharTok{+}\NormalTok{ (X2 }\SpecialCharTok{{-}} \DecValTok{1}\NormalTok{)}\SpecialCharTok{\^{}}\DecValTok{2}\NormalTok{),  }\CommentTok{\# distance to (0,1)}
    \FunctionTok{sqrt}\NormalTok{((X1 }\SpecialCharTok{{-}} \DecValTok{1}\NormalTok{)}\SpecialCharTok{\^{}}\DecValTok{2} \SpecialCharTok{+}\NormalTok{ (X2 }\SpecialCharTok{{-}} \DecValTok{0}\NormalTok{)}\SpecialCharTok{\^{}}\DecValTok{2}\NormalTok{),  }\CommentTok{\# distance to (1,0)}
    \FunctionTok{sqrt}\NormalTok{((X1 }\SpecialCharTok{{-}} \DecValTok{1}\NormalTok{)}\SpecialCharTok{\^{}}\DecValTok{2} \SpecialCharTok{+}\NormalTok{ (X2 }\SpecialCharTok{{-}} \DecValTok{1}\NormalTok{)}\SpecialCharTok{\^{}}\DecValTok{2}\NormalTok{)   }\CommentTok{\# distance to (1,1)}
\NormalTok{  )}
\NormalTok{  vertex\_proportion }\OtherTok{\textless{}{-}} \FunctionTok{mean}\NormalTok{(vertex\_distances }\SpecialCharTok{\textless{}} \FloatTok{0.25}\NormalTok{)}
  
  \FunctionTok{return}\NormalTok{(}\FunctionTok{list}\NormalTok{(}\AttributeTok{edge\_proportion =}\NormalTok{ edge\_proportion, }\AttributeTok{vertex\_proportion =}\NormalTok{ vertex\_proportion))}
\NormalTok{\}}
\end{Highlighting}
\end{Shaded}

\subparagraph{Example Usage:}\label{example-usage}

Here's how you can use the functions to simulate observations and
calculate the proportions:

\begin{Shaded}
\begin{Highlighting}[]
\CommentTok{\# Simulate 1000 observations}
\FunctionTok{set.seed}\NormalTok{(}\DecValTok{123}\NormalTok{)  }\CommentTok{\# Setting seed for reproducibility}
\NormalTok{observations }\OtherTok{\textless{}{-}} \FunctionTok{simulate\_observations}\NormalTok{(}\DecValTok{1000}\NormalTok{)}

\CommentTok{\# Calculate proportions}
\NormalTok{proportions }\OtherTok{\textless{}{-}} \FunctionTok{calculate\_proportions}\NormalTok{(observations)}
\FunctionTok{print}\NormalTok{(proportions)}
\end{Highlighting}
\end{Shaded}

\begin{verbatim}
## $edge_proportion
## [1] 0.735
## 
## $vertex_proportion
## [1] 0.198
\end{verbatim}

\subsubsection{6.}\label{section-5}

Mortality rates per 100,000 from male suicides for a number of age
groups and a number of countries are given in the following data frame.

\begin{Shaded}
\begin{Highlighting}[]
 \FunctionTok{library}\NormalTok{(tidyverse)}
\NormalTok{ suicrates }\OtherTok{\textless{}{-}} \FunctionTok{tibble}\NormalTok{(}\AttributeTok{Country =} \FunctionTok{c}\NormalTok{(}\StringTok{\textquotesingle{}Canada\textquotesingle{}}\NormalTok{, }\StringTok{\textquotesingle{}Israel\textquotesingle{}}\NormalTok{, }\StringTok{\textquotesingle{}Japan\textquotesingle{}}\NormalTok{, }\StringTok{\textquotesingle{}Austria\textquotesingle{}}\NormalTok{, }\StringTok{\textquotesingle{}France\textquotesingle{}}\NormalTok{, }\StringTok{\textquotesingle{}Germany\textquotesingle{}}\NormalTok{,}
 \StringTok{\textquotesingle{}Hungary\textquotesingle{}}\NormalTok{, }\StringTok{\textquotesingle{}Italy\textquotesingle{}}\NormalTok{, }\StringTok{\textquotesingle{}Netherlands\textquotesingle{}}\NormalTok{, }\StringTok{\textquotesingle{}Poland\textquotesingle{}}\NormalTok{, }\StringTok{\textquotesingle{}Spain\textquotesingle{}}\NormalTok{, }\StringTok{\textquotesingle{}Sweden\textquotesingle{}}\NormalTok{, }\StringTok{\textquotesingle{}Switzerland\textquotesingle{}}\NormalTok{, }\StringTok{\textquotesingle{}UK\textquotesingle{}}\NormalTok{, }\StringTok{\textquotesingle{}USA\textquotesingle{}}\NormalTok{),}
 \AttributeTok{Age25.34 =} \FunctionTok{c}\NormalTok{(}\DecValTok{22}\NormalTok{, }\DecValTok{9}\NormalTok{, }\DecValTok{22}\NormalTok{, }\DecValTok{29}\NormalTok{, }\DecValTok{16}\NormalTok{, }\DecValTok{28}\NormalTok{, }\DecValTok{48}\NormalTok{, }\DecValTok{7}\NormalTok{, }\DecValTok{8}\NormalTok{, }\DecValTok{26}\NormalTok{, }\DecValTok{4}\NormalTok{, }\DecValTok{28}\NormalTok{, }\DecValTok{22}\NormalTok{, }\DecValTok{10}\NormalTok{, }\DecValTok{20}\NormalTok{),}
 \AttributeTok{Age35.44 =} \FunctionTok{c}\NormalTok{(}\DecValTok{27}\NormalTok{, }\DecValTok{19}\NormalTok{, }\DecValTok{19}\NormalTok{, }\DecValTok{40}\NormalTok{, }\DecValTok{25}\NormalTok{, }\DecValTok{35}\NormalTok{, }\DecValTok{65}\NormalTok{, }\DecValTok{8}\NormalTok{, }\DecValTok{11}\NormalTok{, }\DecValTok{29}\NormalTok{, }\DecValTok{7}\NormalTok{, }\DecValTok{41}\NormalTok{, }\DecValTok{34}\NormalTok{, }\DecValTok{13}\NormalTok{, }\DecValTok{22}\NormalTok{),}
 \AttributeTok{Age45.54 =} \FunctionTok{c}\NormalTok{(}\DecValTok{31}\NormalTok{, }\DecValTok{10}\NormalTok{, }\DecValTok{21}\NormalTok{, }\DecValTok{52}\NormalTok{, }\DecValTok{36}\NormalTok{, }\DecValTok{41}\NormalTok{, }\DecValTok{84}\NormalTok{, }\DecValTok{11}\NormalTok{, }\DecValTok{18}\NormalTok{, }\DecValTok{36}\NormalTok{, }\DecValTok{10}\NormalTok{, }\DecValTok{46}\NormalTok{, }\DecValTok{41}\NormalTok{, }\DecValTok{15}\NormalTok{, }\DecValTok{28}\NormalTok{),}
 \AttributeTok{Age55.64 =} \FunctionTok{c}\NormalTok{(}\DecValTok{34}\NormalTok{, }\DecValTok{14}\NormalTok{, }\DecValTok{31}\NormalTok{, }\DecValTok{53}\NormalTok{, }\DecValTok{47}\NormalTok{, }\DecValTok{49}\NormalTok{, }\DecValTok{81}\NormalTok{, }\DecValTok{18}\NormalTok{, }\DecValTok{20}\NormalTok{, }\DecValTok{32}\NormalTok{, }\DecValTok{16}\NormalTok{, }\DecValTok{51}\NormalTok{, }\DecValTok{50}\NormalTok{, }\DecValTok{17}\NormalTok{, }\DecValTok{33}\NormalTok{),}
 \AttributeTok{Age65.74 =} \FunctionTok{c}\NormalTok{(}\DecValTok{24}\NormalTok{, }\DecValTok{27}\NormalTok{, }\DecValTok{49}\NormalTok{, }\DecValTok{69}\NormalTok{, }\DecValTok{56}\NormalTok{, }\DecValTok{52}\NormalTok{, }\DecValTok{107}\NormalTok{, }\DecValTok{27}\NormalTok{, }\DecValTok{28}\NormalTok{, }\DecValTok{28}\NormalTok{, }\DecValTok{22}\NormalTok{, }\DecValTok{35}\NormalTok{, }\DecValTok{51}\NormalTok{, }\DecValTok{22}\NormalTok{, }\DecValTok{37}\NormalTok{))}
\end{Highlighting}
\end{Shaded}

\subparagraph{(a) Transform suicrates into long
form.}\label{a-transform-suicrates-into-long-form.}

\begin{Shaded}
\begin{Highlighting}[]
 \FunctionTok{library}\NormalTok{(tidyr)}

\CommentTok{\# Transforming the data to long form}
\NormalTok{suicrates\_long }\OtherTok{\textless{}{-}}\NormalTok{ suicrates }\SpecialCharTok{\%\textgreater{}\%}
  \FunctionTok{pivot\_longer}\NormalTok{(}\AttributeTok{cols =} \FunctionTok{starts\_with}\NormalTok{(}\StringTok{"Age"}\NormalTok{), }\AttributeTok{names\_to =} \StringTok{"AgeGroup"}\NormalTok{, }\AttributeTok{values\_to =} \StringTok{"Rate"}\NormalTok{)}

\FunctionTok{print}\NormalTok{(suicrates\_long)}
\end{Highlighting}
\end{Shaded}

\begin{verbatim}
## # A tibble: 75 x 3
##    Country AgeGroup  Rate
##    <chr>   <chr>    <dbl>
##  1 Canada  Age25.34    22
##  2 Canada  Age35.44    27
##  3 Canada  Age45.54    31
##  4 Canada  Age55.64    34
##  5 Canada  Age65.74    24
##  6 Israel  Age25.34     9
##  7 Israel  Age35.44    19
##  8 Israel  Age45.54    10
##  9 Israel  Age55.64    14
## 10 Israel  Age65.74    27
## # i 65 more rows
\end{verbatim}

\subparagraph{(b) Construct side-by-side box plots for the data from
different age groups, and comment on what the graphic tells us about the
data.}\label{b-construct-side-by-side-box-plots-for-the-data-from-different-age-groups-and-comment-on-what-the-graphic-tells-us-about-the-data.}

\begin{Shaded}
\begin{Highlighting}[]
 \FunctionTok{library}\NormalTok{(ggplot2)}

\CommentTok{\# Constructing side{-}by{-}side box plots}
\FunctionTok{ggplot}\NormalTok{(suicrates\_long, }\FunctionTok{aes}\NormalTok{(}\AttributeTok{x =}\NormalTok{ AgeGroup, }\AttributeTok{y =}\NormalTok{ Rate)) }\SpecialCharTok{+}
  \FunctionTok{geom\_boxplot}\NormalTok{() }\SpecialCharTok{+}
  \FunctionTok{labs}\NormalTok{(}\AttributeTok{title =} \StringTok{"Suicide Rates by Age Group"}\NormalTok{,}
       \AttributeTok{x =} \StringTok{"Age Group"}\NormalTok{,}
       \AttributeTok{y =} \StringTok{"Suicide Rate per 100,000"}\NormalTok{) }\SpecialCharTok{+}
  \FunctionTok{theme\_minimal}\NormalTok{()}
\end{Highlighting}
\end{Shaded}

\includegraphics{3220103172_files/figure-latex/unnamed-chunk-21-1.pdf}

\subsubsection{7.}\label{section-6}

In the data set pressure, the relevant theory is that associated with
the Claudius- Clapeyron equation, by which the logarithm of the vapor
pressure is approximately inversely proportional to the absolute
temperature (temperature + 273). Transform the data in the manner
suggested by this theoretical relationship, plot the data, fit a
regression line, and add the line to the graph. Does the fit seem
adequate? \#\#\#\#\# (1) Load the `pressure' data set

\begin{Shaded}
\begin{Highlighting}[]
 \FunctionTok{data}\NormalTok{(pressure)}
\FunctionTok{head}\NormalTok{(pressure)}
\end{Highlighting}
\end{Shaded}

\begin{verbatim}
##   temperature pressure
## 1           0   0.0002
## 2          20   0.0012
## 3          40   0.0060
## 4          60   0.0300
## 5          80   0.0900
## 6         100   0.2700
\end{verbatim}

\subparagraph{(2) Transform the data}\label{transform-the-data}

According to the Clausius-Clapeyron equation, the logarithm of the vapor
pressure is approximately inversely proportional to the absolute
temperature (temperature + 273). We'll transform the temperature to its
absolute value and compute the logarithm of the vapor pressure:

\begin{Shaded}
\begin{Highlighting}[]
\NormalTok{pressure}\SpecialCharTok{$}\NormalTok{TemperatureK }\OtherTok{\textless{}{-}}\NormalTok{ pressure}\SpecialCharTok{$}\NormalTok{temperature }\SpecialCharTok{+} \DecValTok{273}
\NormalTok{pressure}\SpecialCharTok{$}\NormalTok{log\_pressure }\OtherTok{\textless{}{-}} \FunctionTok{log}\NormalTok{(pressure}\SpecialCharTok{$}\NormalTok{pressure)}
\end{Highlighting}
\end{Shaded}

\subparagraph{(3) Plot the transformed
data}\label{plot-the-transformed-data}

\begin{Shaded}
\begin{Highlighting}[]
\FunctionTok{plot}\NormalTok{(}\DecValTok{1} \SpecialCharTok{/}\NormalTok{ pressure}\SpecialCharTok{$}\NormalTok{TemperatureK, pressure}\SpecialCharTok{$}\NormalTok{log\_pressure,}
     \AttributeTok{xlab =} \StringTok{"1 / Temperature (1/K)"}\NormalTok{,}
     \AttributeTok{ylab =} \StringTok{"Log(Pressure)"}\NormalTok{,}
     \AttributeTok{main =} \StringTok{"Clausius{-}Clapeyron Plot"}\NormalTok{)}
\end{Highlighting}
\end{Shaded}

\includegraphics{3220103172_files/figure-latex/unnamed-chunk-24-1.pdf}
\#\#\#\#\# (4) Fit a regression line Fit a linear model to the
transformed data:

\begin{Shaded}
\begin{Highlighting}[]
\NormalTok{fit }\OtherTok{\textless{}{-}} \FunctionTok{lm}\NormalTok{(log\_pressure }\SpecialCharTok{\textasciitilde{}} \FunctionTok{I}\NormalTok{(}\DecValTok{1} \SpecialCharTok{/}\NormalTok{ TemperatureK), }\AttributeTok{data =}\NormalTok{ pressure)}
\FunctionTok{summary}\NormalTok{(fit)}
\end{Highlighting}
\end{Shaded}

\begin{verbatim}
## 
## Call:
## lm(formula = log_pressure ~ I(1/TemperatureK), data = pressure)
## 
## Residuals:
##       Min        1Q    Median        3Q       Max 
## -0.074183 -0.028889 -0.002508  0.021144  0.151494 
## 
## Coefficients:
##                     Estimate Std. Error t value Pr(>|t|)    
## (Intercept)        1.827e+01  4.429e-02   412.4   <2e-16 ***
## I(1/TemperatureK) -7.301e+03  1.822e+01  -400.7   <2e-16 ***
## ---
## Signif. codes:  0 '***' 0.001 '**' 0.01 '*' 0.05 '.' 0.1 ' ' 1
## 
## Residual standard error: 0.04874 on 17 degrees of freedom
## Multiple R-squared:  0.9999, Adjusted R-squared:  0.9999 
## F-statistic: 1.606e+05 on 1 and 17 DF,  p-value: < 2.2e-16
\end{verbatim}

\subparagraph{(5) Add the regression line to the
plot}\label{add-the-regression-line-to-the-plot}

\subparagraph{Here's the complete R code for the entire
process:}\label{heres-the-complete-r-code-for-the-entire-process}

\begin{Shaded}
\begin{Highlighting}[]
\CommentTok{\# Load the pressure data set}
\FunctionTok{data}\NormalTok{(pressure)}
\FunctionTok{head}\NormalTok{(pressure)}
\end{Highlighting}
\end{Shaded}

\begin{verbatim}
##   temperature pressure
## 1           0   0.0002
## 2          20   0.0012
## 3          40   0.0060
## 4          60   0.0300
## 5          80   0.0900
## 6         100   0.2700
\end{verbatim}

\begin{Shaded}
\begin{Highlighting}[]
\CommentTok{\# Transform the data}
\NormalTok{pressure}\SpecialCharTok{$}\NormalTok{TemperatureK }\OtherTok{\textless{}{-}}\NormalTok{ pressure}\SpecialCharTok{$}\NormalTok{temperature }\SpecialCharTok{+} \DecValTok{273}
\NormalTok{pressure}\SpecialCharTok{$}\NormalTok{log\_pressure }\OtherTok{\textless{}{-}} \FunctionTok{log}\NormalTok{(pressure}\SpecialCharTok{$}\NormalTok{pressure)}

\CommentTok{\# Plot the transformed data}
\FunctionTok{plot}\NormalTok{(}\DecValTok{1} \SpecialCharTok{/}\NormalTok{ pressure}\SpecialCharTok{$}\NormalTok{TemperatureK, pressure}\SpecialCharTok{$}\NormalTok{log\_pressure,}
     \AttributeTok{xlab =} \StringTok{"1 / Temperature (1/K)"}\NormalTok{,}
     \AttributeTok{ylab =} \StringTok{"Log(Pressure)"}\NormalTok{,}
     \AttributeTok{main =} \StringTok{"Clausius{-}Clapeyron Plot"}\NormalTok{)}

\CommentTok{\# Fit a regression line}
\NormalTok{fit }\OtherTok{\textless{}{-}} \FunctionTok{lm}\NormalTok{(log\_pressure }\SpecialCharTok{\textasciitilde{}} \FunctionTok{I}\NormalTok{(}\DecValTok{1} \SpecialCharTok{/}\NormalTok{ TemperatureK), }\AttributeTok{data =}\NormalTok{ pressure)}
\FunctionTok{summary}\NormalTok{(fit)}
\end{Highlighting}
\end{Shaded}

\begin{verbatim}
## 
## Call:
## lm(formula = log_pressure ~ I(1/TemperatureK), data = pressure)
## 
## Residuals:
##       Min        1Q    Median        3Q       Max 
## -0.074183 -0.028889 -0.002508  0.021144  0.151494 
## 
## Coefficients:
##                     Estimate Std. Error t value Pr(>|t|)    
## (Intercept)        1.827e+01  4.429e-02   412.4   <2e-16 ***
## I(1/TemperatureK) -7.301e+03  1.822e+01  -400.7   <2e-16 ***
## ---
## Signif. codes:  0 '***' 0.001 '**' 0.01 '*' 0.05 '.' 0.1 ' ' 1
## 
## Residual standard error: 0.04874 on 17 degrees of freedom
## Multiple R-squared:  0.9999, Adjusted R-squared:  0.9999 
## F-statistic: 1.606e+05 on 1 and 17 DF,  p-value: < 2.2e-16
\end{verbatim}

\begin{Shaded}
\begin{Highlighting}[]
\CommentTok{\# Add the regression line to the plot}
\FunctionTok{abline}\NormalTok{(fit, }\AttributeTok{col =} \StringTok{"red"}\NormalTok{)}
\end{Highlighting}
\end{Shaded}

\includegraphics{3220103172_files/figure-latex/unnamed-chunk-26-1.pdf}

Note that the \texttt{echo\ =\ FALSE} parameter was added to the code
chunk to prevent printing of the R code that generated the plot.

\end{document}
