% Options for packages loaded elsewhere
\PassOptionsToPackage{unicode}{hyperref}
\PassOptionsToPackage{hyphens}{url}
%
\documentclass[
]{article}
\usepackage{amsmath,amssymb}
\usepackage{iftex}
\ifPDFTeX
  \usepackage[T1]{fontenc}
  \usepackage[utf8]{inputenc}
  \usepackage{textcomp} % provide euro and other symbols
\else % if luatex or xetex
  \usepackage{unicode-math} % this also loads fontspec
  \defaultfontfeatures{Scale=MatchLowercase}
  \defaultfontfeatures[\rmfamily]{Ligatures=TeX,Scale=1}
\fi
\usepackage{lmodern}
\ifPDFTeX\else
  % xetex/luatex font selection
\fi
% Use upquote if available, for straight quotes in verbatim environments
\IfFileExists{upquote.sty}{\usepackage{upquote}}{}
\IfFileExists{microtype.sty}{% use microtype if available
  \usepackage[]{microtype}
  \UseMicrotypeSet[protrusion]{basicmath} % disable protrusion for tt fonts
}{}
\makeatletter
\@ifundefined{KOMAClassName}{% if non-KOMA class
  \IfFileExists{parskip.sty}{%
    \usepackage{parskip}
  }{% else
    \setlength{\parindent}{0pt}
    \setlength{\parskip}{6pt plus 2pt minus 1pt}}
}{% if KOMA class
  \KOMAoptions{parskip=half}}
\makeatother
\usepackage{xcolor}
\usepackage[margin=1in]{geometry}
\usepackage{color}
\usepackage{fancyvrb}
\newcommand{\VerbBar}{|}
\newcommand{\VERB}{\Verb[commandchars=\\\{\}]}
\DefineVerbatimEnvironment{Highlighting}{Verbatim}{commandchars=\\\{\}}
% Add ',fontsize=\small' for more characters per line
\usepackage{framed}
\definecolor{shadecolor}{RGB}{248,248,248}
\newenvironment{Shaded}{\begin{snugshade}}{\end{snugshade}}
\newcommand{\AlertTok}[1]{\textcolor[rgb]{0.94,0.16,0.16}{#1}}
\newcommand{\AnnotationTok}[1]{\textcolor[rgb]{0.56,0.35,0.01}{\textbf{\textit{#1}}}}
\newcommand{\AttributeTok}[1]{\textcolor[rgb]{0.13,0.29,0.53}{#1}}
\newcommand{\BaseNTok}[1]{\textcolor[rgb]{0.00,0.00,0.81}{#1}}
\newcommand{\BuiltInTok}[1]{#1}
\newcommand{\CharTok}[1]{\textcolor[rgb]{0.31,0.60,0.02}{#1}}
\newcommand{\CommentTok}[1]{\textcolor[rgb]{0.56,0.35,0.01}{\textit{#1}}}
\newcommand{\CommentVarTok}[1]{\textcolor[rgb]{0.56,0.35,0.01}{\textbf{\textit{#1}}}}
\newcommand{\ConstantTok}[1]{\textcolor[rgb]{0.56,0.35,0.01}{#1}}
\newcommand{\ControlFlowTok}[1]{\textcolor[rgb]{0.13,0.29,0.53}{\textbf{#1}}}
\newcommand{\DataTypeTok}[1]{\textcolor[rgb]{0.13,0.29,0.53}{#1}}
\newcommand{\DecValTok}[1]{\textcolor[rgb]{0.00,0.00,0.81}{#1}}
\newcommand{\DocumentationTok}[1]{\textcolor[rgb]{0.56,0.35,0.01}{\textbf{\textit{#1}}}}
\newcommand{\ErrorTok}[1]{\textcolor[rgb]{0.64,0.00,0.00}{\textbf{#1}}}
\newcommand{\ExtensionTok}[1]{#1}
\newcommand{\FloatTok}[1]{\textcolor[rgb]{0.00,0.00,0.81}{#1}}
\newcommand{\FunctionTok}[1]{\textcolor[rgb]{0.13,0.29,0.53}{\textbf{#1}}}
\newcommand{\ImportTok}[1]{#1}
\newcommand{\InformationTok}[1]{\textcolor[rgb]{0.56,0.35,0.01}{\textbf{\textit{#1}}}}
\newcommand{\KeywordTok}[1]{\textcolor[rgb]{0.13,0.29,0.53}{\textbf{#1}}}
\newcommand{\NormalTok}[1]{#1}
\newcommand{\OperatorTok}[1]{\textcolor[rgb]{0.81,0.36,0.00}{\textbf{#1}}}
\newcommand{\OtherTok}[1]{\textcolor[rgb]{0.56,0.35,0.01}{#1}}
\newcommand{\PreprocessorTok}[1]{\textcolor[rgb]{0.56,0.35,0.01}{\textit{#1}}}
\newcommand{\RegionMarkerTok}[1]{#1}
\newcommand{\SpecialCharTok}[1]{\textcolor[rgb]{0.81,0.36,0.00}{\textbf{#1}}}
\newcommand{\SpecialStringTok}[1]{\textcolor[rgb]{0.31,0.60,0.02}{#1}}
\newcommand{\StringTok}[1]{\textcolor[rgb]{0.31,0.60,0.02}{#1}}
\newcommand{\VariableTok}[1]{\textcolor[rgb]{0.00,0.00,0.00}{#1}}
\newcommand{\VerbatimStringTok}[1]{\textcolor[rgb]{0.31,0.60,0.02}{#1}}
\newcommand{\WarningTok}[1]{\textcolor[rgb]{0.56,0.35,0.01}{\textbf{\textit{#1}}}}
\usepackage{graphicx}
\makeatletter
\def\maxwidth{\ifdim\Gin@nat@width>\linewidth\linewidth\else\Gin@nat@width\fi}
\def\maxheight{\ifdim\Gin@nat@height>\textheight\textheight\else\Gin@nat@height\fi}
\makeatother
% Scale images if necessary, so that they will not overflow the page
% margins by default, and it is still possible to overwrite the defaults
% using explicit options in \includegraphics[width, height, ...]{}
\setkeys{Gin}{width=\maxwidth,height=\maxheight,keepaspectratio}
% Set default figure placement to htbp
\makeatletter
\def\fps@figure{htbp}
\makeatother
\setlength{\emergencystretch}{3em} % prevent overfull lines
\providecommand{\tightlist}{%
  \setlength{\itemsep}{0pt}\setlength{\parskip}{0pt}}
\setcounter{secnumdepth}{-\maxdimen} % remove section numbering
\ifLuaTeX
  \usepackage{selnolig}  % disable illegal ligatures
\fi
\usepackage{bookmark}
\IfFileExists{xurl.sty}{\usepackage{xurl}}{} % add URL line breaks if available
\urlstyle{same}
\hypersetup{
  pdftitle={3220103172},
  pdfauthor={Zhehao Chen},
  hidelinks,
  pdfcreator={LaTeX via pandoc}}

\title{3220103172}
\author{Zhehao Chen}
\date{2024-07-08}

\begin{document}
\maketitle

\subsubsection{1.}\label{section}

\subparagraph{(a) (5,8,1,5,8,1,\ldots,5,8,1) where there are 10
occurrences of
5.}\label{a-581581581-where-there-are-10-occurrences-of-5.}

To create a vector that repeats the sequence (5, 8, 1) so that there are
10 occurrences of 5, you can use the rep function:

\begin{Shaded}
\begin{Highlighting}[]
\NormalTok{tmp }\OtherTok{\textless{}{-}} \FunctionTok{c}\NormalTok{(}\DecValTok{5}\NormalTok{, }\DecValTok{8}\NormalTok{, }\DecValTok{1}\NormalTok{)}
\NormalTok{vec\_a }\OtherTok{\textless{}{-}} \FunctionTok{rep}\NormalTok{(tmp, }\AttributeTok{times =} \DecValTok{10}\NormalTok{)}
\FunctionTok{print}\NormalTok{(vec\_a)}
\end{Highlighting}
\end{Shaded}

\begin{verbatim}
##  [1] 5 8 1 5 8 1 5 8 1 5 8 1 5 8 1 5 8 1 5 8 1 5 8 1 5 8 1 5 8 1
\end{verbatim}

\subparagraph{(b) (5,5,\ldots,5,8,8,\ldots,8,1,1,\ldots,1) where there
are 30 occurrences of 5, 20 occurrences of 8 and 10 occurrences of
1.}\label{b-555888111-where-there-are-30-occurrences-of-5-20-occurrences-of-8-and-10-occurrences-of-1.}

To create a vector with 30 occurrences of 5, 20 occurrences of 8, and 10
occurrences of 1, you can use the rep function with the times argument
as a vector:

\begin{Shaded}
\begin{Highlighting}[]
\NormalTok{tmp }\OtherTok{\textless{}{-}} \FunctionTok{c}\NormalTok{(}\DecValTok{5}\NormalTok{, }\DecValTok{8}\NormalTok{, }\DecValTok{1}\NormalTok{)}
\NormalTok{times }\OtherTok{\textless{}{-}} \FunctionTok{c}\NormalTok{(}\DecValTok{30}\NormalTok{, }\DecValTok{20}\NormalTok{, }\DecValTok{10}\NormalTok{)}
\NormalTok{vec\_b }\OtherTok{\textless{}{-}} \FunctionTok{rep}\NormalTok{(tmp, }\AttributeTok{times =}\NormalTok{ times)}
\FunctionTok{print}\NormalTok{(vec\_b)}
\end{Highlighting}
\end{Shaded}

\begin{verbatim}
##  [1] 5 5 5 5 5 5 5 5 5 5 5 5 5 5 5 5 5 5 5 5 5 5 5 5 5 5 5 5 5 5 8 8 8 8 8 8 8 8
## [39] 8 8 8 8 8 8 8 8 8 8 8 8 1 1 1 1 1 1 1 1 1 1
\end{verbatim}

\subsubsection{2.}\label{section-1}

Execute the following lines which create two vectors of random integers
which are chosen with replacement from the integers 0,1,\ldots,999. Both
vectors have length 250.

\begin{Shaded}
\begin{Highlighting}[]
\NormalTok{ xVec }\OtherTok{\textless{}{-}} \FunctionTok{sample}\NormalTok{(}\DecValTok{0}\SpecialCharTok{:}\DecValTok{999}\NormalTok{, }\DecValTok{250}\NormalTok{, }\AttributeTok{replace=}\NormalTok{T)}
\NormalTok{ yVec }\OtherTok{\textless{}{-}} \FunctionTok{sample}\NormalTok{(}\DecValTok{0}\SpecialCharTok{:}\DecValTok{999}\NormalTok{, }\DecValTok{250}\NormalTok{, }\AttributeTok{replace=}\NormalTok{T)}
\NormalTok{ xVec}
\end{Highlighting}
\end{Shaded}

\begin{verbatim}
##   [1] 204 532 876 419 841 875 165 531 605  92 529 761 600 286 600 261 787  78
##  [19] 430 244 960 907 409 184 552 511 141 118 497 397 325 265 524 153 832 545
##  [37] 119 540 416 998 669  51 810 754 440 472 903 730 296 642 980 135 518 993
##  [55] 982 700 238 283 985 763 457 101  60 683 702 907  82 831 516 490 600 274
##  [73] 893 446 802 671 190 663 741  93 982 561 241 135 947 522 839 813   9 574
##  [91] 288 404 525 523 311  87 663 444 782 607 226 747 816 799  31 660 633 561
## [109]   1 128 928 105 138 550 202 755 709 804 991 882 233 845 264 507 202 550
## [127] 547 229 774 743 255 980 690 525 656 661 618 300 308 515  36 489 821 204
## [145] 454  53 704   9 535 663 752 885 701 248 656   4 965 162 133 102 510  15
## [163] 843 395  70  80 279 762  61 872 773 736 177 557 217 824 639  32 691 245
## [181]  66 850 708 882 420 860 630 147 476 780 704 152 950 755 360 329 677 496
## [199] 573 168   2 610 855 458 581   7 146 917 748 987 130 806  12 988 100 653
## [217] 757 243 278  34 366 550 969 301 910 971  99 186 426 792 386 521 269 868
## [235] 912 413 401 208 335 784 137 580 565 245 170 216 133 102 390 728
\end{verbatim}

\begin{Shaded}
\begin{Highlighting}[]
\NormalTok{ yVec}
\end{Highlighting}
\end{Shaded}

\begin{verbatim}
##   [1] 664 172 922 260 469  89 968 930  39 766 110 316 144 924  52 182 436 111
##  [19] 533 741 263 621 244 462 751  31 567 776 916 517 337 812 468 467 724 390
##  [37] 738 847  18 919 924 757 197 554 848 444 540 195 422 413  29 447 157 983
##  [55] 201 424 839 485 793 618 346 414 907 839 969  86 542 343 717 187 201 891
##  [73] 149 815 522 503 217 239 989 770 236 784 796 593 977 975 522 318  74 634
##  [91] 261 116 699  66 460 555 666 142 282 149 692  43 948  43 885  73 824 965
## [109] 714 349 386 922 550 264  46 316  25  89 755 229 210 490 143 656 818 986
## [127] 761 879  80 231 176 265 259 713 869 185 977 866 834 974  61 845 805  17
## [145] 715 557 889 851  89  94 176  58 205 288 704  72 923 193 301 177 521 475
## [163] 876 224 428 354 506 333 657  42 904 811  46 199 914 176 334 656 296 581
## [181] 979 734  70 497 780 112 896 106 777 612 550 393 678 174 129 669 248 446
## [199] 758 520 100 997 210 639 394 447  37 335 922 611 188 585 842 459 965 192
## [217] 388 712  91 432  89 486 146  40 182   2 151 451 337 117 522  73 730  43
## [235]  87  71 234 981 173 903 370 704 207 228 243 653 954 718 586 536
\end{verbatim}

\subparagraph{(a) Create the vector (y2 − x1,··· ,yn −
xn−1).}\label{a-create-the-vector-y2-x1-yn-xn1.}

yVec{[}2:250{]} - xVec{[}1:249{]} creates a vector by subtracting each
element in xVec (from the 1st to the 249th) from the corresponding next
element in yVec (from the 2nd to the 250th).

\begin{Shaded}
\begin{Highlighting}[]
\NormalTok{aVec }\OtherTok{\textless{}{-}}\NormalTok{ yVec[}\DecValTok{2}\SpecialCharTok{:}\DecValTok{250}\NormalTok{] }\SpecialCharTok{{-}}\NormalTok{ xVec[}\DecValTok{1}\SpecialCharTok{:}\DecValTok{249}\NormalTok{]}
\FunctionTok{print}\NormalTok{(aVec)}
\end{Highlighting}
\end{Shaded}

\begin{verbatim}
##   [1]  -32  390 -616   50 -752   93  765 -492  161   18 -213 -617  324 -234 -418
##  [16]  175 -676  455  311   19 -339 -663   53  567 -521   56  635  798   20  -60
##  [31]  487  203  -57  571 -442  193  728 -522  503  -74   88  146 -256   94    4
##  [46]   68 -708 -308  117 -613 -533   22  465 -792 -558  139  247  510 -367 -417
##  [61]  -43  806  779  286 -616 -365  261 -114 -329 -289  291 -125  -78   76 -299
##  [76] -454   49  326   29  143 -198  235  352  842   28    0 -521 -739  625 -313
##  [91] -172  295 -459  -63  244  579 -521 -162 -633   85 -183  201 -773   86   42
## [106]  164  332  153  348  258   -6  445  126 -504  114 -730 -620  -49 -762 -672
## [121]  257 -702  392  311  784  211  332 -149 -543 -567   10 -721   23  344 -471
## [136]  316  248  534  666 -454  809  316 -804  511  103  836  147   80 -441 -487
## [151] -694 -680 -413  456 -584  919 -772  139   44  419  -35  861 -619   33  284
## [166]  426   54 -105  -19   32   38 -690   22  357  -41 -490   17  264 -110  734
## [181]  668 -780 -211 -102 -308   36 -524  630  136 -230 -311  526 -776 -626  309
## [196]  -81 -231  262  -53  -68  995 -400 -216  -64 -134   30  189    5 -137 -799
## [211]  455   36  447  -23   92 -265  -45 -152  154   55  120 -404 -929 -119 -908
## [226] -820  352  151 -309 -270 -313  209 -226 -781 -841 -179  580  -35  568 -414
## [241]  567 -373 -337   -2  483  738  585  484  146
\end{verbatim}

\subparagraph{(b) Pick out the values in yVec which are \textgreater{}
600.}\label{b-pick-out-the-values-in-yvec-which-are-600.}

yVec{[}yVec \textgreater{} 600{]} selects the values from yVec that are
greater than 600.

\begin{Shaded}
\begin{Highlighting}[]
\NormalTok{bVec }\OtherTok{\textless{}{-}}\NormalTok{ yVec[yVec }\SpecialCharTok{\textgreater{}} \DecValTok{600}\NormalTok{]}
\FunctionTok{print}\NormalTok{(bVec)}
\end{Highlighting}
\end{Shaded}

\begin{verbatim}
##  [1] 664 922 968 930 766 924 741 621 751 776 916 812 724 738 847 919 924 757 848
## [20] 983 839 793 618 907 839 969 717 891 815 989 770 784 796 977 975 634 699 666
## [39] 692 948 885 824 965 714 922 755 656 818 986 761 879 713 869 977 866 834 974
## [58] 845 805 715 889 851 704 923 876 657 904 811 914 656 979 734 780 896 777 612
## [77] 678 669 758 997 639 922 611 842 965 712 730 981 903 704 653 954 718
\end{verbatim}

\subparagraph{(c) What are the index positions in yVec of the values
which are \textgreater{}
600?}\label{c-what-are-the-index-positions-in-yvec-of-the-values-which-are-600}

which(yVec \textgreater{} 600) returns the index positions of elements
in yVec that are greater than 600.

\begin{Shaded}
\begin{Highlighting}[]
\NormalTok{cVec }\OtherTok{\textless{}{-}} \FunctionTok{which}\NormalTok{(yVec }\SpecialCharTok{\textgreater{}} \DecValTok{600}\NormalTok{)}
\FunctionTok{print}\NormalTok{(cVec)}
\end{Highlighting}
\end{Shaded}

\begin{verbatim}
##  [1]   1   3   7   8  10  14  20  22  25  28  29  32  35  37  38  40  41  42  45
## [20]  54  57  59  60  63  64  65  69  72  74  79  80  82  83  85  86  90  93  97
## [39] 101 103 105 107 108 109 112 119 124 125 126 127 128 134 135 137 138 139 140
## [58] 142 143 145 147 148 155 157 163 169 171 172 175 178 181 182 185 187 189 190
## [77] 193 196 199 202 204 209 210 213 215 218 233 238 240 242 246 247 248
\end{verbatim}

\subparagraph{(d) Sort the numbers in the vector xVec in the order of
increasing values in
yVec.}\label{d-sort-the-numbers-in-the-vector-xvec-in-the-order-of-increasing-values-in-yvec.}

xVec{[}order(yVec){]} sorts xVec based on the order of the corresponding
elements in yVec.

\begin{Shaded}
\begin{Highlighting}[]
\NormalTok{dVec }\OtherTok{\textless{}{-}}\NormalTok{ xVec[}\FunctionTok{order}\NormalTok{(yVec)]}
\FunctionTok{print}\NormalTok{(dVec)}
\end{Highlighting}
\end{Shaded}

\begin{verbatim}
##   [1] 971 204 416 709 980 511 146 605 301 872 747 799 868 202 177 600 885  36
##  [19] 523 708 413   4 660 521   9 774 907 912 875 804 535 366 278 663   2 147
##  [37] 529  78 860 404 792 360 444 264 600 969 893 607  99 518 532 335 755 255
##  [55] 752 824 102 261 910 661 490 130 653 162 730 810 557 982 600 701 565 233
##  [73] 855 190 395 245 882 743 401 982 663 170 409 677 690 419 288 960 550 980
##  [91] 782 248 691 133 761 755 813 762 639 917 325 426 831 457 128  80 137 928
## [109] 757 545 152 581 642 101 296 700  70  34 787 472 496 135   7 186 988 311
## [127] 184 153 524 841  15 283 550 845 882 671 279 397 168 510 802 839 386 430
## [145] 728 903  82 138 704 754  87  53 141 245 806 390 135 987 780 763 907 574
## [163] 458 216 507  32  61 204 663 329 950 226 525 656 580 243 525   1 454 516
## [181] 102 832 269 850 119 244 552 991  51 573 547  92  93 118 476 420 561 985
## [199] 241 821 736 265 446 202 633 308 238 683  12 489 540 440   9 300 656 843
## [217] 229  31 704 274 630 784 773  60 217 497 998 876 105 748 965 286 669 531
## [235] 816 133 561 100 165 702 515 522 947 618  66 208 993 550 741 610
\end{verbatim}

\subparagraph{(e) Pick out the elements in yVec at index positions
1,4,7,10,13,···}\label{e-pick-out-the-elements-in-yvec-at-index-positions-1471013}

yVec{[}seq(1, length(yVec), by = 3){]} picks out every third element in
yVec starting from the first position.

\begin{Shaded}
\begin{Highlighting}[]
\NormalTok{eVec }\OtherTok{\textless{}{-}}\NormalTok{ yVec[}\FunctionTok{seq}\NormalTok{(}\DecValTok{1}\NormalTok{, }\FunctionTok{length}\NormalTok{(yVec), }\AttributeTok{by =} \DecValTok{3}\NormalTok{)]}
\FunctionTok{print}\NormalTok{(eVec)}
\end{Highlighting}
\end{Shaded}

\begin{verbatim}
##  [1] 664 260 968 766 144 182 533 621 751 776 337 467 738 919 197 444 422 447 201
## [20] 485 346 839 542 187 149 503 989 784 977 318 261  66 666 149 948  73 714 922
## [39]  46  89 210 656 761 231 259 185 834 845 715 851 176 288 923 177 876 354 657
## [58] 811 914 656 979 497 896 612 678 669 758 997 394 335 188 459 388 432 146   2
## [77] 337  73  87 981 370 228 954 536
\end{verbatim}

\subsubsection{3.}\label{section-2}

\begin{Shaded}
\begin{Highlighting}[]
\NormalTok{X }\OtherTok{\textless{}{-}} \FunctionTok{c}\NormalTok{(}\DecValTok{34}\NormalTok{, }\DecValTok{33}\NormalTok{, }\DecValTok{65}\NormalTok{, }\DecValTok{37}\NormalTok{, }\DecValTok{89}\NormalTok{, }\ConstantTok{NA}\NormalTok{, }\DecValTok{43}\NormalTok{, }\ConstantTok{NA}\NormalTok{, }\DecValTok{11}\NormalTok{, }\ConstantTok{NA}\NormalTok{, }\DecValTok{23}\NormalTok{, }\ConstantTok{NA}\NormalTok{)}
\NormalTok{X}
\end{Highlighting}
\end{Shaded}

\begin{verbatim}
##  [1] 34 33 65 37 89 NA 43 NA 11 NA 23 NA
\end{verbatim}

\subparagraph{Write a piece of R code to count the number of occurrences
of NA in
X?}\label{write-a-piece-of-r-code-to-count-the-number-of-occurrences-of-na-in-x}

is.na(X) returns a logical vector of the same length as X with TRUE for
each NA and FALSE otherwise.\\
sum(is.na(X)) sums the TRUE values, effectively counting the number of
NA values in X.

\begin{Shaded}
\begin{Highlighting}[]
\NormalTok{na\_count }\OtherTok{\textless{}{-}} \FunctionTok{sum}\NormalTok{(}\FunctionTok{is.na}\NormalTok{(X))}
\FunctionTok{print}\NormalTok{(na\_count)}
\end{Highlighting}
\end{Shaded}

\begin{verbatim}
## [1] 4
\end{verbatim}

\subsubsection{4.}\label{section-3}

For this problem we'll use the (built-in) dataset state.x77.

\begin{Shaded}
\begin{Highlighting}[]
\FunctionTok{library}\NormalTok{(tidyverse)}
\end{Highlighting}
\end{Shaded}

\begin{verbatim}
## -- Attaching core tidyverse packages ------------------------ tidyverse 2.0.0 --
## v dplyr     1.1.4     v readr     2.1.5
## v forcats   1.0.0     v stringr   1.5.1
## v ggplot2   3.5.1     v tibble    3.2.1
## v lubridate 1.9.3     v tidyr     1.3.1
## v purrr     1.0.2     
## -- Conflicts ------------------------------------------ tidyverse_conflicts() --
## x dplyr::filter() masks stats::filter()
## x dplyr::lag()    masks stats::lag()
## i Use the conflicted package (<http://conflicted.r-lib.org/>) to force all conflicts to become errors
\end{verbatim}

\begin{Shaded}
\begin{Highlighting}[]
\FunctionTok{data}\NormalTok{(state)}
\NormalTok{state.x77 }\OtherTok{\textless{}{-}} \FunctionTok{as\_tibble}\NormalTok{(state.x77, }\AttributeTok{rownames =} \StringTok{\textquotesingle{}State\textquotesingle{}}\NormalTok{)}
\end{Highlighting}
\end{Shaded}

\subparagraph{(a) Select all the states having an income less than 4300,
and calculate the average income of these
states.}\label{a-select-all-the-states-having-an-income-less-than-4300-and-calculate-the-average-income-of-these-states.}

\begin{Shaded}
\begin{Highlighting}[]
\CommentTok{\# Select states with income less than 4300}
\NormalTok{low\_income\_states }\OtherTok{\textless{}{-}}\NormalTok{ state.x77 }\SpecialCharTok{\%\textgreater{}\%} \FunctionTok{filter}\NormalTok{(Income }\SpecialCharTok{\textless{}} \DecValTok{4300}\NormalTok{)}

\CommentTok{\# Calculate the average income of these states}
\NormalTok{average\_low\_income }\OtherTok{\textless{}{-}} \FunctionTok{mean}\NormalTok{(low\_income\_states}\SpecialCharTok{$}\NormalTok{Income)}
\NormalTok{average\_low\_income}
\end{Highlighting}
\end{Shaded}

\begin{verbatim}
## [1] 3830.6
\end{verbatim}

\subparagraph{(b) Sort the data by income and select the state with the
highest
income.}\label{b-sort-the-data-by-income-and-select-the-state-with-the-highest-income.}

\begin{Shaded}
\begin{Highlighting}[]
\CommentTok{\# Sort the data by income}
\NormalTok{sorted\_states }\OtherTok{\textless{}{-}}\NormalTok{ state.x77 }\SpecialCharTok{\%\textgreater{}\%} \FunctionTok{arrange}\NormalTok{(}\FunctionTok{desc}\NormalTok{(Income))}

\CommentTok{\# Select the state with the highest income}
\NormalTok{highest\_income\_state }\OtherTok{\textless{}{-}}\NormalTok{ sorted\_states }\SpecialCharTok{\%\textgreater{}\%} \FunctionTok{slice}\NormalTok{(}\DecValTok{1}\NormalTok{)}
\NormalTok{highest\_income\_state}
\end{Highlighting}
\end{Shaded}

\begin{verbatim}
## # A tibble: 1 x 9
##   State  Population Income Illiteracy `Life Exp` Murder `HS Grad` Frost   Area
##   <chr>       <dbl>  <dbl>      <dbl>      <dbl>  <dbl>     <dbl> <dbl>  <dbl>
## 1 Alaska        365   6315        1.5       69.3   11.3      66.7   152 566432
\end{verbatim}

\subparagraph{(c) Add a variable to the data frame which categorizes the
size of population: \textless= 4500 is S, \$\textgreater{} 4500 \$ is
L.}\label{c-add-a-variable-to-the-data-frame-which-categorizes-the-size-of-population-4500-is-s-4500-is-l.}

\begin{Shaded}
\begin{Highlighting}[]
\CommentTok{\# Add a variable categorizing the size of population}
\NormalTok{state.x77 }\OtherTok{\textless{}{-}}\NormalTok{ state.x77 }\SpecialCharTok{\%\textgreater{}\%} 
  \FunctionTok{mutate}\NormalTok{(}\AttributeTok{PopulationSize =} \FunctionTok{ifelse}\NormalTok{(Population }\SpecialCharTok{\textless{}=} \DecValTok{4500}\NormalTok{, }\StringTok{\textquotesingle{}S\textquotesingle{}}\NormalTok{, }\StringTok{\textquotesingle{}L\textquotesingle{}}\NormalTok{))}
\FunctionTok{head}\NormalTok{(state.x77)}
\end{Highlighting}
\end{Shaded}

\begin{verbatim}
## # A tibble: 6 x 10
##   State    Population Income Illiteracy `Life Exp` Murder `HS Grad` Frost   Area
##   <chr>         <dbl>  <dbl>      <dbl>      <dbl>  <dbl>     <dbl> <dbl>  <dbl>
## 1 Alabama        3615   3624        2.1       69.0   15.1      41.3    20  50708
## 2 Alaska          365   6315        1.5       69.3   11.3      66.7   152 566432
## 3 Arizona        2212   4530        1.8       70.6    7.8      58.1    15 113417
## 4 Arkansas       2110   3378        1.9       70.7   10.1      39.9    65  51945
## 5 Califor~      21198   5114        1.1       71.7   10.3      62.6    20 156361
## 6 Colorado       2541   4884        0.7       72.1    6.8      63.9   166 103766
## # i 1 more variable: PopulationSize <chr>
\end{verbatim}

\subparagraph{(d) Find out the average income and illiteracy of the two
groups of states, distinguishing by whether the states are small or
large.}\label{d-find-out-the-average-income-and-illiteracy-of-the-two-groups-of-states-distinguishing-by-whether-the-states-are-small-or-large.}

\begin{Shaded}
\begin{Highlighting}[]
\CommentTok{\# Find the average income and illiteracy of the two groups of states}
\NormalTok{grouped\_stats }\OtherTok{\textless{}{-}}\NormalTok{ state.x77 }\SpecialCharTok{\%\textgreater{}\%} 
  \FunctionTok{group\_by}\NormalTok{(PopulationSize) }\SpecialCharTok{\%\textgreater{}\%} 
  \FunctionTok{summarise}\NormalTok{(}
    \AttributeTok{Average\_Income =} \FunctionTok{mean}\NormalTok{(Income),}
    \AttributeTok{Average\_Illiteracy =} \FunctionTok{mean}\NormalTok{(Illiteracy)}
\NormalTok{  )}

\NormalTok{grouped\_stats}
\end{Highlighting}
\end{Shaded}

\begin{verbatim}
## # A tibble: 2 x 3
##   PopulationSize Average_Income Average_Illiteracy
##   <chr>                   <dbl>              <dbl>
## 1 L                       4608.               1.2 
## 2 S                       4355.               1.16
\end{verbatim}

\subsubsection{5.}\label{section-4}

\subparagraph{(a) Write a function to simulate n observations of (X1,X2)
which follow the uniform distribution over the square {[}0,1{]} ×
{[}0,1{]}.}\label{a-write-a-function-to-simulate-n-observations-of-x1x2-which-follow-the-uniform-distribution-over-the-square-01-01.}

simulate\_observations(n) generates n pairs of (X1,X2) uniformly
distributed over {[}0,1{]}×{[}0,1{]}.

\begin{Shaded}
\begin{Highlighting}[]
\NormalTok{simulate\_observations }\OtherTok{\textless{}{-}} \ControlFlowTok{function}\NormalTok{(n) \{}
\NormalTok{  X1 }\OtherTok{\textless{}{-}} \FunctionTok{runif}\NormalTok{(n, }\DecValTok{0}\NormalTok{, }\DecValTok{1}\NormalTok{)}
\NormalTok{  X2 }\OtherTok{\textless{}{-}} \FunctionTok{runif}\NormalTok{(n, }\DecValTok{0}\NormalTok{, }\DecValTok{1}\NormalTok{)}
  \FunctionTok{return}\NormalTok{(}\FunctionTok{data.frame}\NormalTok{(X1, X2))}
\NormalTok{\}}
\end{Highlighting}
\end{Shaded}

\subparagraph{(b) Write a function to calculate the proportion of the
observations that the distance between (X1,X2) and the nearest edge is
less than 0.25, and the proportion of them with the distance to the
nearest vertex less than
0.25.}\label{b-write-a-function-to-calculate-the-proportion-of-the-observations-that-the-distance-between-x1x2-and-the-nearest-edge-is-less-than-0.25-and-the-proportion-of-them-with-the-distance-to-the-nearest-vertex-less-than-0.25.}

edge\_proportion: The proportion of observations where the distance to
the nearest edge is less than 0.25.\\
vertex\_proportion: The proportion of observations where the distance to
the nearest vertex is less than 0.25.

\begin{Shaded}
\begin{Highlighting}[]
\NormalTok{calculate\_proportions }\OtherTok{\textless{}{-}} \ControlFlowTok{function}\NormalTok{(observations) \{}
\NormalTok{  n }\OtherTok{\textless{}{-}} \FunctionTok{nrow}\NormalTok{(observations)}
\NormalTok{  X1 }\OtherTok{\textless{}{-}}\NormalTok{ observations}\SpecialCharTok{$}\NormalTok{X1}
\NormalTok{  X2 }\OtherTok{\textless{}{-}}\NormalTok{ observations}\SpecialCharTok{$}\NormalTok{X2}
  
  \CommentTok{\# Distance to the nearest edge}
\NormalTok{  edge\_distances }\OtherTok{\textless{}{-}} \FunctionTok{pmin}\NormalTok{(X1, }\DecValTok{1} \SpecialCharTok{{-}}\NormalTok{ X1, X2, }\DecValTok{1} \SpecialCharTok{{-}}\NormalTok{ X2)}
\NormalTok{  edge\_proportion }\OtherTok{\textless{}{-}} \FunctionTok{mean}\NormalTok{(edge\_distances }\SpecialCharTok{\textless{}} \FloatTok{0.25}\NormalTok{)}
  
  \CommentTok{\# Distance to the nearest vertex}
\NormalTok{  vertex\_distances }\OtherTok{\textless{}{-}} \FunctionTok{pmin}\NormalTok{(}
    \FunctionTok{sqrt}\NormalTok{((X1 }\SpecialCharTok{{-}} \DecValTok{0}\NormalTok{)}\SpecialCharTok{\^{}}\DecValTok{2} \SpecialCharTok{+}\NormalTok{ (X2 }\SpecialCharTok{{-}} \DecValTok{0}\NormalTok{)}\SpecialCharTok{\^{}}\DecValTok{2}\NormalTok{),  }\CommentTok{\# distance to (0,0)}
    \FunctionTok{sqrt}\NormalTok{((X1 }\SpecialCharTok{{-}} \DecValTok{0}\NormalTok{)}\SpecialCharTok{\^{}}\DecValTok{2} \SpecialCharTok{+}\NormalTok{ (X2 }\SpecialCharTok{{-}} \DecValTok{1}\NormalTok{)}\SpecialCharTok{\^{}}\DecValTok{2}\NormalTok{),  }\CommentTok{\# distance to (0,1)}
    \FunctionTok{sqrt}\NormalTok{((X1 }\SpecialCharTok{{-}} \DecValTok{1}\NormalTok{)}\SpecialCharTok{\^{}}\DecValTok{2} \SpecialCharTok{+}\NormalTok{ (X2 }\SpecialCharTok{{-}} \DecValTok{0}\NormalTok{)}\SpecialCharTok{\^{}}\DecValTok{2}\NormalTok{),  }\CommentTok{\# distance to (1,0)}
    \FunctionTok{sqrt}\NormalTok{((X1 }\SpecialCharTok{{-}} \DecValTok{1}\NormalTok{)}\SpecialCharTok{\^{}}\DecValTok{2} \SpecialCharTok{+}\NormalTok{ (X2 }\SpecialCharTok{{-}} \DecValTok{1}\NormalTok{)}\SpecialCharTok{\^{}}\DecValTok{2}\NormalTok{)   }\CommentTok{\# distance to (1,1)}
\NormalTok{  )}
\NormalTok{  vertex\_proportion }\OtherTok{\textless{}{-}} \FunctionTok{mean}\NormalTok{(vertex\_distances }\SpecialCharTok{\textless{}} \FloatTok{0.25}\NormalTok{)}
  
  \FunctionTok{return}\NormalTok{(}\FunctionTok{list}\NormalTok{(}\AttributeTok{edge\_proportion =}\NormalTok{ edge\_proportion, }\AttributeTok{vertex\_proportion =}\NormalTok{ vertex\_proportion))}
\NormalTok{\}}
\end{Highlighting}
\end{Shaded}

\subparagraph{Example Usage:}\label{example-usage}

Here's how you can use the functions to simulate observations and
calculate the proportions:

\begin{Shaded}
\begin{Highlighting}[]
\CommentTok{\# Simulate 1000 observations}
\FunctionTok{set.seed}\NormalTok{(}\DecValTok{123}\NormalTok{)  }\CommentTok{\# Setting seed for reproducibility}
\NormalTok{observations }\OtherTok{\textless{}{-}} \FunctionTok{simulate\_observations}\NormalTok{(}\DecValTok{1000}\NormalTok{)}

\CommentTok{\# Calculate proportions}
\NormalTok{proportions }\OtherTok{\textless{}{-}} \FunctionTok{calculate\_proportions}\NormalTok{(observations)}
\FunctionTok{print}\NormalTok{(proportions)}
\end{Highlighting}
\end{Shaded}

\begin{verbatim}
## $edge_proportion
## [1] 0.735
## 
## $vertex_proportion
## [1] 0.198
\end{verbatim}

\subsubsection{6.}\label{section-5}

Mortality rates per 100,000 from male suicides for a number of age
groups and a number of countries are given in the following data frame.

\begin{Shaded}
\begin{Highlighting}[]
 \FunctionTok{library}\NormalTok{(tidyverse)}
\NormalTok{ suicrates }\OtherTok{\textless{}{-}} \FunctionTok{tibble}\NormalTok{(}\AttributeTok{Country =} \FunctionTok{c}\NormalTok{(}\StringTok{\textquotesingle{}Canada\textquotesingle{}}\NormalTok{, }\StringTok{\textquotesingle{}Israel\textquotesingle{}}\NormalTok{, }\StringTok{\textquotesingle{}Japan\textquotesingle{}}\NormalTok{, }\StringTok{\textquotesingle{}Austria\textquotesingle{}}\NormalTok{, }\StringTok{\textquotesingle{}France\textquotesingle{}}\NormalTok{, }\StringTok{\textquotesingle{}Germany\textquotesingle{}}\NormalTok{,}
 \StringTok{\textquotesingle{}Hungary\textquotesingle{}}\NormalTok{, }\StringTok{\textquotesingle{}Italy\textquotesingle{}}\NormalTok{, }\StringTok{\textquotesingle{}Netherlands\textquotesingle{}}\NormalTok{, }\StringTok{\textquotesingle{}Poland\textquotesingle{}}\NormalTok{, }\StringTok{\textquotesingle{}Spain\textquotesingle{}}\NormalTok{, }\StringTok{\textquotesingle{}Sweden\textquotesingle{}}\NormalTok{, }\StringTok{\textquotesingle{}Switzerland\textquotesingle{}}\NormalTok{, }\StringTok{\textquotesingle{}UK\textquotesingle{}}\NormalTok{, }\StringTok{\textquotesingle{}USA\textquotesingle{}}\NormalTok{),}
 \AttributeTok{Age25.34 =} \FunctionTok{c}\NormalTok{(}\DecValTok{22}\NormalTok{, }\DecValTok{9}\NormalTok{, }\DecValTok{22}\NormalTok{, }\DecValTok{29}\NormalTok{, }\DecValTok{16}\NormalTok{, }\DecValTok{28}\NormalTok{, }\DecValTok{48}\NormalTok{, }\DecValTok{7}\NormalTok{, }\DecValTok{8}\NormalTok{, }\DecValTok{26}\NormalTok{, }\DecValTok{4}\NormalTok{, }\DecValTok{28}\NormalTok{, }\DecValTok{22}\NormalTok{, }\DecValTok{10}\NormalTok{, }\DecValTok{20}\NormalTok{),}
 \AttributeTok{Age35.44 =} \FunctionTok{c}\NormalTok{(}\DecValTok{27}\NormalTok{, }\DecValTok{19}\NormalTok{, }\DecValTok{19}\NormalTok{, }\DecValTok{40}\NormalTok{, }\DecValTok{25}\NormalTok{, }\DecValTok{35}\NormalTok{, }\DecValTok{65}\NormalTok{, }\DecValTok{8}\NormalTok{, }\DecValTok{11}\NormalTok{, }\DecValTok{29}\NormalTok{, }\DecValTok{7}\NormalTok{, }\DecValTok{41}\NormalTok{, }\DecValTok{34}\NormalTok{, }\DecValTok{13}\NormalTok{, }\DecValTok{22}\NormalTok{),}
 \AttributeTok{Age45.54 =} \FunctionTok{c}\NormalTok{(}\DecValTok{31}\NormalTok{, }\DecValTok{10}\NormalTok{, }\DecValTok{21}\NormalTok{, }\DecValTok{52}\NormalTok{, }\DecValTok{36}\NormalTok{, }\DecValTok{41}\NormalTok{, }\DecValTok{84}\NormalTok{, }\DecValTok{11}\NormalTok{, }\DecValTok{18}\NormalTok{, }\DecValTok{36}\NormalTok{, }\DecValTok{10}\NormalTok{, }\DecValTok{46}\NormalTok{, }\DecValTok{41}\NormalTok{, }\DecValTok{15}\NormalTok{, }\DecValTok{28}\NormalTok{),}
 \AttributeTok{Age55.64 =} \FunctionTok{c}\NormalTok{(}\DecValTok{34}\NormalTok{, }\DecValTok{14}\NormalTok{, }\DecValTok{31}\NormalTok{, }\DecValTok{53}\NormalTok{, }\DecValTok{47}\NormalTok{, }\DecValTok{49}\NormalTok{, }\DecValTok{81}\NormalTok{, }\DecValTok{18}\NormalTok{, }\DecValTok{20}\NormalTok{, }\DecValTok{32}\NormalTok{, }\DecValTok{16}\NormalTok{, }\DecValTok{51}\NormalTok{, }\DecValTok{50}\NormalTok{, }\DecValTok{17}\NormalTok{, }\DecValTok{33}\NormalTok{),}
 \AttributeTok{Age65.74 =} \FunctionTok{c}\NormalTok{(}\DecValTok{24}\NormalTok{, }\DecValTok{27}\NormalTok{, }\DecValTok{49}\NormalTok{, }\DecValTok{69}\NormalTok{, }\DecValTok{56}\NormalTok{, }\DecValTok{52}\NormalTok{, }\DecValTok{107}\NormalTok{, }\DecValTok{27}\NormalTok{, }\DecValTok{28}\NormalTok{, }\DecValTok{28}\NormalTok{, }\DecValTok{22}\NormalTok{, }\DecValTok{35}\NormalTok{, }\DecValTok{51}\NormalTok{, }\DecValTok{22}\NormalTok{, }\DecValTok{37}\NormalTok{))}
\end{Highlighting}
\end{Shaded}

\subparagraph{(a) Transform suicrates into long
form.}\label{a-transform-suicrates-into-long-form.}

\begin{Shaded}
\begin{Highlighting}[]
 \FunctionTok{library}\NormalTok{(tidyr)}

\CommentTok{\# Transforming the data to long form}
\NormalTok{suicrates\_long }\OtherTok{\textless{}{-}}\NormalTok{ suicrates }\SpecialCharTok{\%\textgreater{}\%}
  \FunctionTok{pivot\_longer}\NormalTok{(}\AttributeTok{cols =} \FunctionTok{starts\_with}\NormalTok{(}\StringTok{"Age"}\NormalTok{), }\AttributeTok{names\_to =} \StringTok{"AgeGroup"}\NormalTok{, }\AttributeTok{values\_to =} \StringTok{"Rate"}\NormalTok{)}

\FunctionTok{print}\NormalTok{(suicrates\_long)}
\end{Highlighting}
\end{Shaded}

\begin{verbatim}
## # A tibble: 75 x 3
##    Country AgeGroup  Rate
##    <chr>   <chr>    <dbl>
##  1 Canada  Age25.34    22
##  2 Canada  Age35.44    27
##  3 Canada  Age45.54    31
##  4 Canada  Age55.64    34
##  5 Canada  Age65.74    24
##  6 Israel  Age25.34     9
##  7 Israel  Age35.44    19
##  8 Israel  Age45.54    10
##  9 Israel  Age55.64    14
## 10 Israel  Age65.74    27
## # i 65 more rows
\end{verbatim}

\subparagraph{(b) Construct side-by-side box plots for the data from
different age groups, and comment on what the graphic tells us about the
data.}\label{b-construct-side-by-side-box-plots-for-the-data-from-different-age-groups-and-comment-on-what-the-graphic-tells-us-about-the-data.}

\begin{Shaded}
\begin{Highlighting}[]
 \FunctionTok{library}\NormalTok{(ggplot2)}

\CommentTok{\# Constructing side{-}by{-}side box plots}
\FunctionTok{ggplot}\NormalTok{(suicrates\_long, }\FunctionTok{aes}\NormalTok{(}\AttributeTok{x =}\NormalTok{ AgeGroup, }\AttributeTok{y =}\NormalTok{ Rate)) }\SpecialCharTok{+}
  \FunctionTok{geom\_boxplot}\NormalTok{() }\SpecialCharTok{+}
  \FunctionTok{labs}\NormalTok{(}\AttributeTok{title =} \StringTok{"Suicide Rates by Age Group"}\NormalTok{,}
       \AttributeTok{x =} \StringTok{"Age Group"}\NormalTok{,}
       \AttributeTok{y =} \StringTok{"Suicide Rate per 100,000"}\NormalTok{) }\SpecialCharTok{+}
  \FunctionTok{theme\_minimal}\NormalTok{()}
\end{Highlighting}
\end{Shaded}

\includegraphics{3220103172_files/figure-latex/unnamed-chunk-21-1.pdf}

\subsubsection{7.}\label{section-6}

In the data set pressure, the relevant theory is that associated with
the Claudius- Clapeyron equation, by which the logarithm of the vapor
pressure is approximately inversely proportional to the absolute
temperature (temperature + 273). Transform the data in the manner
suggested by this theoretical relationship, plot the data, fit a
regression line, and add the line to the graph. Does the fit seem
adequate? \#\#\#\#\# (1) Load the `pressure' data set

\begin{Shaded}
\begin{Highlighting}[]
 \FunctionTok{data}\NormalTok{(pressure)}
\FunctionTok{head}\NormalTok{(pressure)}
\end{Highlighting}
\end{Shaded}

\begin{verbatim}
##   temperature pressure
## 1           0   0.0002
## 2          20   0.0012
## 3          40   0.0060
## 4          60   0.0300
## 5          80   0.0900
## 6         100   0.2700
\end{verbatim}

\subparagraph{(2) Transform the data}\label{transform-the-data}

According to the Clausius-Clapeyron equation, the logarithm of the vapor
pressure is approximately inversely proportional to the absolute
temperature (temperature + 273). We'll transform the temperature to its
absolute value and compute the logarithm of the vapor pressure:

\begin{Shaded}
\begin{Highlighting}[]
\NormalTok{pressure}\SpecialCharTok{$}\NormalTok{TemperatureK }\OtherTok{\textless{}{-}}\NormalTok{ pressure}\SpecialCharTok{$}\NormalTok{temperature }\SpecialCharTok{+} \DecValTok{273}
\NormalTok{pressure}\SpecialCharTok{$}\NormalTok{log\_pressure }\OtherTok{\textless{}{-}} \FunctionTok{log}\NormalTok{(pressure}\SpecialCharTok{$}\NormalTok{pressure)}
\end{Highlighting}
\end{Shaded}

\subparagraph{(3) Plot the transformed
data}\label{plot-the-transformed-data}

\begin{Shaded}
\begin{Highlighting}[]
\FunctionTok{plot}\NormalTok{(}\DecValTok{1} \SpecialCharTok{/}\NormalTok{ pressure}\SpecialCharTok{$}\NormalTok{TemperatureK, pressure}\SpecialCharTok{$}\NormalTok{log\_pressure,}
     \AttributeTok{xlab =} \StringTok{"1 / Temperature (1/K)"}\NormalTok{,}
     \AttributeTok{ylab =} \StringTok{"Log(Pressure)"}\NormalTok{,}
     \AttributeTok{main =} \StringTok{"Clausius{-}Clapeyron Plot"}\NormalTok{)}
\end{Highlighting}
\end{Shaded}

\includegraphics{3220103172_files/figure-latex/unnamed-chunk-24-1.pdf}
\#\#\#\#\# (4) Fit a regression line Fit a linear model to the
transformed data:

\begin{Shaded}
\begin{Highlighting}[]
\NormalTok{fit }\OtherTok{\textless{}{-}} \FunctionTok{lm}\NormalTok{(log\_pressure }\SpecialCharTok{\textasciitilde{}} \FunctionTok{I}\NormalTok{(}\DecValTok{1} \SpecialCharTok{/}\NormalTok{ TemperatureK), }\AttributeTok{data =}\NormalTok{ pressure)}
\FunctionTok{summary}\NormalTok{(fit)}
\end{Highlighting}
\end{Shaded}

\begin{verbatim}
## 
## Call:
## lm(formula = log_pressure ~ I(1/TemperatureK), data = pressure)
## 
## Residuals:
##       Min        1Q    Median        3Q       Max 
## -0.074183 -0.028889 -0.002508  0.021144  0.151494 
## 
## Coefficients:
##                     Estimate Std. Error t value Pr(>|t|)    
## (Intercept)        1.827e+01  4.429e-02   412.4   <2e-16 ***
## I(1/TemperatureK) -7.301e+03  1.822e+01  -400.7   <2e-16 ***
## ---
## Signif. codes:  0 '***' 0.001 '**' 0.01 '*' 0.05 '.' 0.1 ' ' 1
## 
## Residual standard error: 0.04874 on 17 degrees of freedom
## Multiple R-squared:  0.9999, Adjusted R-squared:  0.9999 
## F-statistic: 1.606e+05 on 1 and 17 DF,  p-value: < 2.2e-16
\end{verbatim}

\subparagraph{(5) Add the regression line to the
plot}\label{add-the-regression-line-to-the-plot}

\subparagraph{Here's the complete R code for the entire
process:}\label{heres-the-complete-r-code-for-the-entire-process}

\begin{Shaded}
\begin{Highlighting}[]
\CommentTok{\# Load the pressure data set}
\FunctionTok{data}\NormalTok{(pressure)}
\FunctionTok{head}\NormalTok{(pressure)}
\end{Highlighting}
\end{Shaded}

\begin{verbatim}
##   temperature pressure
## 1           0   0.0002
## 2          20   0.0012
## 3          40   0.0060
## 4          60   0.0300
## 5          80   0.0900
## 6         100   0.2700
\end{verbatim}

\begin{Shaded}
\begin{Highlighting}[]
\CommentTok{\# Transform the data}
\NormalTok{pressure}\SpecialCharTok{$}\NormalTok{TemperatureK }\OtherTok{\textless{}{-}}\NormalTok{ pressure}\SpecialCharTok{$}\NormalTok{temperature }\SpecialCharTok{+} \DecValTok{273}
\NormalTok{pressure}\SpecialCharTok{$}\NormalTok{log\_pressure }\OtherTok{\textless{}{-}} \FunctionTok{log}\NormalTok{(pressure}\SpecialCharTok{$}\NormalTok{pressure)}

\CommentTok{\# Plot the transformed data}
\FunctionTok{plot}\NormalTok{(}\DecValTok{1} \SpecialCharTok{/}\NormalTok{ pressure}\SpecialCharTok{$}\NormalTok{TemperatureK, pressure}\SpecialCharTok{$}\NormalTok{log\_pressure,}
     \AttributeTok{xlab =} \StringTok{"1 / Temperature (1/K)"}\NormalTok{,}
     \AttributeTok{ylab =} \StringTok{"Log(Pressure)"}\NormalTok{,}
     \AttributeTok{main =} \StringTok{"Clausius{-}Clapeyron Plot"}\NormalTok{)}

\CommentTok{\# Fit a regression line}
\NormalTok{fit }\OtherTok{\textless{}{-}} \FunctionTok{lm}\NormalTok{(log\_pressure }\SpecialCharTok{\textasciitilde{}} \FunctionTok{I}\NormalTok{(}\DecValTok{1} \SpecialCharTok{/}\NormalTok{ TemperatureK), }\AttributeTok{data =}\NormalTok{ pressure)}
\FunctionTok{summary}\NormalTok{(fit)}
\end{Highlighting}
\end{Shaded}

\begin{verbatim}
## 
## Call:
## lm(formula = log_pressure ~ I(1/TemperatureK), data = pressure)
## 
## Residuals:
##       Min        1Q    Median        3Q       Max 
## -0.074183 -0.028889 -0.002508  0.021144  0.151494 
## 
## Coefficients:
##                     Estimate Std. Error t value Pr(>|t|)    
## (Intercept)        1.827e+01  4.429e-02   412.4   <2e-16 ***
## I(1/TemperatureK) -7.301e+03  1.822e+01  -400.7   <2e-16 ***
## ---
## Signif. codes:  0 '***' 0.001 '**' 0.01 '*' 0.05 '.' 0.1 ' ' 1
## 
## Residual standard error: 0.04874 on 17 degrees of freedom
## Multiple R-squared:  0.9999, Adjusted R-squared:  0.9999 
## F-statistic: 1.606e+05 on 1 and 17 DF,  p-value: < 2.2e-16
\end{verbatim}

\begin{Shaded}
\begin{Highlighting}[]
\CommentTok{\# Add the regression line to the plot}
\FunctionTok{abline}\NormalTok{(fit, }\AttributeTok{col =} \StringTok{"red"}\NormalTok{)}
\end{Highlighting}
\end{Shaded}

\includegraphics{3220103172_files/figure-latex/unnamed-chunk-26-1.pdf}

Note that the \texttt{echo\ =\ FALSE} parameter was added to the code
chunk to prevent printing of the R code that generated the plot.

\end{document}
